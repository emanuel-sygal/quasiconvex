% \usepackage{pgfplots}
% \usetikzlibrary{calc}
% \usepgfplotslibrary{colormaps}


\RequirePackage[l2tabu, orthodox]{nag}
\documentclass[12pt]{article}
\usepackage{geometry}                % See geometry.pdf to learn the layout options. There are lots.
\geometry{letterpaper}
\usepackage[autostyle=false, style=english]{csquotes}
\usepackage{mathtools}
%\MakeOuterQuote{"}
\setlength{\marginparwidth}{2cm}

\usepackage{nomencl}
\makenomenclature
% Run the following command in terminal to update:
% makeindex "main.nlo" -s nomencl.ist -o "main.nls"
\nomenclature{$f, f_c$}{The map $z \mapsto z^2+c.$}
\nomenclature{$\mathcal J$}{The Julia set of $f$.}
\nomenclature{$K$}{The filled Julia set of $f$.}
\nomenclature{$\gamma_{z_1, z_2}$}{The track connecting $z_1$ and $z_2$. It can be either angular (\enquote{peripheral}) or radial (\enquote{express}).}
\nomenclature{$\eta_{z_1, z_2}$}{The itinerary connecting two points. When $z_1$ and $z_2$ are stations, this is the same as $\gamma_{z_1,z_2}$.}
\nomenclature{$A_\infty(f_c)$}{The exterior of the Julia set of $f_c$. The complement of $K_c$.}
\nomenclature{$\mathrm{Exterior}(\mathcal{J})$}{ $\;$ An Alternative notation for $A_\infty(f_c)$.}
\nomenclature{$\psi$}{The Böttcher coordinate $\mathbb D ^* \to \Exterior(\mathcal J)$ conjugating $f_0$ and $f$.}
\nomenclature{$s_{n,k}$}{A station in $\mathbb D^*$ or its image under $\psi$.}
\nomenclature{$\Delta (\gamma, \mathcal P)$}{The relative distance to the post-critical set.}
\nomenclature{$I_{n}$}{The $n$-th departure set.}
\nomenclature{$\ell_n$ }{$\Length([s_{n},s_{n+1}]) = s_{n,0}-s_{n+1,0}.$}
\nomenclature{$\alpha_n$}{the union of the two outermost tracks emanating from the station $s_{n,0}$.}


% Prevents line breaks at math inline
\relpenalty=9999
\binoppenalty=9999

\usepackage{graphicx,color,mathtools}
\usepackage{amssymb,amsmath,amsthm,mathrsfs}
%\usepackage[all,cmtip]{xy}
\usepackage{comment}
%\usepackage{todonotes}
\usepackage{enumitem}   
\usepackage{tikz} 
\usepackage{pgfplots}
\pgfplotsset{compat=1.18}
\usepackage{graphicx}
\usepackage{xcolor}

\usepackage{hyperref}
\usepackage[capitalize,nameinlink,noabbrev]{cleveref}

\usepackage[pdftex,bookmarks,pdfnewwindow,plainpages=false,unicode,pdfencoding=auto]{}


\DeclareGraphicsRule{.tif}{png}{.png}{`convert #1 `dirname #1`/`basename #1 .tif`.png}
\linespread{1.2}

\numberwithin{equation}{section}

\newtheorem{theorem}{Theorem}[section]
\newtheorem{conjecture}[theorem]{Conjecture}
\newtheorem{lemma}[theorem]{Lemma}
\newtheorem{claim}[theorem]{Claim}
\newtheorem{proposition}[theorem]{Proposition}
\newtheorem{corollary}[theorem]{Corollary}

\theoremstyle{remark}
\newtheorem*{remark}{Remark}

\theoremstyle{definition}
\newtheorem{definition}[theorem]{Definition}
\newtheorem{example}[theorem]{Example}

\DeclareMathOperator{\crit}{crit}

\DeclareMathOperator{\Hdim}{H.dim }
\DeclareMathOperator{\supp}{supp }
\DeclareMathOperator{\diam}{diam}
\DeclareMathOperator{\BV}{BV}
\DeclareMathOperator{\shadow}{Shadow}
\DeclareMathOperator{\idiam}{i.diam}
\DeclareMathOperator{\aut}{Aut}
\DeclareMathOperator{\hol}{Hol}
\DeclareMathOperator{\hyp}{hyp}
\DeclareMathOperator{\dist}{dist}
\DeclareMathOperator{\area}{Area}
\DeclareMathOperator{\loc}{loc}
\DeclareMathOperator{\Mod}{Mod}
\DeclareMathOperator{\Arg}{Arg}
\DeclareMathOperator{\Length}{Length}
\DeclareMathOperator{\HypLength}{HypLength}
\DeclareMathOperator{\CriticalOrbit}{CriticalOrbit}
\DeclareMathOperator{\Postcrit}{\mathcal P}
\DeclareMathOperator{\PostCrit}{\mathcal P}
\DeclareMathOperator{\Closure}{Closure}
\DeclareMathOperator{\RadialSuccessor}{RadialSuccessor}
\DeclareMathOperator{\Exterior}{Exterior}

\global\long\def\R{\mathbb{R}}%
\global\long\def\C{\mathbb{C}}%
\global\long\def\N{\mathbb{N}}%
\global\long\def\Z{\mathbb{Z}}%
\global\long\def\D{\mathbb{D}}%
\global\long\def\H{\mathbb{H}}%
\global\long\def\do#1#2{\frac{\partial#1}{\partial#2}}%
\global\long\def\nor#1{\left|#1\right|}%
\global\long\def\o#1{\overline{#1}}%
\global\long\def\norm#1{\left\Vert #1\right\Vert }%
\global\long\def\\sphere{\hat{\mathbb{C}}}%
\global\long\def\ep{\varepsilon}%
\global\long\def\mp#1{\left\langle #1\right\rangle }%
\global\long\def\lap{\triangle}%
\global\long\def\dz{\mathrm{\mathrm{dz}}}%
\global\long\def\dt{\mathrm{\mathrm{dt}}}%
\global\long\def\dtheta{\mathrm{\mathrm{d\theta}}}%
\global\long\def\dx{\mathrm{\mathrm{dx}}}%
\global\long\def\dw{\mathrm{\mathrm{dw}}}%
\global\long\def\T{\mathrm{\mathbb{T}}}%
\begin{document}

\section{Introduction}
A domain $\Omega\subseteq\C$ is called \textbf{quasiconvex }if its
intrinsic metric is comparable to the ambient Euclidean metric. Explicitly,
this means that there exists a constant $A\geq1$ such that every
two points $z_{1},z_{2}\in\Omega$ have a rectifiable path $\gamma:\left[0,1\right]\to\Omega$
connecting them which satisfies
\[
\Length(\gamma)\leq A\cdot\left|z_{1}-z_{2}\right|.
\]
We call such a path $\gamma$ a \emph{quasiconvexity certificate for
$z_{1}$ and $z_{2}$}.
% or "witness"

If $\Omega$ is the interior of a Jordan curve, then by \cite[Corollary F]{hakobyan_euclidean_2008}
it is enough to find certificates for points $z_{1},z_{2}$ that are
on the boundary curve $\partial\Omega$.%
\begin{comment}
It is also shown in \cite{hakobyan_euclidean_2008} that any quasidisk is quasiconvex.
\end{comment}

The \emph{cauliflower} is the filled Julia set of the map $z^2+\frac 14$.
We show that its complement, $\mathrm{Exterior}(\mathcal{J}(z^{2}+1/4))$,
is quasiconvex. We then adapt our argument to establish that the exterior of the developed deltoid is quasiconvex.

One motivation to study quasiconvexity stems from its connection with the
John property: If $\Omega$ is a quasiconvex Jordan domain, then its complement has a John interior. See \cite[Corollary 3.4]{hakobyan_euclidean_2008} for a proof.
Thus this result is a strengthening of \cite[Theorem 6.1]{carleson_julia_1994}, in which it is shown directly that the cauliflower is a John domain.

This result also has a function-theoretic interpretation: By \cite[Theorem 1.1]{koskela_geometric_2010}, it shows that the cauliflower is a BV-extension domain.
% A bounded, simply connected planar domain is a BV-extension domain if and only if its complement is quasiconvex.

\begin{comment}
The Filled Julia set of $z^{2}+1/4$, called the cauliflower, has
an inward-pointing cusp and hence is not quasiconvex.
\end{comment}

\begin{comment}
Thus, for any $c$ in 

For values $c$ in quadratic polynomials $f_{c}(z)=z^{2}+c$
\end{comment}
\begin{comment}
If $f_{c}$ has an attracting fixed point then its Julia set $\mathcal{J}(f_{c})$
is a quasicircle, hence its interior and exterior are both quasiconvex.
This is the case for values of $c$ in the main cardioid of the Mandelbrot
set.
, i.e.\ for 
\[
c\in\left\{ -\frac{\lambda}{2}-\frac{\lambda^{2}}{4}:\,\left|\lambda\right|<1\right\} .
\]
\end{comment}

\subsection{Sketch of the argument}

We show quasiconvexity by an explicit construction of a certificate connecting two given points on the Julia set.

We first build certificates for the exterior unit disk $\mathbb D ^*$ and then transport them by the Böttcher coordinate $\psi$ of $f_{1/4}$ to the exterior of the cauliflower.

In order to retain control on the certificates after applying $\psi$, we build the certificates on $\mathbb D^*$ in a manner invariant under the map $f_0: z\mapsto z^2$. This is done by only traveling along the boundaries of Carleson boxes in $\mathbb D^{*}$. 

The image of a certificate $\eta$ in $\mathbb D^{*}$ under the conjugacy $\psi$ is invariant under $f_{1/4}$. We use this invariance to show that $\psi(\eta)$ is indeed a certificate, by employing a parabolic variant of the principle of the conformal elevator: We repeatedly apply $f_{1/4}$ on $\psi(\eta)$ until either the distance between the endpoints grows to a definite size or one endpoint becomes sufficiently close to the parabolic fixed point $1/2$.

To facilitate the reading, we first demonstrate the proof in the hyperbolic case of maps $f_c$ where  $c\in\left(-\frac 34,\frac{1}{4}\right)$. In this case the usual conformal elevator applies. We then treat the case of $c=\frac 14$.

% Although $J_{0.1}^{\text{exterior}}$ is known in advance to be quasiconvex, in virtue of being a quasidisk.


\section{The exterior disk}

\begin{comment}
The exterior $\D^{*}=\left\{ \left|z\right|>1\right\} $ of the unit
disk is trivially quasiconvex by connecting points along the perimeter of the circle. However, these paths follow the boundary closely and their length would blow up if we transport them to the exterior of $\mathcal{J}(f_{c})$, $c\neq0$, via the Riemann map. Instead,
\end{comment}

We connect boundary points by moving along the boundaries of Carleson boxes which we now define.
\begin{definition}
Let $n\in\N_{0}$ and $k\in\left\{ 0,\ldots,2^{n}-1\right\} $. We call the set
%\[B_{n,k}=\exp\left(\biggl(2^{-n-1}\log2,\,2^{-n}\log2\biggl]\times\biggl(\frac{k}{2^{n}} \cdot 2\pi,\,\frac{k+1}{2^{n}} \cdot 2\pi\biggl]\right)\]

\[
B_{n,k}=\left\{ z:\quad\left|z\right|\in\biggl(2^{1/2^{n+1}}, 2^{1/2^{n}} \biggl],\qquad\mathrm{arg}(z)\in\biggl(\frac{k}{2^{n}} \, 2\pi,\frac{k+1}{2^{n}} \, 2\pi\biggl]\right\} 
\]
a \emph{Carleson box}\,.
Observe that for a fixed $n$, the union $\bigsqcup_{k=0}^{2^{n}-1}B_{k,n}$
is a partition of the annulus 
\[
\left\{ 2^{1/2^{n+1}} <\left|z\right|\leq 2^{1/2^{n}} \right\} 
\]
 into $2^{n}$ equally-spaced sectors.
 
The \emph{Carleson box decomposition} is the partition of $\D^{*}$ into Carleson
boxes:
\[
\D^{*}=\left\{ \zeta:\,\left|\zeta\right|>2\right\} \sqcup\bigsqcup_{n=0}^{\infty}\bigsqcup_{k=0}^{2^{n}-1}B_{n,k}.
\]
The crucial property of this decomposition is its invariance under $f_{0}$,
stemming from the relation
\begin{equation*}
f_{0}\left(B_{n+1,k}\right)=B_{n,k \,(\operatorname{mod} \,2^n)}.
\end{equation*}
\end{definition}

We describe the motion along Carleson boxes using the metaphor of a passenger who travels by trains. 
We now define \enquote{stations} and \enquote{tracks}.

\begin{definition}
A \emph{terminal} is a point $\zeta \in \partial \mathbb D^*$ on the unit circle.
The \emph{central station} is the point\emph{ $s_{0,0}=2$. Stations
}are the iterated preimages of the central station under the map $f_{0}:\zeta\mapsto \zeta^{2}$.
We index them as 
\[
s_{n,k}=2^{1/2^{n}} \exp\left(\frac{k}{2^{n}}2\pi i\right),\qquad n\in\N_{0},\quad k\in\left\{ 0,\ldots,2^{n}-1\right\} .
\]
Stations are naturally structured in a binary tree, where the root is the central station and the children of a node are its preimages. The \emph{generation} of a station $s_{n,k}$ is its level $n$ in this tree. The $2^{n}$ stations of generation $n$ in the tree are equally spaced on the circle $C_{n}=\left\{ \left|\zeta\right|=2^{1/2^{n}}\right\} $. 
\end{definition}

We next lay two types of \enquote{rail tracks} on the boundaries of Carleson boxes, which we use to travel between stations.

\begin{definition}
Let $s=s_{n,k}$ be a station.

1. The \emph{peripheral neighbors} of $s$ are the two stations $s_{n,\left(k\pm1\right)\mod 2^{n}}$ adjacent to $s_{n,k}$ on $C_{n}$.

2. The \emph{peripheral track }$\gamma_{s,s'}$ from $s$ to a peripheral neighbor $s'$
is the short arc of the circle $C_{n}$ connecting $s$ to $s'$.

3. The \emph{radial successor} of $s$ is $\RadialSuccessor(s)=s_{n+1,2k}$, the unique station of generation $n+1$ on the radial segment $[0,s]$.

4. The \emph{Express track} $\gamma_{s,s'}$ from $s$ to its radial successor $s''$ is the radial segment $[s,s']$.

%5. An \emph{itinerary} is a sequence of stations $\left(\sigma_{0},\sigma_{1},\ldots\right)$. We identify an itinerary with the curve obtained by traveling along it.
\end{definition}

Notice that the tracks preserve the dynamics: applying $f_0$ to a peripheral track between stations $s,s'$ gives a peripheral track between the parents of $s,s'$ in the tree, and likewise for an express track.

When a passenger travels between two stations $s_1$ and $s_2$, they must follow a particular itinerary from $s_1$ to $s_2$.
If $s_1$ is the central station, then this itinerary is determined by the rule that the passenger stays as close as possible to its destination $s_2$ in the peripheral distance. 
This also determines how to travel from the central station to a terminal $\zeta\in \partial \mathbb D^*$, by continuity. See \cref{fig:Carleson1} and the next definition.

% There is a caveat in the analogy, because this gives rise to an itinerary between any two points and if we approximate terminal points by bearby stations then we do not get the same path.

\begin{definition}
Let $\zeta=\exp(2\pi i\theta)\in\partial\D$. The \emph{central itinerary} of $\zeta$ is a path $\eta_\zeta = \gamma _{\sigma_0,\sigma_1} + \gamma_{\sigma_1,\sigma_2}+\ldots$ from the central station to $\zeta$, made of tracks between stations $\sigma_0,\sigma_1,\ldots$. It is defined inductively as follows:

Start at the main station $\sigma_0=s_{0,0}$. Suppose that we already chose $\sigma_0,\ldots,\sigma_k$. If there is a peripheral neighbor $\sigma$ of $\sigma_k$ that is closer peripherally to $\zeta$, meaning that $$\left|\Arg\left(\zeta\right)-\Arg\left(\sigma\right)\right|
< \left|\Arg\left(\zeta\right)-\Arg\left(\sigma_{k}\right)\right|,$$ then take $\sigma_{k+1}=\sigma$. Otherwise, take 
$\sigma_{k+1}=\RadialSuccessor(\sigma)$.
\end{definition}

\input{"marker path.tex"}

We identify $\eta_{\zeta}$ with its sequence of stations $(\sigma_0,\ldots)$. We record two properties of central itineraries:

\begin{itemize}
	\item There are no two consecutive peripheral tracks in $\eta_{\zeta}$ and thus
	\begin{equation} \label{generation-lower-bound}
		\operatorname{Generation}(\sigma_k)\geq \frac k2.
	\end{equation}
%	where as before the generation of $s_{n,k}$ is defined to be $n$.
	
	\item Central itineraries are essentially invariant under $f_{0}$, in the sense that
	\begin{equation*}
		f_{0}(\eta_{\zeta})=\eta{}_{f_{0}(\zeta)}\cup[s_{0,0},f_0(s_{0,0})]
	\end{equation*}
	for every $\zeta\in \partial \mathbb D^*$.
\end{itemize}

\begin{lemma} \label{track_decay}
Given $\zeta\in \partial \mathbb D^*$, decompose the central itinerary $\eta_{\zeta}$ into its constituent tracks, 
$$\eta_{\zeta}=\gamma _1 + \gamma_2 + \dots \; .$$ 
	The lengths of $\gamma_k$ decay exponentially:
	$$\Length(\gamma_{k})\lesssim \theta^{k},$$ uniformly in $\zeta$, for some constant $\theta<1$.
In particular, the total length of $\eta_\zeta$ is bounded above by a definite constant independent of $\zeta$.
\end{lemma}
\begin{proof}
	The radial distances have size 
	$$2^{1/2^n}-2^{1/2^{n+1}} = 2^{1/2^{n+1}} \sum_{k=1}^{\infty} \binom{1/2^{n+1}}{k} \asymp 2^{-n}.$$
	By \eqref{generation-lower-bound}, the radial tracks of $\eta_\zeta$ satisfy the required bound with $\theta = \sqrt 2$. To conclude, note that a peripheral track of generation $n$ has length $\asymp 2^{-n}$ and reuse \eqref{generation-lower-bound}.
\end{proof}

%In particular, the length of a suffix $\sigma_{k}+\sigma_{k+1}+\ldots$ decays exponentially in $k$, uniformly in $\zeta$.

%Note that this may result in an "inefficient" itinerary: two stations that are peripheral neighbors might have a rather long itinerary between them.
%By continuity, this defines the itinerary between any two terminals $\zeta_1$ and $\zeta_2$:

% If two terminals are bearby then their common ancestor $N$ must be large, because if $N$ is small then the points must be far away.*/-+

\begin{definition} \label{def-disk-itinerary} Given two distinct terminals $\zeta_{1},\zeta_{2}\in\partial\D^*$, form the central itineraries $\eta_{\zeta_{1}}=\left(\sigma_{n}^{1}\right)_{n=0}^{\infty}$ and 
	$\eta_{\zeta_{2}}=\left(\sigma_{n}^{2}\right)_{n=0}^{\infty}$
	 and let  $\sigma=\sigma^1_i=\sigma^2_j$ be the last station that is in both $\eta_{\zeta_{1}}$ and $\eta_{\zeta_{2}}$. %Note that $\sigma$ is well-defined.
	 We define the \emph{itinerary} between  $\zeta_{1}$ and $\zeta_{2}$ to be the path  %$\eta_{\zeta_{i}}^{\text{truncated}}=\left(\sigma_{N},\sigma_{N+1}^{i},\ldots\right)$ be the truncated paths. Then
 \begin{gather*}
 \eta_{\zeta_{1},\zeta_{2}}=  \left(\dots,\sigma_{i+2}^{1},\sigma_{i+1}^{1},\sigma,\sigma_{j+1}^{2},\sigma_{j+2}^{2},\dots\right).
 \end{gather*}
	This is a simple bi-infinite path connecting $\zeta_{1}$ and $\zeta_{2}$, see \cref{fig:Carleson2}. Note that itineraries are equivariant under the dynamics: we have  \begin{equation}
		f(\eta_{\zeta_1,\zeta_2})=\eta_{f(\zeta_1),f(\zeta_2)}
	\end{equation} for every pair of terminals $\zeta_1,\zeta_2 \in \partial \mathbb D^*$.
	
	
\end{definition}
\input{"marker path 2.tex"}

\begin{theorem} \label{quasiconvex disk}
The domain $\D^{*}$ is quasiconvex with the itineraries $\eta_{\zeta_1,\zeta_2}$ as certificates.
\end{theorem}

\begin{proof}
We decompose the itinerary into two paths, so that
\begin{equation}
\Length(\eta_{\zeta_{1},\zeta_{2}})=\Length\left(\sigma,\sigma_{j+1}^{2},\sigma_{j+2}^{2},\dots\right)
+\Length\left(\sigma,\sigma_{i+1}^{1},\sigma_{i+2}^{1},\dots\right),
\end{equation}
and bound each summand using \cref{track_decay}. Denoting $\operatorname{Generation}(\sigma)=n$, we obtain
\begin{align*}
\Length(\eta_{\zeta_{1},\zeta_{2}})
&\lesssim 2 \sum_{k=n}^{\infty}\frac{1}{2^{k}} 
\lesssim2^{-n}. \label{eq:1}
\end{align*}
We conclude by noticing that
\begin{align*}
\left|\zeta_{1}-\zeta_{2}\right|
&\asymp
\left|\Arg\left(\zeta_{1}\right)-\Arg\left(\zeta_{2}\right)\right|\\
&\geq \frac {2\pi}{2^{n+2}}\\
&\gtrsim 	\Length\left(\eta_{\zeta_{1},\zeta_{2}}\right).
\end{align*}
%Suppose two stations diverge at generation $N$ and never meet after that. Then we can bound the distance between the endpoints from below. This is not quite as trivial as it sounds but it is correct.
\end{proof}

	
\section{Transporting the Rails} \label{rails-section}
Let $c\in\left[-\frac 34,\frac{1}{4}\right]$ and denote by $\psi$ the Böttcher coordinate of $f: z\mapsto z^2+c$ at infinity. 
This means that $\psi$ is the unique conformal map $\D^{*}\to\mathrm{Exterior}(\mathcal{J})$  which fixes $\infty$ and satisfies the conjugacy $$f\circ\psi=\psi\circ f_{0}.$$
Since the Julia set $\mathcal J$ is a Jordan curve, the map $\psi$ extends to a homeomorphism between the circle $\partial\D$ and the Julia set $\mathcal{J}(f)$ by Carathéodory's
theorem.

We apply $\psi$ to the rails that we constructed in $\mathbb D^*$ to obtain the corresponding rails in $\mathrm{Exterior}(\mathcal{J})$:
\begin{definition} \leavevmode
\begin{enumerate}
	\item The 	\emph{stations} of $f_c$ are the points  $s_{n,k,c}=\psi(s_{n,k})$.


\item The \emph{$c$-tracks} are the curves of the form $\psi \left(\gamma_{s,s'}\right)$, where $\gamma_{s,s'}$ is a track. They are classified as express or peripheral according to the corresponding classification of $\gamma_{s,s'}$. 
Express tracks lie on \emph{external} rays of the filled Julia set $\mathcal K$, while peripheral tracks lie on the equipotentials of $\mathcal K$.

\item Let $z_1,z_2\in \mathcal J$ and let $\zeta_i=\psi^{-1}(z_i)$ be the corresponding points on $\partial \mathbb D^*$. 
The \emph{$c$-itineraries} are $\eta_{z_1,z_2}=\psi(\eta_{\zeta_1,\zeta_2})$.
\end{enumerate}

We omit $c$ from the notation for ease of reading. It will be clear from the context whether we work in $\mathbb D^*$ or in $\mathrm{Exterior}(\mathcal J)$.

%These itineraries can equivalently be obtained directly as in the case of $\mathbb D^*$, in terms of following the central itineraries from the common ancestor in the tree structure.
%Trivially since $\psi$ is a bijective correspondence between the two decompositions.
\end{definition}

Note that $\psi\left((1,\infty)\right)\subseteq\R$ since $\mathcal{J}$ is symmetric with respect to the real line. In particular $\psi(s_{0,0})\in\R$, i.e.\ the $c$-central station is real.

\begin{comment}
\begin{proof}
Formally, $\overline{\psi}(\overline{z})$
	is another conformal conjugacy between $f$ and $f_0$ which fixes infinity, so by uniqueness of the Böttcher coordinate we obtain $\psi(z)=\overline{\psi}(\overline{z})$,
	hence $\psi(z)\in\R$ for $z\in\R$.
\end{proof}
\end{comment}

\section{Hyperbolic Maps}

A rational map is \emph{hyperbolic} if under iteration, every critical point converges to an attracting cycle.
Hyperbolic maps enjoy the principle of the conformal elevator, which roughly says that any ball centered on the Julia set can be blown up to a definite size. More precisely, we have the following:

\begin{proposition}[The Principle of the Conformal Elevator] \label{elevator}
	Let $f$ be a hyperbolic rational map, $z\in \mathcal J$ be a point on the Julia set of $f$ and $r>0$. There exists some forward iterate $f^{\circ n}$ of $f$ which is injective on the ball $B(z,2r)$ such that
	$\diam {f^{\circ n}(B(z,r))}$ is bounded below uniformly in $z$ and $r$. 
\end{proposition}

%See \cite{bonk_quasisymmetries_2016} for a stronger version, details and a proof.

\begin{corollary} \label{elevator for points on julia}
	Let $f$ be a hyperbolic rational map. There exists  $\epsilon > 0$ such that every pair of points $z,w\in\mathcal{J}(f)$ has a forward iterate $f^{\circ n}$ for which $\left|f^{\circ n}(z)-f^{\circ n}(w)\right|>\epsilon$.	
\end{corollary}

\begin{comment}
\begin{proof}
	Apply \cref{elevator} to a ball centered on the Julia set which contains $z,w$ on its boundary at roughly antipodal points. After blowing up we get points $f^{\circ n}(z),f^{\circ n}(w)$ which are a definite distance apart by Koebe's distortion theorem. %\todo{make this correct}
\end{proof}
\end{comment}

We are now ready to show the analogue of \cref{quasiconvex disk}:
%\todo{We use more than hyperbolicity, also that the Julia set is a Jordan domain}
\begin{theorem}  \leavevmode
\begin{enumerate}[label=\normalfont(\roman*)]

\item Given $z\in\mathcal{J}$ decompose its central itinerary into tracks, 
\begin{equation*}
\eta_z = \gamma _1 +\gamma_2 +\dots.
\end{equation*}
We have the estimate
\begin{equation*}
\Length(\gamma_{k})\lesssim\theta^{-k},
\end{equation*}
uniformly in $z$, for some constant $\theta=\theta(c)>1$. In particular, any point on \(\mathcal{J}\) can be reached from $s_{0,0}$ by a curve of bounded length.

\item The domain $\mathrm{Exterior}(\mathcal{J})$ is quasiconvex with the itineraries $\eta_{z_1,z_2}$ as certificates.
	\end{enumerate}
\end{theorem}

\begin{proof} \leavevmode
\begin{enumerate}[label=\normalfont(\roman*)]
\item The map $f$ has some iterate $f^{\circ N}$ such that $|(f^{\circ N})'(z)|>1$ for all $z \in \mathcal{J}$.
By the compactness of $\mathcal J$, there is a $\theta >1$ such that $\left|(f^{\circ N})'\right (z)|>\theta$ on some neighborhood $\mathcal{U}$ of $\mathcal{J}$.
Since every itinerary is eventually contained in $\mathcal{U}$, for almost all itineraries $\gamma$ we have 
\begin{equation*}
\Length ( f^{\circ N}(\gamma) ) \geq \theta \cdot \Length( \gamma). 
\end{equation*}

The peripheral tracks on circles $C_n$ of index $n \equiv k$ modulo $N$ have a total length bounded by a geometric series of rate $\theta$, hence finite. 
The lengths of the express tracks can be bounded in the same way. % We conclude that the total length of the itinerary is bounded.

\item Since we already know that the lengths of tracks in the itinerary decay exponentially with rate $\theta>1$, the same proof of the case $c=0$ also shows quasiconvexity in this case.

We give a second proof, relying on \cref{elevator for points on julia}. This proof will better prepare us for the parabolic $c=1/4$ case, where we don't have uniform expansion of $f$ on the Julia set.

By \cref{elevator for points on julia}, there exists an $\epsilon>0$ such that any two points are $\epsilon$-apart under some iterate $f$. 
Let $z_{1},z_{2}\in\mathcal{J}(f)$. If $\left|z_{1}-z_{2}\right|\geq\epsilon,$ we are done since the length of $\eta_{z_1z,_2}$ is bounded above by part (i).
% we may just concatenate $\eta_{z_{1}}$ and $\eta_{z_{2}}$ and absorb this bounded length into the quasiconvexity constant

\begin{comment}
Explicitly, if $\Length\left(\eta_{z}\right)\leq L$
for all $z\in\mathcal{J}$ then we take $A\geq\frac{2L}{\epsilon}$
and then automatically $\Length\left(\eta_{z_{1}}+\eta_{z_{2}}\right)\leq A\left|z_{1}-z_{2}\right|$.
\end{comment}

On the other hand, if $\left|z_{1}-z_{2}\right|<\epsilon$, then
we may use  \cref{elevator for points on julia} to find an iterate $f^{\circ n}$ such that 
\begin{equation}
	|w_1-w_2|:=\left|f^{\circ n}(z_{1})-f^{\circ n}(z_{2})\right|\geq\epsilon.
\end{equation}
Koebe's distortion theorem implies that 
\begin{equation}
	\frac{\Length\left(\eta_{z_{0},z_{1}}\right)}{|z_1-z_2|}\asymp\frac{\Length\left(\eta_{w_0,w_1}\right)}{|w_1-w_2|}.
\end{equation}
Since the itineraries $\eta_{w_1,w_2}$ are certificates, the original itineraries $\eta_{z_1z,_2}$ are also certificates. 
\begin{comment}
By a distortion estimate
\begin{equation*}
\Length\left(\eta_{z_{0},z_{1}}\right)\asymp\frac{\Length\left(\eta_{w_0,w_1}\right)}{\left|\left(f^{\circ n}\right)'\left(\zeta\right)\right|}
\end{equation*}
 for some point $\zeta \in \mathcal{J}$. The denominator grows
with $n$ exponentially at rate $\theta$, while the numerator has
a bound of the form 
\[
\Length\left(\eta_{w_1,w_2}\right)\lesssim\left|w_1-w_2\right|\lesssim\theta^{n}\left|z_{1}-z_{2}\right|.
\]
Altogether 
\[
\Length\left(\eta_{z_{1},z_{2}}\right)\lesssim\frac{\theta^{n}\left|z_{1}-z_{2}\right|}{\theta^{n}}=\left|z_{1}-z_{2}\right|
\]
 so $\eta_{z_{1},z_{2}}$ is a quasiconvexity certificate.
\end{comment}

\end{enumerate}
\end{proof}

\section{The Cauliflower}
In this section $c=\frac 14$ and $f=f_{1/4}: z\mapsto z^2+ \frac 14$.
%The cauliflower has cusps at the landing points of the dyadic external rays, i.e. the points whose central itinerary has only finitely-many peripheral tracks.
% We use  the same certificates as in the hyperbolic case, but now it is trickier to show that they work.
Our goal is to prove the quasiconvexity of $\mathrm{Exterior}(\mathcal{J})$, \cref{quasiconvex-cauliflower}. This parabolic case is more complicated than the hyperbolic case because the postcritical set $\PostCrit$ of $f$ has an accumulation point $p=\frac 12$ on $\mathcal J$.

\subsection{Itineraries have finite length}
We first show that each itinerary $\eta_{z_1,z_2}$ has finite length. We will in fact show an exponential decay of the lengths of the constituent tracks. For this to hold it is necessary to glue together consecutive express tracks: for example, the tracks of the central itinerary $\eta_{\frac 12}$ have only a quadratic rate of length decay.

\begin{definition}
	The \emph{reduced decomposition} of an itinerary $\eta$ is the unique decomposition $\eta=\eta_1+\dots+\eta_k$ where each $\eta_i$ is a concatenation of express tracks followed by a single peripheral track.
\end{definition}

\begin{comment}
The following definition is used in the proof of \cref{prop:finite-length}.
\begin{definition}
	The \emph{shadow} of a station $s$ is the set of all terminals $z\in \mathcal J$ that are the endpoints of itineraries starting at $s$.
\end{definition}

Using this definition, we claim:
\end{comment}

\begin{proposition} \label{prop:finite-length}
	Let $z \in \mathcal J$, and let $\eta_z= \gamma_1 + \delta_1 + \dots  $ be the reduced decomposition of its itinerary. Then $\Length(\gamma_j)\lesssim \theta^j$ and $\Length(\delta_j)\lesssim \theta^j$ for some $\theta < 1$. In particular, $\Length(\eta_z)<\infty$ and all points $z\mathcal \in \mathcal J$ are accessible from the main station.
\end{proposition}

For the proof, call $s_{-1}:=s_{1,1}$ the \emph{pre-main station} and let $\mathcal U_{-1}$ be the Jordan domain enclosed by the unit circle, the rightmost itinerary starting from the pre-main station and the leftmost one. This domain was constructed so that it contains all itineraries that start at the pre-main station. 
\begin{comment}
We call an itinerary \emph{non-positive} if it is not contained in the positive real axis.
\end{comment}

\begin{lemma} \label{lemma-enough-visits-of-premain_station}
	Let $\gamma = \gamma_1 + \delta_1 + \dots $ be the reduced decomposition of an itinerary $\gamma$, and assume that $\gamma$ is not contained in the positive real axis. Then for every $k \geq 1$ there are $k$ iterates $1  \leq n_1 < \dots < n_k \leq n_\gamma$ such that $f^{\circ {n_i}}(\gamma_k) \subset \mathcal U_{-1}$.
\end{lemma}

\begin{proof}[Proof (\cref{lemma-enough-visits-of-premain_station})]
	Every station $s \not \in [0,\infty)$ is a preimage of a station on the negative real axis. Thus every express track $\gamma_i$ which is not contained in $[\tfrac 12, \infty)$ has a first iterate $n_i$ such that $f^{\circ n_i}(\gamma_i)$ is contained in the negative real axis, and in particular is contained in $\mathcal U_{-1}$. By definition of $\,\mathcal U_{-1}$, all tracks of the itinerary $f^{\circ n_i}(\gamma)$ that appear after $f^{\circ n_i}(\gamma_i)$ are contained in $\mathcal U_{-1}$ too, and in particular $f^{\circ n_i}(\gamma_k) \subset \mathcal U_{-1}$.
\end{proof}
\begin{proof}[Proof (\cref{prop:finite-length})] Any inverse branch of $f^{-1}: \hat{\mathbb C} \setminus \Postcrit \to \hat{\mathbb C} \setminus \Postcrit$ is a contraction in the hyperbolic metric of the domain $\hat{\mathbb C} \setminus \Postcrit$, by the Schwarz lemma. The contraction is stict since it is a composition $ \iota \circ \tilde f^{-1}$ of the contraction $\tilde f^{-1}:\hat{\mathbb C} \setminus \Postcrit \to \hat{\mathbb C} \setminus f^{-1} (\Postcrit)$ and the inclusion $\iota: \hat{\mathbb C} \setminus f^{-1} (\Postcrit)  \to \hat{\mathbb C} \setminus \Postcrit$ which is a strict contraction.
	
Fix a bound $\Vert (f^{-1})' \Vert _{\mathrm{hyp}}< \theta < 1$ on $\mathcal U_{-1}$ that holds for both branches $f^{-1}: \mathcal U_{-1} \to \mathcal U_{\pm i}$. Then 
$$\HypLength(\gamma) \leq \theta \cdot \HypLength(f(\gamma)).$$
By \cref{lemma-enough-visits-of-premain_station} we thus have 
$$\HypLength(\gamma_k) \lesssim \theta^k.$$
Since the hyperbolic metric is locally equivalent to the Euclidean metric, $\HypLength \asymp \Length$ on $\hat {\mathbb C} \setminus \Delta$ where $\Delta$ is a small neighborhood of the point $\tfrac 12 \in \PostCrit$. 
We conclude that $\Length(\gamma_k) \lesssim \theta^k$ for paths $\gamma$ that are disjoint from $\Delta$. In particular, this holds for all paths that start at the pre-main station $s_{1,1}$.

For such paths, we deduce a bound $\Length(\delta_i) \lesssim \theta^i$ from the corresponding bound on $\gamma_i$
by applying Koebe's distortion theorem on a neighborhood of a given itinerary $\gamma$: 
$$\Length(\delta_i) \, \asymp \, \frac{\Length(f^{\circ n}(\delta_i))}{\Length(f^{\circ n}(\gamma_i))} \cdot \Length(\gamma_i)$$ for any $n$. We choose as before the minimal $n$ for which $f^{\circ n}(\gamma_i)$ lies on the real axis, and then both the numerator $\Length(f^{\circ n}(\delta_i))$ and the denominator $\Length(f^{\circ n}(\gamma_i))$ of the first term are bounded from above and below. Hence, $\Length(\delta_i) \lesssim \Length(\gamma_i) \lesssim \theta ^i$.
 
This concludes the proof for itineraries that start at the pre-main station. The result for a general itinerary follows by Koebe's distortion theorem: Given an itinerary $\gamma_s$ whose first station is $s \neq s_{\mathrm{main}}$, let $\mathcal U_s$ be the preimage of $\mathcal U_{-1}$ under $f$ corresponding to $s$. Koebe's distortion on the corresponding iterate $f^{\circ(-m)}: \mathcal U_{-1} \to \mathcal U_{-s}$ shows that $\Length(\gamma_{s,i}) \lesssim \theta ^i$ since this bound holds for $\gamma := f^{\circ m} (\gamma_s)$ by the previous case.
\end{proof}

\begin{comment}
\begin{proof}
The idea is that peripheral tracks never turn twice in the same direction, so cannot approach the cusp $p$ in a parabolic manner. Thus at large scales the length decays by a definite factor, and by the self-similarity of itineraries the same happens in the small-scale.
To make this rigorous, consider first the set $\Gamma_{\downarrow \downarrow \downarrow}$ of reduced tracks that start at the main station $s_{0,0}$ with at least three consecutive express tracks. Take a neighborhood $N$ of the union of all those tracks on which $f$ is univalent. % so in particular does not contain the critical point $0$
For every station $s$, the corresponding family $\Gamma^s_{\downarrow \downarrow \downarrow}$ of tracks starting at $s$ is the preimage of $\Gamma_{\downarrow \downarrow \downarrow}$  under some branch of an iterate of $f^{-1}$ which maps the main station $s_{0,0}$ to $s$.
Koebe's distortion theorem on this branch gives that
\begin{equation}
\diam(\shadow(s)) \asymp \dist (s,\mathcal J) \asymp \diam (N_s) \asymp \min_{\gamma \in \Gamma^s_{\downarrow \downarrow \downarrow}} \Length(\gamma)
\end{equation}
where $N_s$ is the preimage of $N$. Thus the length of $\eta_j$ decays at the same rate as $\dist(s,\mathcal J)$. %We claim that this rate is exponential.

We used the assumption that the itinerary starts with at least three express tracks in order to have the neighborhood $N$. For itineraries that do not start with three consecutive express tracks, we consider separately the family $\Gamma_{\downarrow \downarrow \leftarrow}$ of those itineraries which start with two express tracks followed by a left itinerary, and the similarly-defined $\Gamma_{\downarrow \downarrow \rightarrow}$. For each of these families the previous argument holds since there is such a neighborhood $N$ of univalence.
\end{proof}
The shadows were not strictly necessary for the preceding proof, but they draw an analogy with similar estimates for the Farey tesselation of the disk.
\end{comment}

We proceed to show quasiconvexity of the itineraries, as sketched above. The purpose of the following definition is to organize points according to their distance from the main cusp in an $f$-invariant way. We decompose the points of $\mathcal J$ according to the first \emph{departure}\,: the first time that the central itinerary made a turn.
\input{"marker path 3.tex"}

\begin{definition}
Let $n \in \mathbb N$. We define the $n$-th \emph{departure set} $I_{n, \mathbb D} \subset \partial \mathbb D ^*$ to be the set of points $\zeta \in \partial \mathbb D^*$ whose central itinerary $\eta_{\zeta}$ starts with $n$ express tracks, followed by a peripheral track.
See \cref{fig:departure-decomposition}.
\end{definition}

This decomposition is invariant under $f_0$ in the sense that $f_0(I_{n+1, \mathbb D})=I_{n, \mathbb D}$, because of the invariance of $\eta _\zeta$.
Thus by applying the Böttcher map $\psi$ we obtain a corresponding departure decomposition $I_n = \psi(I_{n, \mathbb D})$ of $\mathcal J$ that is invariant under $f$.

%Fix a small ball $B$ around the main cusp $1/2$ and trace its preimages until we encounter the points $z_1,z_2$.

\newpage

\subsection{Quasiconvexity: two special cases}
We now show that the itineraries $\eta_{w_1,w_2}$ are certificates for pairs of points $w_1,w_2$ that satisfy either of two conditions.

\emph{Case 1.} The points $w_1,w_2$ are ``far apart'': $|w_1-w_2| \geq \epsilon$ for some fixed $\epsilon >0$.
In this case the length of the itinerary $\eta_{w_1,w_2}$ is bounded above by compactness, so we have a uniform bound on the quasiconvexity constant.

\emph{Case 2.} The points $w_1,w_2$ are in ``well-separated cusp'':
\begin{align} \label{parabolic separation}
	w_1 \in I_n, \quad w_2 \in I_m, \quad m-n \geq d,
\end{align}
where $d$ is a large enough integer, to be chosen later. 

Condition \eqref{parabolic separation} gives some control from below on $|w_1-w_2|$. We now bound the length of the itinerary $\psi(\eta)=\psi(\eta_{w_1,w_2})$ from above.
We represent $\psi(\eta)$ as a concatenation of three paths: the radial segment $\gamma_{m,n}=\psi([s_{m,0},s_{n,0}])$ and the two other components, $\gamma _m$ and $\gamma_n$. See \cref{fig:Three-parts-of-eta}.
Thus we have
\begin{equation}
	\Length(\psi(\eta))=\Length(\gamma_m)+\Length(\gamma_{m,n})+\Length(\gamma_n).
\end{equation}

\input{"departure-decomposition.tex"}

To estimate the length of each summand, we introduce the notation 
\begin{equation}
\ell_k = \Length(\psi([s_{k,0},s_{k+1,0}])) = \psi(s_{k,0})-\psi(s_{k+1,0}).
\end{equation}

\begin{lemma} \label{lem-ell_n}
	The lengths ${\ell_k}$ satisfy:
	
	\emph{(i)} 
	\begin{equation}
	\frac {\ell_k}{\dist(p,I_k)} \to 0,
	\end{equation}
	and 
	
	\emph{(ii)} 
	\begin{equation}
		\frac{\ell_k}{\ell_{k+1}} \to 1.
	\end{equation}
	In particular, for any constant $C \geq 0$ there is a sufficiently large integer $d$ such that
	\begin{align*}
		\Length(\gamma_{m,n}) = \ell_{m}+\ldots+\ell_{n} \\ \leq C (\ell_m+\ell_n)
	\end{align*}
	whenever $|m-n| \geq d$.
\end{lemma}
\begin{proof} \leavevmode
	
	\emph{(i)}
	
	\emph{(ii)} This follows from Koebe's distortion theorem
	
	The map $f$ is affinely conjugate to the map $g: z\mapsto z^2+z$, and the fixed point $\frac 12$ of $f$ corresponds to the fixed point $0$ of $g$. A point near $0$ get repelled from $0$ at rate $\asymp 1/n$, so the distance between consecutive images in the trajectory is $\asymp  1/{k^2}$. Hence $\ell_k \asymp 1/{k^2}$.
\end{proof}

%\begin{definition}
%The \emph{inner diameter} of a connected set $\Omega \subset \mathbb C$ is its diameter relative to the inner metric,
%\begin{equation}
%	\operatorname{InnerDiameter}(\Omega)=\sup _{p,q\in \Omega} \inf_\gamma \Length(\gamma)
%\end{equation}
%where the infimum is over all paths $\gamma$ in $\Omega$ that connect the point $p$ to $q$.
%\end{definition}
%
%We have the following:
%\begin{lemma}
%	\begin{equation*}
%		\operatorname{InnerDiameter}(\psi(I_n)) \lesssim \ell_n
%	\end{equation*}
%	where the hidden constant is independent of $n$.
%\end{lemma}

The condition $|m-n| \geq d$ prevents the line segment $\gamma _{m,n}$ from being small in comparison to $\gamma_m$ and $\gamma_n$:
\begin{proposition} \leavevmode
	\emph{(i)} We have \begin{equation}
		\Length(\gamma_m) \lesssim \Length(\gamma_{m,n}).
	\end{equation}
	\emph{(ii)} For $d \gg 1$, 
	\begin{equation}
		\Length(\gamma_{m,n}) \lesssim |z_1-z_2|.
	\end{equation}
\end{proposition}

\begin{proof}
(i) 	Koebe's distortion theorem gives that 
\begin{equation}
		\Length(\gamma_m) \lesssim \ell_m
	\end{equation}
	by the invariance of both sides under $f^{-1}$.
	%In terms of $\ell _k$, we can write

(ii) By the triangle inequality and part (i),
\begin{align} \label{eq:triang}
	\begin{split}
	\Length(\gamma_{m,n}) \leq |w_1-w_2|+\Length(\gamma_m)+\Length(\gamma_n) %\nonumber
	\\ \leq |w_1-w_2|+C(\ell_m + \ell_n),
	\end{split}
\end{align}
for a constant $C \geq 0$ independent of $d$. Applying \cref{lem-ell_n} on this constant $C$ we get that for $d \gg 1$, 
\begin{align*}
\Length(\gamma_{m,n}) \lesssim \Length(\gamma_{m,n}) - C(\ell_m + \ell _n)
\end{align*}
which together with \eqref{eq:triang} concludes the proof.
\end{proof}

\begin{corollary}
	Let $d \gg 1$ be a positive integer. There exists a constant $A>0$ such that for every pair of points $w_1,w_2$ that satisfy \cref{parabolic separation}, the itinerary $\eta_{w_1,w_2}$ is $A$-quasiconvex.
\end{corollary}

\subsection{Quasiconvexity: general case}

\begin{theorem} \label{quasiconvex-cauliflower} %\leavevmode
	\begin{enumerate}[label=\normalfont(\roman*)]
	The domain $\mathrm{Exterior}(\mathcal{J})$ is quasiconvex with the itineraries $\eta_{z_1,z_2}$ as certificates.
	\end{enumerate}
\end{theorem}

\begin{proof}[Proof. (Parabolic Conformal Elevator on $\mathcal J$)] \label{parabolic-elevator}
Let $d\gg 1$ be a sufficiently large integer and let $z_1,z_2 \in \mathcal J$ be a pair of points. Repeatedly apply $f$ on $z_1,z_2$ until one of the following two stopping conditions happens:
\begin{enumerate}[label=\normalfont(\roman*)]
		\item $\left|f^{\circ N}(z_1)-f^{\circ N}(z_2)\right|>\epsilon$, where $\epsilon=\epsilon(d)>0$ is a constant to be chosen later

	or
	\item $f^{\circ N }(z_1) \in I_n$ and $f^{\circ N }(z_2) \in I_m$ with $|m-n| \geq d$;
\end{enumerate}

Denote $w_i=f^{\circ N}(z_i)$. We already proved that the itinerary $\eta_{w_1,w_2}$ satisfies
\begin{equation*}
	\Length(\eta_{w_1,w_2}) \leq A |w_1-w_2|
\end{equation*}
for some $A>0$. We now deduce that the original points $z_1,z_2$ enjoy a similar estimate,
\begin{equation*}
	\Length(\eta_{z_1,z_2}) \leq C |z_1-z_2|,
\end{equation*}
where $C$ depends only on $A$. This follows from Koebe's distortion theorem applied $N$ times on the univalent function $f^{-1}$, but some care is due in constructing the inverse branches in a neighborhood of the itineraries.
	
We fix a small ball $B=B(p,\eta)$ around the main cusp $p=\tfrac 12$ of radius $\eta = \eta (\epsilon, d)$.
We choose $\eta$ small enough so that any two points $w_1,w_2$ with $|w_1-w_2| \geq \epsilon$ and $w_1 \in B(p, \eta)$ must satisfy criterion \emph{(ii)} with strict inequality.
 The preimages of $B$ are topological balls centered at cusps $q$ of $\mathcal J$, and we index them by $B_{\zeta}$ for $\zeta=\phi^{-1}(q) \in \partial \mathbb D$ where $\phi$ is the Riemann map.

%Each ball $B_{\zeta}$ has a \emph{level}\,, which is the first iterate $k$ for which $f^{\circ k}(B_{\zeta}) = B$. 
Repeatedly applying $f^{-1}$ to the pair $(w_1,w_2)$ until we reach $(z_1,z_2)$, we get two sequences of balls
$B=B^1_1, B^1_2, \dots$ and $B=B^2_1,B^2_2,\dots$. We now consider each case separately.

\emph{Case (i).} Suppose stopping criterion (i) occured, namely $|w_1-w_2| > \epsilon (d)$. 
\begin{comment}
Namely, the points $w_1,w_2$ are a positive distance apart and away from the cusp. Hence also the itinerary $\eta_{w_1,w_2}$ is away from the cusp. 
\end{comment}
We choose a neighborhood $N_1$ of $w_1,w_2$ such that its preimage $f^{-1}(N_1)$ under the branch sending $p=\tfrac 12 $ to $-\tfrac 12$ is disjoint from the post-critical set $\Postcrit$. Next we choose a neighborhood $N_2 \subset N_1$ such that the annulus $N_1 \setminus N_2$ has a definite modulus, and finally we choose a neighborhood $N_3 \subset N_2$ such that for every $w_1,w_2 \in N_3$ we have $\eta_{w_1,w_2} \subset N_2$. 
%Choosing a branch of $f^{\circ -N}$ in $N_3$ that sends $\eta_{w_1,w_2}$ to $\eta_{z_1,z_2}$, 
By compactness of $\mathcal J \setminus B$, these neighborhoods can be chosen from a finite collection, hence the quasiconvexity constant is uniform in the pair $()z_1, z_2)$.


\emph{Case (ii).} Suppose that stopping criterion (ii) occured. We can assume that $w_1,w_2$ are both inside the ball $B$, since $m-n>d$. We claim that the two sequences of balls $(B^1_k), (B^2_k)$ coincide. Indeed, suppose otherwise and let $j$ be the first index for which $B^1_j \neq B^2_j$. The two points $f^{-j}(z_1), f^{-j}(z_2)$ belong to two distinct preimages of the same ball $B^1_{j-1}= B^2_{j-1}$, hence they are a positive distance apart and stopping criterion \emph{i} applies. This is a contradiction, hence the two sequences coincide. 

There is thus a branch of $f^{\circ -(N-1)}$ sending $B_{-1}$ to the preimage topological ball containing $(z_1,z_2)$ and sending $(w_1,w_2)$ to $(z_1,z_2)$. The necessary bound on $\eta_{z_1,z_2}$ follows from applying Koebe's distortion theorem on the composition $f^{-N}$ of this branch with the branch of $f^{-1}$ sending $p$ to $-p$. Explicitly, Koebe's distortion theorem gives 
\begin{equation}
	\frac{\Length(\eta_{z_1,z_2})}{|z_1-z_2|} \asymp \frac{\Length(\eta_{w_1,w_2})}{|w_1-w_2|}.
\end{equation}

%We again consider the branch of $f^{}-1}$
%and we obtain quasiconvexity of $\eta_{z_1,z_2}$ as in case (\emph{i}).


%We define the level of each of $z_1, z_2$ to be the level of the corresponding ball containing them.
%By the stopping criterion, 
%, which implies that
%	\begin{equation*}
%		|w_1-w_2| \lesssim |z_1-z_2| \cdot \|(f^{\circ N})'|_{\mathcal J}\|_{\infty}
%	\end{equation*}
%	and that
%	\begin{align*}
%		\Length (\eta_{w_1,w_2}) = \Length(f^{\circ N}(\eta_{z_1,z_2})) \\
%		\asymp \Length(\eta_{z_1,z_2}) \cdot |(f^{\circ N})'|.
%	\end{align*}
%	The conclusion follows. From now suppose that we are in case (ii).
%By the choice of $w_1,w_2$, for every $k=0,\dots,N$ there is an annulus that separates $f^{\circ k}(z_1),f^{\circ k}(z_2)$ from the post-critical set of $f$, with modulus bounded from below uniformly in $N$.
%By Koebe's distortion we have 
%\begin{align*}
%|f^{\circ k}(z_1)-f^{\circ k}(z_2)|\geq c |z_1-z_2|
%\end{align*}
%for some constant $c>1$.

%Indeed, in this case there is an annulus of definite modulus that has both $z_1,z_2$ in its bounded complementary component and has all points of $\CriticalOrbit(f)$ in the unbounded complementary component.
%
%Instead of keeping track of the distance to $\mathrm{CriticalOrbit}(f)$, it is enough to consider the distance to the fixed point $1/2$, which is its only accumulation point on $\mathcal J$.
%
%Thus by repeatedly applying $f$ we either manage to separate $z_1,z_2$ a definite distance apart, in which case the argument for the hyperbolic case works, or we manage to make
%\begin{align*}
%\big|f^{\circ m} (z_1)-\tfrac 12\big| \leq c |f^{\circ m} (z_1)-f^{\circ m} (z_2)|
%\end{align*}
%for a constant $c$.
\end{proof}


%This concludes the proof.

\begin{comment}
\subsubsection*{Diameter Comparisons}

Denote by $\diam\left(A\right)$ the Euclidean diameter of a set $A$,
and by $i.\diam\left(A\right)$ the inner diameter on the escaping
set, i.e.\ the diameter induced by the Riemann mapping from the escaping
set to the disk.

\textbf{Claim. }$\diam T_{p,n}\asymp i.\diam T_{p,n}$.

\textbf{Claim. }$\diam U_{p,n}\asymp i.\diam U_{p,n}$.

\textbf{Observation. }Let $R>0$ be the distance from the post-critical
set to the Julia set. Consider the ball $B=B(z,\frac{R}{2})$.

Since $B$ is disjoint from the post-critical set, none of its preimages
$f^{-n}\left(B\right)$ contains the critical point $0$.

Thus $f$ restricted on each such preimage is univalent and 2-to-1.

By Koebe's distortion Theorem, applied to iterates $f^{\circ n}$,
the diameter of the preimages of $B$ changes by at most a multiplicative
constant.

This principle shows that all points $z$ on the Julia set $J(p_{c})$
are accessible, since after repeated applications of $f^{n}(B)$ we
get a topological ball of definite size which we may use to construct
an explicit path exiting $K$ in a rectifiable way.

\section{Accessibility in the Parabolic Case}

The basic strategy will be to investigate the geometry near the main
cusp $p=\frac{1}{4}$, since all other cusps are preimages of $p$
under $f$,

whence the principle of the conformal elevator reduces the general
case to the case of a point sufficiently close to the main cusp $p$.

To implement this strategy, we take some arbitrary other point on
the Julia set, connect it by a curve $\gamma$ to $a$, then consider
all preimages of $\gamma$.

In general, consecutive images near a parabolic point have consecutive
distances comparable to $\frac{1}{n^{2}}$ , where $n$ is the index
of the preimage.
\end{comment}

\bibliographystyle{acm}
\bibliography{quasiconvex_cauliflower}
\end{document}
