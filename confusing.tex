Things that I don't understand:

1. Why is the domain $\mathcal K_{m,n+1}$ $\eta$-relatively separated from $\mathcal P$ for $\eta$ \textbf{independent} of $m,n$ when $m,n>C_1$?

...

As stated, the region  $\mathcal K_{m,n+1}$ is not $\eta$ relatively separated, with $\eta$ independent of $m, n$. 

It is true under the assumption $|m-n| < d$.

For $m, n$ small, the region $K_{m,n+1}$ is contained in the exterior of the Julia set and its closure does not contain $p$.

...

2. Why $$\operatorname{Len}(\gamma_m) \leq C \ell_m?$$
To use Koebe we need an inverse branch of $f$ defined on both the segment $\ell_m$ and the track $\gamma_m$.

----------

Unimportant things that I don't understand:

0. The case $m$ small, $n$ large is not truly covered, since the estimate $\ell_m \asymp \frac 1{m^2}$ is only useful for large $m$. 
Maybe it can be taken into account by adding a condition on how small $\epsilon$ should be, so that the conditions $|m-n| \geq d$ and $|w_1-w_2| \leq \epsilon$ will force $m$ to be large enough in comparison to how close $\ell _m$ gets to $\frac 1{m^2}$.

1. Why is $\operatorname{Length}(\delta_1)$ bounded from above? This is needed to claim that $\operatorname{Length}(\eta_z)$ is \textbf{uniformly} bounded.
