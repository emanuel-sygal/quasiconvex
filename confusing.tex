Things that I don't understand:

1. Why is the domain $\mathcal K_{m,n+1}$ $\eta$-relatively separated from $\mathcal P$ for $\eta$ \textbf{independent} of $m,n$ when $m,n>C_1$?

2. Why $$\operatorname{Len}(\gamma_m) \leq C \ell_m?$$
To use Koebe we need an inverse branch of $f$ defined on both the segment $\ell_m$ and the track $\gamma_m$.

3. Why do we must land in one of the three special cases?
i.e. Why Corolllary 2.10 holds?




----------

Unimportant things that I don't understand:

1. Why is $\operatorname{Length}(\delta_1)$ bounded from above? This is needed to claim that $\operatorname{Length}(\eta_z)$ is \textbf{uniformly} bounded.

2. Where does the proof of Case 3 really use its assumption?

3. The case $m$ small, $n$ large is not truly covered, since the estimate $\ell_m \asymp \frac 1{m^2}$ is only useful for large $m$. 
Maybe it can be taken into account by adding a condition on how small $\epsilon$ should be, so that the conditions $|m-n| \geq d$ and $|w_1-w_2| \leq \epsilon$ will force $m$ to be large enough in comparison to how close $\ell _m$ gets to $\frac 1{m^2}$.



% Order of choosing the constants:

% 1. Lemma 6.4: There exists $n_0$ after which $$\ell_n \approx \frac 1{n^2}.$$

% 2. Proof of 6.7: There exists $C$ such that $\gamma _m \leq C \ell _m$ for all $m$.
