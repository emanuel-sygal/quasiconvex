\RequirePackage[l2tabu, orthodox]{nag}
\documentclass[12pt]{article}
\usepackage{geometry}                % See geometry.pdf to learn the layout options. There are lots.
\geometry{letterpaper}
\usepackage[autostyle=false, style=english]{csquotes}
%\MakeOuterQuote{"}
\setlength{\marginparwidth}{2cm}

% Prevents line breaks at math inline
\relpenalty=9999
\binoppenalty=9999

\usepackage{graphicx,color,mathtools}
\usepackage{amssymb,amsmath,amsthm,mathrsfs}
%\usepackage[all,cmtip]{xy}
\usepackage{comment}
%\usepackage{todonotes}
\usepackage{enumitem}   
\usepackage{tikz} 
\usepackage{hyperref}
\usepackage[capitalize,nameinlink,noabbrev]{cleveref}

\usepackage[pdftex,bookmarks,pdfnewwindow,plainpages=false,unicode,pdfencoding=auto]{}


\DeclareGraphicsRule{.tif}{png}{.png}{`convert #1 `dirname #1`/`basename #1 .tif`.png}
\linespread{1.2}

\numberwithin{equation}{section}

\newtheorem{theorem}{Theorem}[section]
\newtheorem{conjecture}[theorem]{Conjecture}
\newtheorem{lemma}[theorem]{Lemma}
\newtheorem{claim}[theorem]{Claim}
\newtheorem{proposition}[theorem]{Proposition}
\newtheorem{corollary}[theorem]{Corollary}

\theoremstyle{remark}
\newtheorem*{remark}{Remark}

\theoremstyle{definition}
\newtheorem{definition}[theorem]{Definition}
\newtheorem{example}[theorem]{Example}

\DeclareMathOperator{\crit}{crit}

\DeclareMathOperator{\Hdim}{H.dim }
\DeclareMathOperator{\supp}{supp }
\DeclareMathOperator{\diam}{diam}
\DeclareMathOperator{\shadow}{Shadow}
\DeclareMathOperator{\idiam}{i.diam}
\DeclareMathOperator{\aut}{Aut}
\DeclareMathOperator{\hol}{Hol}
\DeclareMathOperator{\hyp}{hyp}
\DeclareMathOperator{\dist}{dist}
\DeclareMathOperator{\area}{Area}
\DeclareMathOperator{\loc}{loc}
\DeclareMathOperator{\Mod}{Mod}
\DeclareMathOperator{\Arg}{Arg}
\DeclareMathOperator{\Length}{Length}
\DeclareMathOperator{\HypLength}{HypLength}
\DeclareMathOperator{\CriticalOrbit}{CriticalOrbit}
\DeclareMathOperator{\Postcrit}{\mathcal P}
\DeclareMathOperator{\PostCrit}{\mathcal P}
\DeclareMathOperator{\Closure}{Closure}
\DeclareMathOperator{\RadialSuccessor}{RadialSuccessor}


\global\long\def\R{\mathbb{R}}%
\global\long\def\C{\mathbb{C}}%
\global\long\def\N{\mathbb{N}}%
\global\long\def\Z{\mathbb{Z}}%
\global\long\def\D{\mathbb{D}}%
\global\long\def\H{\mathbb{H}}%
\global\long\def\do#1#2{\frac{\partial#1}{\partial#2}}%
\global\long\def\nor#1{\left|#1\right|}%
\global\long\def\o#1{\overline{#1}}%
\global\long\def\norm#1{\left\Vert #1\right\Vert }%
\global\long\def\\sphere{\hat{\mathbb{C}}}%
\global\long\def\ep{\varepsilon}%
\global\long\def\mp#1{\left\langle #1\right\rangle }%
\global\long\def\lap{\triangle}%
\global\long\def\dz{\mathrm{\mathrm{dz}}}%
\global\long\def\dt{\mathrm{\mathrm{dt}}}%
\global\long\def\dtheta{\mathrm{\mathrm{d\theta}}}%
\global\long\def\dx{\mathrm{\mathrm{dx}}}%
\global\long\def\dw{\mathrm{\mathrm{dw}}}%
\global\long\def\T{\mathrm{\mathbb{T}}}%
