% \usepackage{pgfplots}
% \usetikzlibrary{calc}
% \usepgfplotslibrary{colormaps}


\RequirePackage[l2tabu, orthodox]{nag}
\documentclass[12pt]{article}
\usepackage{geometry}                % See geometry.pdf to learn the layout options. There are lots.
\geometry{letterpaper}
\usepackage[autostyle=false, style=english]{csquotes}
\usepackage{mathtools}
%\MakeOuterQuote{"}
\setlength{\marginparwidth}{2cm}

\usepackage{nomencl}
\makenomenclature
% Run the following command in terminal to update:
% makeindex "main.nlo" -s nomencl.ist -o "main.nls"
\nomenclature{$f, f_c$}{The map $z \mapsto z^2+c.$}
\nomenclature{$\mathcal J$}{The Julia set of $f$.}
\nomenclature{$K$}{The filled Julia set of $f$.}
\nomenclature{$\gamma_{z_1, z_2}$}{The track connecting $z_1$ and $z_2$. It can be either angular (\enquote{peripheral}) or radial (\enquote{express}).}
\nomenclature{$\eta_{z_1, z_2}$}{The itinerary connecting two points. When $z_1$ and $z_2$ are stations, this is the same as $\gamma_{z_1,z_2}$.}
\nomenclature{$A_\infty(f_c)$}{The exterior of the Julia set of $f_c$. The complement of $K_c$.}
\nomenclature{$\mathrm{Exterior}(\mathcal{J})$}{ $\;$ An Alternative notation for $A_\infty(f_c)$.}
\nomenclature{$\psi$}{The Böttcher coordinate $\mathbb D ^* \to \Exterior(\mathcal J)$ conjugating $f_0$ and $f$.}
\nomenclature{$s_{n,k}$}{A station in $\mathbb D^*$ or its image under $\psi$.}
\nomenclature{$\Delta (\gamma, \mathcal P)$}{The relative distance to the post-critical set.}
\nomenclature{$I_{n}$}{The $n$-th departure set.}
\nomenclature{$\ell_n$ }{$\Length([s_{n},s_{n+1}]) = s_{n,0}-s_{n+1,0}.$}
\nomenclature{$\alpha_n$}{the union of the two outermost tracks emanating from the station $s_{n,0}$.}


% Prevents line breaks at math inline
\relpenalty=9999
\binoppenalty=9999

\usepackage{graphicx,color,mathtools}
\usepackage{amssymb,amsmath,amsthm,mathrsfs}
%\usepackage[all,cmtip]{xy}
\usepackage{comment}
%\usepackage{todonotes}
\usepackage{enumitem}   
\usepackage{tikz} 
\usepackage{pgfplots}
\pgfplotsset{compat=1.18}
\usepackage{graphicx}
\usepackage{xcolor}

\usepackage{hyperref}
\usepackage[capitalize,nameinlink,noabbrev]{cleveref}

\usepackage[pdftex,bookmarks,pdfnewwindow,plainpages=false,unicode,pdfencoding=auto]{}


\DeclareGraphicsRule{.tif}{png}{.png}{`convert #1 `dirname #1`/`basename #1 .tif`.png}
\linespread{1.2}

\numberwithin{equation}{section}

\newtheorem{theorem}{Theorem}[section]
\newtheorem{conjecture}[theorem]{Conjecture}
\newtheorem{lemma}[theorem]{Lemma}
\newtheorem{claim}[theorem]{Claim}
\newtheorem{proposition}[theorem]{Proposition}
\newtheorem{corollary}[theorem]{Corollary}

\theoremstyle{remark}
\newtheorem*{remark}{Remark}

\theoremstyle{definition}
\newtheorem{definition}[theorem]{Definition}
\newtheorem{example}[theorem]{Example}

\DeclareMathOperator{\crit}{crit}

\DeclareMathOperator{\Hdim}{H.dim }
\DeclareMathOperator{\supp}{supp }
\DeclareMathOperator{\diam}{diam}
\DeclareMathOperator{\BV}{BV}
\DeclareMathOperator{\shadow}{Shadow}
\DeclareMathOperator{\idiam}{i.diam}
\DeclareMathOperator{\aut}{Aut}
\DeclareMathOperator{\hol}{Hol}
\DeclareMathOperator{\hyp}{hyp}
\DeclareMathOperator{\dist}{dist}
\DeclareMathOperator{\area}{Area}
\DeclareMathOperator{\loc}{loc}
\DeclareMathOperator{\Mod}{Mod}
\DeclareMathOperator{\Arg}{Arg}
\DeclareMathOperator{\Length}{Length}
\DeclareMathOperator{\HypLength}{HypLength}
\DeclareMathOperator{\CriticalOrbit}{CriticalOrbit}
\DeclareMathOperator{\Postcrit}{\mathcal P}
\DeclareMathOperator{\PostCrit}{\mathcal P}
\DeclareMathOperator{\Closure}{Closure}
\DeclareMathOperator{\RadialSuccessor}{RadialSuccessor}
\DeclareMathOperator{\Exterior}{Exterior}

\global\long\def\R{\mathbb{R}}%
\global\long\def\C{\mathbb{C}}%
\global\long\def\N{\mathbb{N}}%
\global\long\def\Z{\mathbb{Z}}%
\global\long\def\D{\mathbb{D}}%
\global\long\def\H{\mathbb{H}}%
\global\long\def\do#1#2{\frac{\partial#1}{\partial#2}}%
\global\long\def\nor#1{\left|#1\right|}%
\global\long\def\o#1{\overline{#1}}%
\global\long\def\norm#1{\left\Vert #1\right\Vert }%
\global\long\def\\sphere{\hat{\mathbb{C}}}%
\global\long\def\ep{\varepsilon}%
\global\long\def\mp#1{\left\langle #1\right\rangle }%
\global\long\def\lap{\triangle}%
\global\long\def\dz{\mathrm{\mathrm{dz}}}%
\global\long\def\dt{\mathrm{\mathrm{dt}}}%
\global\long\def\dtheta{\mathrm{\mathrm{d\theta}}}%
\global\long\def\dx{\mathrm{\mathrm{dx}}}%
\global\long\def\dw{\mathrm{\mathrm{dw}}}%
\global\long\def\T{\mathrm{\mathbb{T}}}%
\begin{document}

\section{Introduction}

A domain $\Omega\subseteq\C$ is called \textbf{quasiconvex} if its intrinsic metric is comparable to the ambient Euclidean metric. Explicitly, this means that there exists a constant $A\geq1$ such that every two points $z_{1},z_{2}\in\Omega$ are connected by a rectifiable path $\gamma:\left[0,1\right]\to\Omega$ which satisfies
$$
\Length(\gamma)\leq A\cdot\left|z_{1}-z_{2}\right|.
$$
We call such a path $\gamma$ a \emph{quasiconvexity certificate} for $z_{1}$ and $z_{2}$.

If $\Omega$ is the interior of a Jordan curve, then by \cite[Corollary F]{hakobyan_euclidean_2008},
it is enough to find certificates for points $z_{1},z_{2}$ that lie
on the boundary curve $\partial\Omega$.%

The \emph{cauliflower} is the filled Julia set of the map $f_{1/4}(z) = z^2+\frac 14$.
We show that its complement, $\mathrm{Exterior}(\mathcal{J}(z^{2}+1/4))$,
is quasiconvex. We then adapt our argument to establish that the exterior of the developed deltoid is quasiconvex.

One motivation to study quasiconvexity stems from its connection with the
John property: If $\Omega$ is a quasiconvex Jordan domain, then its complement has a John interior. See \cite[Corollary 3.4]{hakobyan_euclidean_2008} for a proof.
Thus this result is a strengthening of \cite[Theorem 6.1]{carleson_julia_1994}, in which it is shown directly that the cauliflower is a John domain.

This result also has a function-theoretic interpretation: By \cite[Theorem 1.1]{koskela_geometric_2010} and \cite[Theorem 1.4]{strong_bv_extension_2022}, it shows that the cauliflower is a BV-extension domain and in fact a $W^{1,1}$-extension domain.

\subsection{Sketch of the argument}

We show quasiconvexity by an explicit construction of certificates connecting any given pair of points on the Julia set.
We first build the certificates in the exterior unit disk $\mathbb D ^*$, then we transport them to the exterior of the cauliflower by the Böttcher coordinate $\psi$ of $f_{1/4}$.

To retain control of the certificates after applying $\psi$, we build the certificates on $\mathbb D^*$ in a manner invariant under the map $f_0: z\mapsto z^2$. This is done by only traveling along the boundaries of Carleson boxes in $\mathbb D^{*}$. 

The image of a certificate $\eta$ in $\mathbb D^{*}$ under the conjugacy $\psi$ is invariant under $f_{1/4}$. We use this invariance to show that $\psi(\eta)$ is indeed a certificate, by employing a parabolic variant of the principle of the conformal elevator.

To facilitate the reading, we first demonstrate the proof in the hyperbolic case of maps $f_c(z)=z^2+c$ where  $c\in\left(-\frac {3}{4},\frac{1}{4}\right)$. In this case, the usual conformal elevator applies. We subsequently treat the parabolic case of $c=\frac{1}{4}$.

\section{The exterior disk}

We connect boundary points by moving along the boundaries of Carleson boxes which we now define.
\begin{definition}
Let $n\in\N_{0}$ and $k\in\left\{ 0,\ldots,2^{n}-1\right\} $. We call the set
$$
B_{n,k}=\left\{ z:\quad\left|z\right|\in\biggl(2^{1/2^{n+1}}, 2^{1/2^{n}} \biggl],\qquad\mathrm{arg}(z)\in\biggl(\frac{k}{2^{n}} \, 2\pi,\frac{k+1}{2^{n}} \, 2\pi\biggl]\right\} 
$$
a \emph{Carleson box}\,.
Observe that for a fixed $n$, the union $\bigsqcup_{k=0}^{2^{n}-1}B_{k,n}$
is a partition of the annulus 
$$
\left\{ 2^{1/2^{n+1}} <\left|z\right|\leq 2^{1/2^{n}} \right\} 
$$
 into $2^{n}$ equally-spaced sectors.
 
The \emph{Carleson box decomposition} is the partition of $\D^{*}$ into Carleson
boxes:
$$
\D^{*}=\left\{ \zeta:\,\left|\zeta\right|>2\right\} \sqcup\bigsqcup_{n=0}^{\infty}\bigsqcup_{k=0}^{2^{n}-1}B_{n,k}.
$$
The crucial property of this decomposition is its invariance under $f_{0}$,
stemming from the relation
\begin{equation*}
f_{0}\left(B_{n+1,k}\right)=B_{n,k \,(\operatorname{mod} \,2^n)}.
\end{equation*}
\end{definition}

We describe paths that go along boundaries of Carleson boxes using the metaphor of a passenger who travels by train. 
We now define \enquote{stations} and \enquote{tracks}.

\begin{definition}
A \emph{terminal} is a point $\zeta \in \partial \mathbb D^*$ on the unit circle.
The \emph{central station} is the point $s_{0,0}=2$. \emph{Stations
}are the iterated preimages of the central station under the map $f_{0}:\zeta\mapsto \zeta^{2}$.
We index them as 
$$
s_{n,k}=2^{1/2^{n}} \exp\left(\frac{k}{2^{n}}2\pi i\right),\qquad n\in\N_{0},\quad k\in\left\{ 0,\ldots,2^{n}-1\right\},
$$
and refer to $n$ as the \emph{generation} of the station $s_{n,k}$. The $2^{n}$ stations of generation $n$ are equally spaced on the circle $C_{n}=\left\{ \left|\zeta\right|=2^{1/2^{n}}\right\} $. 
\end{definition}

We next lay two types of \enquote{rail tracks} on the boundaries of Carleson boxes, which we use to travel between stations.

\begin{definition}
Let $s=s_{n,k}$ be a station.

1. The \emph{peripheral neighbors} of $s$ are the two stations $s_{n,\left(k\pm1\right) (\mathrm{mod} 2^{n})}$ adjacent to $s_{n,k}$ on $C_{n}$.

2. The \emph{peripheral track }$\gamma_{s,s'}$ from $s$ to a peripheral neighbor $s'$
is the shorter arc of the circle $C_{n}$ connecting $s$ to $s'$.

3. The \emph{radial successor} of $s$ is $\RadialSuccessor(s)=s_{n+1,2k}$, the unique station of generation $n+1$ on the radial segment $[0,s]$.

4. The \emph{express track} $\gamma_{s,s'}$ from $s$ to its radial successor $s'$ is the radial segment $[s,s']$.
\end{definition}

Notice that the tracks preserve the dynamics: applying $f_0$ to a peripheral track between stations $s,s'$ gives a peripheral track between the parents of $s,s'$ in the tree, and likewise for an express track.

When a passenger travels between two stations $s_1$ and $s_2$, they must follow a particular itinerary from $s_1$ to $s_2$.
If $s_1$ is the central station, then this itinerary is determined by the rule that the passenger stays as close as possible to its destination $s_2$ in the peripheral distance. 
This also determines how to travel from the central station to a terminal $\zeta\in \partial \mathbb D^*$, by continuity. See \cref{fig:Carleson1} and the next definition.


\begin{definition}
Let $\zeta=\exp(2\pi i\theta)\in\partial\D$. The \emph{central itinerary} of $\zeta$ is a path $\eta_\zeta = \gamma _{\sigma_0,\sigma_1} + \gamma_{\sigma_1,\sigma_2}+\ldots$ from the central station to $\zeta$, made of tracks between stations $\sigma_0,\sigma_1,\ldots$. It is defined inductively as follows:

Start at the central station $\sigma_0=s_{0,0}$. Suppose that we already chose $\sigma_0,\ldots,\sigma_k$. If there is a peripheral neighbor $\sigma$ of $\sigma_k$ that is closer peripherally to $\zeta$, meaning that
$$
\left|\Arg\left(\zeta\right)-\Arg\left(\sigma\right)\right|
< \left|\Arg\left(\zeta\right)-\Arg\left(\sigma_{k}\right)\right|,
$$
then take $\sigma_{k+1}=\sigma$. Otherwise, take $\sigma_{k+1}=\RadialSuccessor(\sigma)$.
\end{definition}

\input{"marker path.tex"}

We identify $\eta_{\zeta}$ with its sequence of stations $(\sigma_0,\ldots)$. We record two properties of central itineraries:

\begin{itemize}
	\item There are no two consecutive peripheral tracks in $\eta_{\zeta}$ and thus
	\begin{equation}
	\label{generation-lower-bound}
		\operatorname{Generation}(\sigma_k)\geq \frac k2.
	\end{equation}
	
	\item Central itineraries are essentially invariant under $f_{0}$, in the sense that
	\begin{equation*}
		f_{0}(\eta_{\zeta})=\eta{}_{f_{0}(\zeta)}\cup[s_{0,0},f_0(s_{0,0})]
	\end{equation*}
	for every $\zeta\in \partial \mathbb D^*$.
\end{itemize}

\begin{lemma} \label{track_decay}
Given $\zeta\in \partial \mathbb D^*$, decompose the central itinerary $\eta_{\zeta}$ into its constituent tracks, 
$$
\eta_{\zeta}=\gamma _1 + \gamma_2 + \dots \; .
$$ 
	The lengths of $\gamma_k$ decay exponentially:
	$$\Length(\gamma_{k})\lesssim \theta^{k},$$ uniformly in $\zeta$, for some constant $\theta<1$.
In particular, the total length of $\eta_\zeta$ is bounded above by a definite constant independent of $\zeta$.
\end{lemma}
\begin{proof}
	The radial distances have size $2^{1/2^n}-2^{1/2^{n+1}} \asymp 2^{-n}.$
	By \eqref{generation-lower-bound}, the radial tracks in $\eta_\zeta$ satisfy the required bound with $\theta = \sqrt 2$. The length of a peripheral track of generation $n$ is also $\asymp 2^{-n}$.
\end{proof}

\begin{definition}
\label{def-disk-itinerary}
Given two distinct terminals $\zeta_{1},\zeta_{2}\in\partial\D^*$, form the central itineraries $\eta_{\zeta_{1}}=\left(\sigma_{n}^{1}\right)_{n=0}^{\infty}$ and $\eta_{\zeta_{2}}=\left(\sigma_{n}^{2}\right)_{n=0}^{\infty}$ and let  $\sigma=\sigma^1_i=\sigma^2_j$ be the last station that is in both $\eta_{\zeta_{1}}$ and $\eta_{\zeta_{2}}$.  
	 We define the \emph{itinerary} between  $\zeta_{1}$ and $\zeta_{2}$ to be the path 
$$
 \eta_{\zeta_{1},\zeta_{2}}=  \left(\dots,\sigma_{i+2}^{1},\sigma_{i+1}^{1},\sigma,\sigma_{j+1}^{2},\sigma_{j+2}^{2},\dots\right).
$$
	This is a simple bi-infinite path connecting $\zeta_{1}$ and $\zeta_{2}$, see \cref{fig:Carleson2}. Note that itineraries are equivariant under the dynamics: we have  \begin{equation}
		f(\eta_{\zeta_1,\zeta_2})=\eta_{f(\zeta_1),f(\zeta_2)}
	\end{equation} for every pair of terminals $\zeta_1,\zeta_2 \in \partial \mathbb D^*$ with $|\zeta_1-\zeta_2| < \sqrt{2}$.
\end{definition}
\input{"marker path 2.tex"}

\begin{theorem} \label{quasiconvex disk}
The domain $\D^{*}$ is quasiconvex with the itineraries $\eta_{\zeta_1,\zeta_2}$ as certificates.
\end{theorem}

\begin{proof}
We decompose the itinerary into two paths, so that
\begin{equation}
\Length(\eta_{\zeta_{1},\zeta_{2}})=\Length\left(\sigma,\sigma_{j+1}^{2},\sigma_{j+2}^{2},\dots\right)
+\Length\left(\sigma,\sigma_{i+1}^{1},\sigma_{i+2}^{1},\dots\right),
\end{equation}
and bound each summand using \cref{track_decay}. Denoting $\operatorname{Generation}(\sigma)=n$, we obtain
\begin{align*}
\Length(\eta_{\zeta_{1},\zeta_{2}})
&\lesssim 2 \sum_{k=n}^{\infty}\frac{1}{2^{k}} 
\lesssim2^{-n}, \label{eq:1}
\end{align*}
while 
\begin{align*}
\left|\zeta_{1}-\zeta_{2}\right|
&\asymp
\left|\Arg\left(\zeta_{1}\right)-\Arg\left(\zeta_{2}\right)\right|\\
&\geq \frac {2\pi}{2^{n+2}}.
\end{align*}
%Suppose two stations diverge at generation $N$ and never meet after that. Then we can bound the distance between the endpoints from below. This is not quite as trivial as it sounds but it is correct.
\end{proof}

	
\section{Transporting the Rails} \label{rails-section}
Let $c\in\left[-\frac 34,\frac{1}{4}\right]$ and denote by $\psi$ the Böttcher coordinate of $f: z\mapsto z^2+c$ at infinity. 
Namely, $\psi$ is the unique conformal map $\D^{*}\to\mathrm{Exterior}(\mathcal{J})$  which fixes $\infty$ and satisfies the conjugacy relation 
$$
f\circ\psi=\psi\circ f_{0}.
$$
Since the Julia set $\mathcal J$ is a Jordan curve, the map $\psi$ extends to a homeomorphism between the circle $\partial\D$ and $\mathcal{J}$ by Carathéodory's
theorem.

We apply $\psi$ to the rails that we constructed in $\mathbb D^*$ to obtain the corresponding rails in $\mathrm{Exterior}(\mathcal{J})$:
\begin{definition} \leavevmode
\begin{enumerate}
	\item The 	\emph{stations} of $f_c$ are the points  $s_{n,k,c}=\psi(s_{n,k})$.


\item The \emph{$c$-tracks} are the curves of the form $\psi \left(\gamma_{s,s'}\right)$, where $\gamma_{s,s'}$ is a track. They are classified as express or peripheral according to the corresponding classification of $\gamma_{s,s'}$. 
Express tracks lie on the external rays of the filled Julia set $\mathcal K$, while peripheral tracks lie on the equipotentials of $\mathcal K$.

\item Let $z_1,z_2 \in \mathcal J$ and let $\zeta_i=\psi^{-1}(z_i)$ be the corresponding points on $\partial \mathbb D^*$. 
The \emph{$c$-itineraries} are $\eta_{z_1,z_2}=\psi(\eta_{\zeta_1,\zeta_2})$.
\end{enumerate}

We omit $c$ from the notation for ease of reading. It will be clear from the context whether we work in $\mathbb D^*$ or in $\mathrm{Exterior}(\mathcal J)$.

%These itineraries can equivalently be obtained directly as in the case of $\mathbb D^*$, in terms of following the central itineraries from the common ancestor in the tree structure.
%Trivially since $\psi$ is a bijective correspondence between the two decompositions.
\end{definition}

Note that $\psi\left((1,\infty)\right)\subseteq\R$ since $\mathcal{J}$ is symmetric with respect to the real line. In particular $\psi(s_{0,0})\in\R$, i.e.\ the central station lies on the real axis.


\section{Hyperbolic Maps}

A rational map is \emph{hyperbolic} if under iteration, every critical point converges to an attracting cycle.
Hyperbolic maps enjoy the principle of the conformal elevator, which roughly says that any ball centered on the Julia set can be blown up to a definite size. More precisely, we have the following proposition:

\begin{proposition}[The Principle of the Conformal Elevator] \label{elevator}
	Let $f$ be a hyperbolic rational map, $z\in \mathcal J$ be a point on the Julia set of $f$ and $r>0$. There exists some forward iterate $f^{\circ n}$ of $f$ which is injective on the ball $B(z,2r)$ such that
	$\diam {f^{\circ n}(B(z,r))}$ is bounded below uniformly in $z$ and $r$. 
\end{proposition}

\begin{corollary} \label{elevator for points on julia}
	Let $f$ be a hyperbolic rational map. There exists  $\epsilon > 0$ such that every pair of points $z,w\in\mathcal{J}(f)$ has a forward iterate $f^{\circ n}$ for which $\left|f^{\circ n}(z)-f^{\circ n}(w)\right|>\epsilon$.	
\end{corollary}

\begin{definition}
	A point $z \in \mathcal J$ is \emph{rectifiably accessible} from $\mathrm{Exterior}(\mathcal J)$ if there is a rectifiable curve $\gamma: [0,1) \to \mathrm{Exterior}(\mathcal J)$ such that $\gamma (t) \to z$ as $t \to 1$.
\end{definition}


We are now ready to show the analogue of \cref{quasiconvex disk}:
\begin{theorem}  \leavevmode
\begin{enumerate}[label=\normalfont(\roman*)]

\item Given $z\in\mathcal{J}$ decompose its central itinerary into tracks, 
\begin{equation*}
\eta_z = \gamma _1 +\gamma_2 +\dots.
\end{equation*}
We have the estimate
\begin{equation*}
\Length(\gamma_{k})\lesssim\theta^{-k},
\end{equation*}
uniformly in $z$, for some constant $\theta=\theta(c)>1$. In particular, any point on \(\mathcal{J}\) can be reached from $s_{0,0}$ by a curve of bounded length.

\item The domain $\mathrm{Exterior}(\mathcal{J})$ is quasiconvex with the itineraries $\eta_{z_1,z_2}$ as certificates.
	\end{enumerate}
\end{theorem}

\begin{proof} \leavevmode
(i) We write $\Postcrit$ for the post-critical set of $f$.
Since $\# \mathcal P > 3$,  the Schwarz lemma implies that $f: \hat{\mathbb C} \setminus \Postcrit \to\hat{\mathbb C} \setminus \Postcrit$ is expanding in the hyperbolic metric of $\hat{\mathbb C} \setminus \Postcrit$. The expansion is uniform on any compact subset of $\hat{\mathbb C} \setminus \Postcrit$. See \cite[Theorem 3.5]{mcmullen_1994} for details.
%In particular, any inverse branch $f^{-1}: \hat{\mathbb C} \setminus \Postcrit \to\hat{\mathbb C} \setminus \Postcrit$ is a strict contraction.
 
Let $B(0,R) \subset \mathbb{C}$ be a large ball which contains every central itinerary. As  $\mathrm{Exterior}(\mathcal{J}) \cap B(0,R)$ is compactly contained in $\hat{\mathbb C} \setminus \Postcrit$, 
 $\Vert (f^{-1})' \Vert _{\mathrm{hyp}}< \theta < 1$ on $\mathrm{Exterior}(\mathcal{J}) \cap B(0,R)$. Therefore,
\begin{align*}
\HypLength(\gamma_k)  & \leq \theta \cdot \HypLength(f(\gamma_k))\\ & \leq  \dots
	\\ & \leq \theta^k \cdot \HypLength(f^{\circ k}(\gamma_k)),
	\\ & \leq C \theta^k.
\end{align*}
As the hyperbolic metric is equivalent to the Euclidean metric on $\mathrm{Exterior}(\mathcal{J}) \cap B(0,R)$, 
we conclude that $\Length(\gamma_k) \lesssim \theta^k$ as well.

The corresponding bound $\Length(\delta_k) \lesssim \theta^k$ for peripheral tracks follows in an analogous manner. Alternatively, it can be obtained from the bound for $\Length(\gamma_k)$ using Koebe's distortion theorem.

(ii) Since we already know that the lengths of tracks in the itinerary decay exponentially with rate $\theta>1$, the same proof of the case $c=0$ also shows quasiconvexity in this case.
We give a second proof, relying on \cref{elevator}. This proof will better prepare us for the parabolic $c=1/4$ case, where we do not enjoy a uniform expansion of $f$ on the Julia set.

By \cref{elevator for points on julia}, there exists an $\epsilon>0$ such that any two points are $\epsilon$-apart under some iterate $f$. 
Let $z_{1},z_{2}\in\mathcal{J}(f)$. If $\left|z_{1}-z_{2}\right|\geq\epsilon,$ we are done since the length of $\eta_{z_1,z_2}$ is bounded above by part (i).

On the other hand, if $\left|z_{1}-z_{2}\right|<\epsilon$, then
we may use  \cref{elevator for points on julia} to find an iterate $f^{\circ n}$ such that 
\begin{equation}
	|w_1-w_2|:=\left|f^{\circ n}(z_{1})-f^{\circ n}(z_{2})\right|\geq\epsilon.
\end{equation}
Koebe's distortion theorem implies that 
\begin{equation}
	\frac{\Length\left(\eta_{z_{1},z_{2}}\right)}{|z_1-z_2|}\asymp\frac{\Length\left(\eta_{w_1,w_2}\right)}{|w_1-w_2|}.
\end{equation}
Since the itineraries $\eta_{w_1,w_2}$ are certificates, the original itineraries $\eta_{z_1z,_2}$ are also certificates. 
\end{proof}

\section{The Cauliflower}
In this section, $c=\frac 14$ and $f=f_{1/4}: z\mapsto z^2+ \frac 14$.
%The cauliflower has cusps at the landing points of the dyadic external rays, i.e. the points whose central itinerary has only finitely-many peripheral tracks.
Our goal is to prove the quasiconvexity of $\mathrm{Exterior}(\mathcal{J})$, \cref{quasiconvex-cauliflower}. This is more complicated than the hyperbolic case because the postcritical set $\PostCrit$ of $f$ accumulates at the parabolic fixed point $p=\frac 12$. Since one no longer has a uniform bound on the distortion of inverse iterates, we cannot immediately deduce the quasiconvexity of the itinerary $\eta_{z_1,z_2}$ from the quasiconvexity of $\eta_{w_1,w_2}$ using Koebe's distortion theorem. As a substitute, we present an analog of the principle of the conformal elevator in this parabolic setting.

\subsection{Itineraries have finite length}
We first show that each itinerary $\eta_{z_1,z_2}$ has a finite length. We will in fact show an exponential decay of the lengths of the constituent tracks. For this to hold it is necessary to glue together consecutive express tracks: for example, the tracks in the central itinerary $\eta_{\tfrac 12}$ have only a quadratic rate of length decay.

\begin{definition}
	The \emph{reduced decomposition} of an itinerary $\eta$ is the unique decomposition $\eta=\gamma_1 + \delta_1 + \dots$ where each $\gamma_i$ is a concatenation of express tracks and is followed by a single peripheral track $\delta_i$.
\end{definition}

\begin{proposition} \label{prop:finite-length}
	Let $z \in \mathcal J$, and let $\eta_z= \gamma_1 + \delta_1 + \dots  $ be the reduced decomposition of its itinerary. Then $\Length(\gamma_k)\lesssim \theta^k$ and $\Length(\delta_k)\lesssim \theta^k$ for some $\theta < 1$. In particular, $\Length(\eta_z)<\infty$ and all points $z\mathcal \in \mathcal J$ are rectifiably accessible.
\end{proposition}

For the proof, call $s_{-1}:=s_{1,1}$ the \emph{pre-central station} and let $\mathcal U_{-1}$ be the Jordan domain enclosed by the unit circle, the rightmost itinerary starting from the pre-central station and the leftmost one. This domain is constructed so that it contains all itineraries that start at the pre-central station.  

\begin{lemma} \label{lemma-enough-visits-of-premain_station}
	Let $\gamma = \gamma_1 + \delta_1 + \dots $ be the reduced decomposition of an itinerary $\gamma$. Then for every $k > 1$, there exist $k-1$ iterates $n_1 < \dots < n_{k-1}$ such that $f^{\circ {n_i}}(\gamma_k) \subset \mathcal U_{-1}$. % This is actually an off-by-1 lie, since n_1=0 has no predecessor.
\end{lemma}

\begin{proof}
	Every station $s \not \in (0, \infty)$ has a first iterate $f^{\circ n_s}(s)$ lying on the negative real axis $(-\infty, 0)$.
	For any $i \in \{2, \dots, k-1\}$, let $s_i$ be the first station of $\gamma _i$ and take $n_i := n_{s_i}$.
	By the definition of $\,\mathcal U_{-1}$, the itinerary $f^{\circ n_i}(\gamma)$ is contained in  $\,\mathcal U_{-1}$ from the station $f^{\circ n_i}(s_i)$ onwards, and in particular $f^{\circ n_i}(\gamma_k) \subset \mathcal U_{-1}$.
\end{proof}
\begin{proof}[Proof (\cref{prop:finite-length})] By the Schwarz lemma, any inverse branch $f^{-1}: \hat{\mathbb C} \setminus \Postcrit \to \hat{\mathbb C} \setminus \Postcrit$ is a strict contraction in the hyperbolic metric of the domain $\hat{\mathbb C} \setminus \Postcrit$. 
%The contraction is strict as it is a composition of the contraction $\tilde f^{-1}:\hat{\mathbb C} \setminus \Postcrit \to \hat{\mathbb C} \setminus f^{-1} (\Postcrit)$ and the inclusion $\iota: \hat{\mathbb C} \setminus f^{-1} (\Postcrit)  \to \hat{\mathbb C} \setminus \Postcrit$. The inclusion is a strict contraction as $f^{-1}(\PostCrit) \supsetneq \mathcal \PostCrit$.

As $\mathcal U_{-1}$ is compactly contained in $\hat{\mathbb C} \setminus \Postcrit$, there is a bound $\Vert (f^{-1})' \Vert _{\mathrm{hyp}}< \theta < 1$ on $\mathcal U_{-1}$ that holds for both branches $f^{-1}: \mathcal U_{-1} \to \mathcal U_{\pm i}$. Then, in the notation of \cref{lemma-enough-visits-of-premain_station}, we have 
\begin{equation}
	\begin{split}
	\HypLength(\gamma_k)  &  \leq \HypLength(f^{\circ (n_1-1)}(\gamma_k)) \\ & \leq \theta \cdot \HypLength(f^{\circ n_1}(\gamma_k))\\ & \leq  \dots
	\\ & \leq \theta^k \cdot \HypLength(f^{\circ n_k}(\gamma_k))
		\\ & \leq C \theta^k.
	\end{split}
\end{equation}

As the Euclidean metric is bounded above by a constant multiple of the hyperbolic metric on $B(0,R) \setminus \mathcal P$,  
we a fortiori have $\Length(\gamma_k) \leq C \theta^k$, possibly with a larger constant. This proves the bound for express tracks. The analogous bound $\Length(\delta_k)\lesssim \theta^k$ for peripheral tracks is similar.
\end{proof}


\subsection{Some Estimates and Notation}

To estimate the length of express tracks, we introduce the notations $s_n := s_{n, 0}$ and 
\begin{equation}
	\ell_n := \Length([s_{n},s_{n+1}]) = s_{n}-s_{n+1}.
\end{equation}

\begin{lemma} \label{lem-ell_n}
	The lengths ${\ell_n}$ satisfy:
	
	{\em (i)}
	\begin{equation}
	\label{eq:ell_n1}
		\frac {|p-s_n|}{\ell_n} \to \infty,
	\end{equation}
	
	{\em (ii)}
	\begin{equation}
	\label{eq:ell_n2}
		\frac{\ell_n}{\ell_{n+1}} \to 1.
	\end{equation}
	In particular, for any $C > 0$, there is a sufficiently large integer $d$ such that
	\begin{align*}
		\ell_{m}+\ldots+\ell_{n} \geq C (\ell_m+\ell_n)
	\end{align*}
	whenever $|m-n| \geq d$.
\end{lemma}

\begin{proof}
Using the affine conjugacy of the map $f$ to the map $g: z\mapsto z^2+z$, which sends the parabolic fixed point $\frac{1}{2}$ of $f$ to $0$, one can show that
$$
\ell_n \asymp \frac{1}{n^2} \qquad \text{and} \qquad |p - s_n| \asymp \frac{1}{n}.
$$
After a little arithmetic, we get (\ref{eq:ell_n1}) and (\ref{eq:ell_n2}).
\end{proof}

\begin{definition}
The \emph{relative distance} of a curve $\gamma$ to the post-critical set $\PostCrit$ is 
$$
\Delta (\gamma, \mathcal P) = \frac {\dist(\gamma, \mathcal P)}{\min (\diam (\gamma), \diam (\mathcal P))}.
$$

We say that the curve $\gamma$ is \emph{$\eta$-relatively separated} from the post-critical set if $\Delta (\gamma, \mathcal P) \geq \eta$.  
\end{definition}
 
If an itinerary $\gamma$ is relatively separated from $\PostCrit$, then the preimages of $\gamma$ under $f$ have bounded distortion. In particular, if $\gamma$ is a quasiconvexity certificate, then Koebe's distortion theorem implies that $f^{-1}(\gamma)$ is also a certificate with a comparable constant.

\begin{lemma}
There exists a constant $k>0$ such that for any pair of points $z_1, z_2 \in \mathcal J$, we have $|f(z_1)-f(z_2)| \leq k|z_1-z_2|$.
\end{lemma}

\begin{proof}
By Lagrange's theorem, we may take $k=\max_{z \in B} |f'(z)|$, where $B$ is any ball containing $\mathcal J$. 
\end{proof}
 
\subsection{Dynamics near the cusp}

The purpose of the following definition is to organize points on the Julia set $\mathcal J$ according to their distance from the main cusp in an $f$-invariant way. We decompose the points of $\mathcal J$ according to the first \emph{departure}\,: the first time that the central itinerary makes a turn.
\input{"marker path 3.tex"}

\begin{definition}
Let $n \in \mathbb N$. We define the $n$-th \emph{departure set} $I_{n, \mathbb D} \subset \partial \mathbb D ^*$ to be the set of points $\zeta \in \partial \mathbb D^*$ whose central itinerary $\eta_{\zeta}$ starts with $n$ express tracks, followed by a peripheral track.
See \cref{fig:departure-decomposition}.
\end{definition}

This decomposition is invariant under $f_0$ in the sense that $f_0(I_{n+1, \mathbb D})=I_{n, \mathbb D}$, because of the invariance of $\eta _\zeta$.
Applying the Böttcher map $\psi$, we obtain a corresponding departure decomposition $I_n = \psi(I_{n, \mathbb D})$ of $\mathcal J$ that is invariant under $f$.

We now use this decomposition to analyze the case where the points $w_1,w_2$ lie in ``well-separated cusps''. Namely, suppose that
\begin{align} \label{parabolic separation}
	w_1 \in I_n, \quad w_2 \in I_m, \quad m-n > d,
\end{align}
where $d$ is a sufficiently large integer, to be chosen later. 
This gives some control from below on $|w_1-w_2|$. 
We represent the itinerary $\eta=\eta_{w_1,w_2}$ as a concatenation of three paths: the radial segment $\gamma_{m,n}=[s_{m,0},s_{n,0}]$ and the two other components, $\gamma _m$ and $\gamma_n$. See \cref{fig:Three-parts-of-eta} for the picture in the exterior unit disk.
Thus we have
\begin{equation}
	\Length(\eta)=\Length(\gamma_m)+\Length(\gamma_{m,n})+\Length(\gamma_n).
\end{equation}

\input{"departure-decomposition.tex"}

The condition $m-n \geq d$ prevents the line segment $\gamma _{m,n}$ from being small in comparison to $\gamma_m$ and $\gamma_n$:
\begin{proposition}
\label {case-3-proof}
There exists a sufficiently large integer $d$ so that 
	\begin{equation}
	\label{eq:triangle-comparison}
		\Length(\gamma_{m,n}) \asymp |w_1-w_2|,
	\end{equation}
	whenever $m-n \geq d$.
\end{proposition}

We henceforth fix a value of $d$ as in the proposition.
\begin{proof}
We first make two elementary observations. Koebe's distortion theorem applied to the iterates of $f^{-1}$ shows that
\begin{equation} \label{eq:gamma_m_not_large}
		\Length(\gamma_m) \leq C \ell_m,
	\end{equation}
for some constant $C \geq 0$. Notice that (\ref{eq:gamma_m_not_large}) holds for $m=1$ by \cref{prop:finite-length}, which gives a uniform bound on the length of an itinerary.

Meanwhile, by \cref{lem-ell_n}, there exists an integer $d$ such that
\begin{align}
C(\ell_m + \ell _n) \leq \frac{\Length(\gamma_{m,n})}{2}
\end{align}
whenever $m-n \geq d$.

By the triangle inequality, we have
\begin{align*} 
		\bigl | \Length(\gamma_{m,n}) - |w_1-w_2| \bigr | & \le \Length(\gamma_m)+\Length(\gamma_n) \\
		& \le \frac{\Length(\gamma_{m,n})}{2},
\end{align*}
which clearly implies (\ref{eq:triangle-comparison}).
\end{proof}

\subsection{Quasiconvexity: three special cases}

We now show that the itineraries $\eta_{w_1,w_2}$ are certificates in three special cases. To state them, we introduce some notation.

\subsubsection{Notation}
For each $n$, we denote by  $\alpha_n$ the union of the two outermost tracks emanating from the station $s_{n,0}$. 
Notice that the curves $\alpha_n$ are pairwise disjoint since this is true for their pullbacks to the exterior unit disk.

We define the constants $C_1, C_2, \epsilon$ as follows. We first choose $C_1 \ge 2$, then we let $C_2 = C_1 + d+2$ and choose $\epsilon >0$ small enough so that we have 
\begin{equation}
	\dist(\alpha_{C_2}, \alpha_{C_1}) \geq k \epsilon.
\end{equation}

The constant $C_2$ was chosen so that for any pair $(m,n)$ of integers, we have at least one of the following three cases: either $m,n$ are both greater than $C_1$, or both are smaller than $C_2$, or $|m-n| > d$.

\subsubsection{Three Special Cases}
In this section we treat the following special cases:
\begin{enumerate}
	\item $|w_1-w_2| \geq  \epsilon$, $\quad |m-n|<d$, $\quad m,n < C_2$, $\quad m,n \geq 2$; %relatively separated
	\item $|w_1-w_2| \geq \epsilon$, $\quad |m-n|<d$, $\quad  m,n> C_1$; %relatively separated
	\item $|w_1-w_2| \leq k \epsilon$, $\quad |m-n| \geq d$.
\end{enumerate}

Notice that Case 2 overlaps with Case 1.
 We denote the domain enclosed by $\alpha_m, \alpha_n$ and $\mathcal J$ by $\mathcal K_{m,n}$, and denote the domain enclosed by $\mathcal J$ and $\alpha_n$ by $\mathcal K_n$.

\begin{lemma}
Let $w_1 \in I_m$ and $w_2 \in I_n$, for $n \geq m \geq 2$. Then the itinerary $\eta_{w_1,w_2}$ is contained in the domain $\mathcal K_{m,n+1}$.
\end{lemma}

\begin{lemma}
\label{case 1 rel. sep}
Let $w_1,w_2 \in \mathcal J$. In each of the three special cases, the itinerary $\gamma _{w_1,w_2}$ is a quasiconvexity certificate. In Cases 1 and 2, $\gamma _{w_1,w_2}$ is relatively separated.
\end{lemma}

\begin{proof}
\emph{Case 1.} In this case, the itinerary is contained in the domain $\mathcal K_{2, C_2+1}$.
Since $\dist(\mathcal K_{2, C_2+1}, \mathcal P)>0$, $\gamma _{w_1,w_2}$ is $\eta$-relatively separated for some $\eta>0$. 

\emph{Case 2.}
Assuming without loss of generality that $n \geq m$, the itinerary is contained in $\mathcal K_{m,n+1}$. By Koebe's distortion theorem, $\gamma _{w_1,w_2}$ is also relatively separated.

\emph{Case 3} is the content of \cref{case-3-proof}.
\end{proof}

\subsection{Quasiconvexity: general case}
We apply a stopping time argument to promote the quasiconvexity of $\eta_{w_1,w_2}$ to the quasiconvexity of $\eta_{z_1,z_2}$, thereby proving the following theorem:

\begin{theorem} \label{quasiconvex-cauliflower}
	The domain $\mathrm{Exterior}(\mathcal{J})$ is quasiconvex, with the itineraries $\eta_{z_1,z_2}$ as certificates.
\end{theorem}

\begin{proof}[Proof. (Parabolic Conformal Elevator on $\mathcal J$)] \label{parabolic-elevator}
Let  $(z_1,z_2)$ be a pair of points in $\mathcal J$. Repeatedly apply $f$ to $(z_1, z_2)$ until either of the three special cases occurs. Denote by $w_i=f^{\circ N}(z_i)$ the resulting points. We have already proved that the itinerary $\eta_{w_1,w_2}$ satisfies
\begin{equation*}
	\Length(\eta_{w_1,w_2}) \leq A |w_1-w_2|,
\end{equation*}
for some $A>0$. We deduce that the original pair of points $(z_1,z_2)$ enjoys a similar estimate,
\begin{equation*}
	\Length(\eta_{z_1,z_2}) \leq C |z_1-z_2|,
\end{equation*}
where $C$ depends only on $A$.

In Cases 1 and 2, we are done by \cref{case 1 rel. sep}. In Case 3, 
the itinerary $\eta_{w_1,w_2}$ is contained in $\mathcal K_2$. Let $\mathcal K_{-2}$ be the preimage of $\mathcal K_2$ under $f$ that contains the negative preimage $f^{-1}(p)=-\tfrac 12$ of the cusp $p$. 
As the domain $\mathcal K_{-2}$ is relatively separated from $\PostCrit$ and contains the curve 
$f^{\circ (N-1)}(\eta_{z_1,z_2}) = \eta_{f^{-1}(w_1),f^{-1}(w_2)}$,
we may use Koebe's distortion theorem to conclude that
\begin{equation}
	\frac{\Length(\eta_{z_1,z_2})}{|z_1-z_2|} \, \asymp \,
		\frac{\Length(\eta_{f^{-1}(w_1),f^{-1}(w_2)})}{|f^{-1}(w_1)-f^{-1}(w_2)|} \, \asymp \,
		\frac{\Length(\eta_{w_1,w_2})}{|w_1-w_2|}
\end{equation}
as desired.
\end{proof}


\bibliographystyle{acm}
\bibliography{quasiconvex_cauliflower}
\end{document}
