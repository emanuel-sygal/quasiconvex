\RequirePackage[l2tabu, orthodox]{nag}
\documentclass[12pt]{article}
\usepackage{geometry}                % See geometry.pdf to learn the layout options. There are lots.
\geometry{letterpaper}
\usepackage[autostyle=false, style=english]{csquotes}
%\MakeOuterQuote{"}
\setlength{\marginparwidth}{2cm}

% Prevents line breaks at math inline
\relpenalty=9999
\binoppenalty=9999

\usepackage{graphicx,color,mathtools}
\usepackage{amssymb,amsmath,amsthm,mathrsfs}
\usepackage[all,cmtip]{xy}
\usepackage{comment}
\usepackage{todonotes}
\usepackage{enumitem}   
\usepackage{tikz} 
\usepackage{hyperref}
\usepackage[capitalize,nameinlink,noabbrev]{cleveref}

\usepackage[pdftex,bookmarks,pdfnewwindow,plainpages=false,unicode,pdfencoding=auto]{}


\DeclareGraphicsRule{.tif}{png}{.png}{`convert #1 `dirname #1`/`basename #1 .tif`.png}
\linespread{1.2}

\numberwithin{equation}{section}

\newtheorem{theorem}{Theorem}[section]
\newtheorem{conjecture}[theorem]{Conjecture}
\newtheorem{lemma}[theorem]{Lemma}
\newtheorem{claim}[theorem]{Claim}
\newtheorem{proposition}[theorem]{Proposition}
\newtheorem{corollary}[theorem]{Corollary}

\theoremstyle{remark}
\newtheorem*{remark}{Remark}

\theoremstyle{definition}
\newtheorem{definition}[theorem]{Definition}
\newtheorem{example}[theorem]{Example}

\DeclareMathOperator{\crit}{crit}

\DeclareMathOperator{\Hdim}{H.dim }
\DeclareMathOperator{\supp}{supp }
\DeclareMathOperator{\diam}{diam}
\DeclareMathOperator{\idiam}{i.diam}
\DeclareMathOperator{\aut}{Aut}
\DeclareMathOperator{\hol}{Hol}
\DeclareMathOperator{\hyp}{hyp}
\DeclareMathOperator{\dist}{dist}
\DeclareMathOperator{\area}{Area}
\DeclareMathOperator{\loc}{loc}
\DeclareMathOperator{\Mod}{Mod}
\DeclareMathOperator{\Arg}{Arg}
\DeclareMathOperator{\Length}{Length}
\DeclareMathOperator{\CriticalOrbit}{CriticalOrbit}
\DeclareMathOperator{\Postcrit}{Postcrit}
\DeclareMathOperator{\Closure}{Closure}
\DeclareMathOperator{\RadialSuccessor}{RadialSuccessor}


\global\long\def\R{\mathbb{R}}%
\global\long\def\C{\mathbb{C}}%
\global\long\def\N{\mathbb{N}}%
\global\long\def\Z{\mathbb{Z}}%
\global\long\def\D{\mathbb{D}}%
\global\long\def\H{\mathbb{H}}%
\global\long\def\do#1#2{\frac{\partial#1}{\partial#2}}%
\global\long\def\nor#1{\left|#1\right|}%
\global\long\def\o#1{\overline{#1}}%
\global\long\def\norm#1{\left\Vert #1\right\Vert }%
\global\long\def\\sphere{\hat{\mathbb{C}}}%
\global\long\def\ep{\varepsilon}%
\global\long\def\mp#1{\left\langle #1\right\rangle }%
\global\long\def\lap{\triangle}%
\global\long\def\dz{\mathrm{\mathrm{dz}}}%
\global\long\def\dt{\mathrm{\mathrm{dt}}}%
\global\long\def\dtheta{\mathrm{\mathrm{d\theta}}}%
\global\long\def\dx{\mathrm{\mathrm{dx}}}%
\global\long\def\dw{\mathrm{\mathrm{dw}}}%
\global\long\def\T{\mathrm{\mathbb{T}}}%

\begin{document}

\section{Introduction}
A domain $\Omega\subseteq\C$ is called \textbf{quasiconvex }if its
intrinsic metric is comparable to the ambient Euclidean metric. Explicitly,
this means that there exists a constant $A\geq1$ such that every
two points $z_{1},z_{2}\in\Omega$ are connected by a rectifiable path $\gamma:\left[0,1\right]\to\Omega$
which satisfies
\[
\Length(\gamma)\leq A\cdot\left|z_{1}-z_{2}\right|.
\]
We call such a path $\gamma$ a \emph{quasiconvexity certificate for
	$z_{1}$ and $z_{2}$}.
% or "witness"

If $\Omega$ is the interior of a Jordan curve, then by \cite[Corollary F]{hakobyan_euclidean_2008},
it is enough to find certificates for points $z_{1},z_{2}$ that lie
on the boundary curve $\partial\Omega$.%
\begin{comment}
It is also shown in \cite{hakobyan_euclidean_2008} that any quasidisk is quasiconvex.
\end{comment}

The \emph{cauliflower} is the filled Julia set of the map $f_{1/4}(z) = z^2+\frac 14$.
We show that its complement, $\mathrm{Exterior}(\mathcal{J}(z^{2}+1/4))$,
is quasiconvex. We then adapt our argument to establish that the exterior of the developed deltoid is quasiconvex.

One motivation to study quasiconvexity stems from its connection with the
John property: If $\Omega$ is a quasiconvex Jordan domain, then its complement has a John interior. See \cite[Corollary 3.4]{hakobyan_euclidean_2008} for a proof.
Thus this result is a strengthening of \cite[Theorem 6.1]{carleson_julia_1994}, in which it is shown directly that the cauliflower is a John domain.

This result also has a function-theoretic interpretation: By \cite[Theorem 1.1]{koskela_geometric_2010}, it shows that the cauliflower is a BV-extension domain.
% A bounded, simply connected planar domain is a BV-extension domain if and only if its complement is quasiconvex.

\begin{comment}
The Filled Julia set of $z^{2}+1/4$, called the cauliflower, has
an inward-pointing cusp and hence is not quasiconvex.
\end{comment}

\begin{comment}
Thus, for any $c$ in 

For values $c$ in quadratic polynomials $f_{c}(z)=z^{2}+c$
\end{comment}
\begin{comment}
If $f_{c}$ has an attracting fixed point then its Julia set $\mathcal{J}(f_{c})$
is a quasicircle, hence its interior and exterior are both quasiconvex.
This is the case for values of $c$ in the main cardioid of the Mandelbrot
set.
, i.e.\ for 
\[
c\in\left\{ -\frac{\lambda}{2}-\frac{\lambda^{2}}{4}:\,\left|\lambda\right|<1\right\} .
\]
\end{comment}

\subsection{Sketch of the argument}

We show quasiconvexity by an explicit construction of certificates connecting any given pair of points on the Julia set.
We first build the certificates in the exterior unit disk $\mathbb D ^*$, then we transport them to the exterior of the cauliflower by the Böttcher coordinate $\psi$ of $f_{1/4}$.

To retain control of the certificates after applying $\psi$, we build the certificates on $\mathbb D^*$ in a manner invariant under the map $f_0: z\mapsto z^2$. This is done by only traveling along the boundaries of Carleson boxes in $\mathbb D^{*}$. 

The image of a certificate $\eta$ in $\mathbb D^{*}$ under the conjugacy $\psi$ is invariant under $f_{1/4}$. We use this invariance to show that $\psi(\eta)$ is indeed a certificate, by employing a parabolic variant of the principle of the conformal elevator.

To facilitate the reading, we first demonstrate the proof in the hyperbolic case of maps $f_c(z)=z^2+c$ where  $c\in\left(-\frac 34,\frac{1}{4}\right)$. In this case, the usual conformal elevator applies. We subsequently treat the parabolic case of $c=\frac 14$.

% Although $J_{0.1}^{\text{exterior}}$ is known in advance to be quasiconvex, by virtue of being a quasidisk.


\section{The exterior disk}

\begin{comment}
The exterior $\D^{*}=\left\{ \left|z\right|>1\right\} $ of the unit
disk is trivially quasiconvex by connecting points along the perimeter of the circle. However, these paths follow the boundary closely and their length would blow up if we transport them to the exterior of $\mathcal{J}(f_{c})$, $c\neq0$, via the Riemann map. Instead,
\end{comment}

We connect boundary points by moving along the boundaries of Carleson boxes which we now define.
\begin{definition}
Let $n\in\N_{0}$ and $k\in\left\{ 0,\ldots,2^{n}-1\right\} $. We call the set
%\[B_{n,k}=\exp\left(\biggl(2^{-n-1}\log2,\,2^{-n}\log2\biggl]\times\biggl(\frac{k}{2^{n}} \cdot 2\pi,\,\frac{k+1}{2^{n}} \cdot 2\pi\biggl]\right)\]

\[
B_{n,k}=\left\{ z:\quad\left|z\right|\in\biggl(2^{1/2^{n+1}}, 2^{1/2^{n}} \biggl],\qquad\mathrm{arg}(z)\in\biggl(\frac{k}{2^{n}} \, 2\pi,\frac{k+1}{2^{n}} \, 2\pi\biggl]\right\} 
\]
a \emph{Carleson box}\,.
Observe that for a fixed $n$, the union $\bigsqcup_{k=0}^{2^{n}-1}B_{k,n}$
is a partition of the annulus 
\[
\left\{ 2^{1/2^{n+1}} <\left|z\right|\leq 2^{1/2^{n}} \right\} 
\]
 into $2^{n}$ equally-spaced sectors.
 
The \emph{Carleson box decomposition} is the partition of $\D^{*}$ into Carleson
boxes:
\[
\D^{*}=\left\{ \zeta:\,\left|\zeta\right|>2\right\} \sqcup\bigsqcup_{n=0}^{\infty}\bigsqcup_{k=0}^{2^{n}-1}B_{n,k}.
\]
The crucial property of this decomposition is its invariance under $f_{0}$,
stemming from the relation
\begin{equation*}
f_{0}\left(B_{n+1,k}\right)=B_{n,k \,(\operatorname{mod} \,2^n)}.
\end{equation*}
\end{definition}

We describe the motion along Carleson boxes using the metaphor of a passenger who travels by train. 
We now define \enquote{stations} and \enquote{tracks}.

\begin{definition}
A \emph{terminal} is a point $\zeta \in \partial \mathbb D^*$ on the unit circle.
The \emph{central station} is the point $s_{0,0}=2$. \emph{Stations
}are the iterated preimages of the central station under the map $f_{0}:\zeta\mapsto \zeta^{2}$.
We index them as 
\[
s_{n,k}=2^{1/2^{n}} \exp\left(\frac{k}{2^{n}}2\pi i\right),\qquad n\in\N_{0},\quad k\in\left\{ 0,\ldots,2^{n}-1\right\},
\]
and refer to $n$ as the \emph{generation} of the station $s_{n,k}$. The $2^{n}$ stations of generation $n$ are equally spaced on the circle $C_{n}=\left\{ \left|\zeta\right|=2^{1/2^{n}}\right\} $. 
\end{definition}

We next lay two types of \enquote{rail tracks} on the boundaries of Carleson boxes, which we use to travel between stations.

\begin{definition}
Let $s=s_{n,k}$ be a station.

1. The \emph{peripheral neighbors} of $s$ are the two stations $s_{n,\left(k\pm1\right) (\mathrm{mod} 2^{n})}$ adjacent to $s_{n,k}$ on $C_{n}$.

2. The \emph{peripheral track }$\gamma_{s,s'}$ from $s$ to a peripheral neighbor $s'$
is the shorter arc of the circle $C_{n}$ connecting $s$ to $s'$.

3. The \emph{radial successor} of $s$ is $\RadialSuccessor(s)=s_{n+1,2k}$, the unique station of generation $n+1$ on the radial segment $[0,s]$.

4. The \emph{express track} $\gamma_{s,s'}$ from $s$ to its radial successor $s'$ is the radial segment $[s,s']$.

%5. An \emph{itinerary} is a sequence of stations $\left(\sigma_{0},\sigma_{1},\ldots\right)$. We identify an itinerary with the curve obtained by traveling along it.
\end{definition}

Notice that the tracks preserve the dynamics: applying $f_0$ to a peripheral track between stations $s,s'$ gives a peripheral track between the parents of $s,s'$ in the tree, and likewise for an express track.

When a passenger travels between two stations $s_1$ and $s_2$, they must follow a particular itinerary from $s_1$ to $s_2$.
If $s_1$ is the central station, then this itinerary is determined by the rule that the passenger stays as close as possible to its destination $s_2$ in the peripheral distance. 
This also determines how to travel from the central station to a terminal $\zeta\in \partial \mathbb D^*$, by continuity. See \cref{fig:Carleson1} and the next definition.


\begin{definition}
Let $\zeta=\exp(2\pi i\theta)\in\partial\D$. The \emph{central itinerary} of $\zeta$ is a path $\eta_\zeta = \gamma _{\sigma_0,\sigma_1} + \gamma_{\sigma_1,\sigma_2}+\ldots$ from the central station to $\zeta$, made of tracks between stations $\sigma_0,\sigma_1,\ldots$. It is defined inductively as follows:

Start at the central station $\sigma_0=s_{0,0}$. Suppose that we already chose $\sigma_0,\ldots,\sigma_k$. If there is a peripheral neighbor $\sigma$ of $\sigma_k$ that is closer peripherally to $\zeta$, meaning that $$\left|\Arg\left(\zeta\right)-\Arg\left(\sigma\right)\right|
< \left|\Arg\left(\zeta\right)-\Arg\left(\sigma_{k}\right)\right|,$$ then take $\sigma_{k+1}=\sigma$. Otherwise, take 
$\sigma_{k+1}=\RadialSuccessor(\sigma)$.
\end{definition}

\input{"marker path.tex"}

We identify $\eta_{\zeta}$ with its sequence of stations $(\sigma_0,\ldots)$. We record two properties of central itineraries:

\begin{itemize}
	\item There are no two consecutive peripheral tracks in $\eta_{\zeta}$ and thus
	\begin{equation} \label{generation-lower-bound}
		\operatorname{Generation}(\sigma_k)\geq \frac k2.
	\end{equation}
	
	\item Central itineraries are essentially invariant under $f_{0}$, in the sense that
	\begin{equation*}
		f_{0}(\eta_{\zeta})=\eta{}_{f_{0}(\zeta)}\cup[s_{0,0},f_0(s_{0,0})]
	\end{equation*}
	for every $\zeta\in \partial \mathbb D^*$.
\end{itemize}

\begin{lemma} \label{track_decay}
Given $\zeta\in \partial \mathbb D^*$, decompose the central itinerary $\eta_{\zeta}$ into its constituent tracks, 
$$\eta_{\zeta}=\gamma _1 + \gamma_2 + \dots \; .$$ 
	The lengths of $\gamma_k$ decay exponentially:
	$$\Length(\gamma_{k})\lesssim \theta^{k},$$ uniformly in $\zeta$, for some constant $\theta<1$.
In particular, the total length of $\eta_\zeta$ is bounded above by a definite constant independent of $\zeta$.
\end{lemma}
\begin{proof}
	The radial distances have size $2^{1/2^n}-2^{1/2^{n+1}} \asymp 2^{-n}.$
	\begin{comment}
		$$2^{1/2^n}-2^{1/2^{n+1}} = 2^{1/2^{n+1}} \sum_{k=1}^{\infty} \binom{1/2^{n+1}}{k} \asymp 2^{-n}.$$
	\end{comment}
	By \eqref{generation-lower-bound}, the radial tracks in $\eta_\zeta$ satisfy the required bound with $\theta = \sqrt 2$. The length of a peripheral track of generation $n$ is also $\asymp 2^{-n}$.
\end{proof}

%In particular, the length of a suffix $\sigma_{k}+\sigma_{k+1}+\ldots$ decays exponentially in $k$, uniformly in $\zeta$.

%Note that this may result in an "inefficient" itinerary: two stations that are peripheral neighbors might have a rather long itinerary between them.
%By continuity, this defines the itinerary between any two terminals $\zeta_1$ and $\zeta_2$:


\begin{definition} \label{def-disk-itinerary} Given two distinct terminals $\zeta_{1},\zeta_{2}\in\partial\D^*$, form the central itineraries $\eta_{\zeta_{1}}=\left(\sigma_{n}^{1}\right)_{n=0}^{\infty}$ and 
	$\eta_{\zeta_{2}}=\left(\sigma_{n}^{2}\right)_{n=0}^{\infty}$
	 and let  $\sigma=\sigma^1_i=\sigma^2_j$ be the last station that is in both $\eta_{\zeta_{1}}$ and $\eta_{\zeta_{2}}$. %Note that $\sigma$ is well-defined.
	 We define the \emph{itinerary} between  $\zeta_{1}$ and $\zeta_{2}$ to be the path  %$\eta_{\zeta_{i}}^{\text{truncated}}=\left(\sigma_{N},\sigma_{N+1}^{i},\ldots\right)$ be the truncated paths. Then
 \begin{gather*}
 \eta_{\zeta_{1},\zeta_{2}}=  \left(\dots,\sigma_{i+2}^{1},\sigma_{i+1}^{1},\sigma,\sigma_{j+1}^{2},\sigma_{j+2}^{2},\dots\right).
 \end{gather*}
	This is a simple bi-infinite path connecting $\zeta_{1}$ and $\zeta_{2}$, see \cref{fig:Carleson2}. Note that itineraries are equivariant under the dynamics: we have  \begin{equation}
		f(\eta_{\zeta_1,\zeta_2})=\eta_{f(\zeta_1),f(\zeta_2)}
	\end{equation} for every pair of terminals $\zeta_1,\zeta_2 \in \partial \mathbb D^*$.
	
	
\end{definition}
\input{"marker path 2.tex"}

\begin{theorem} \label{quasiconvex disk}
The domain $\D^{*}$ is quasiconvex with the itineraries $\eta_{\zeta_1,\zeta_2}$ as certificates.
\end{theorem}

\begin{proof}
We decompose the itinerary into two paths, so that
\begin{equation}
\Length(\eta_{\zeta_{1},\zeta_{2}})=\Length\left(\sigma,\sigma_{j+1}^{2},\sigma_{j+2}^{2},\dots\right)
+\Length\left(\sigma,\sigma_{i+1}^{1},\sigma_{i+2}^{1},\dots\right),
\end{equation}
and bound each summand using \cref{track_decay}. Denoting $\operatorname{Generation}(\sigma)=n$, we obtain
\begin{align*}
\Length(\eta_{\zeta_{1},\zeta_{2}})
&\lesssim 2 \sum_{k=n}^{\infty}\frac{1}{2^{k}} 
\lesssim2^{-n}, \label{eq:1}
\end{align*}
while 
\begin{align*}
\left|\zeta_{1}-\zeta_{2}\right|
&\asymp
\left|\Arg\left(\zeta_{1}\right)-\Arg\left(\zeta_{2}\right)\right|\\
&\geq \frac {2\pi}{2^{n+2}}.
\end{align*}
%Suppose two stations diverge at generation $N$ and never meet after that. Then we can bound the distance between the endpoints from below. This is not quite as trivial as it sounds but it is correct.
\end{proof}

	
\section{Transporting the Rails} \label{rails-section}
Let $c\in\left[-\frac 34,\frac{1}{4}\right]$ and denote by $\psi$ the Böttcher coordinate of $f: z\mapsto z^2+c$ at infinity. 
Namely, $\psi$ is the unique conformal map $\D^{*}\to\mathrm{Exterior}(\mathcal{J})$  which fixes $\infty$ and satisfies the conjugacy $$f\circ\psi=\psi\circ f_{0}.$$
Since the Julia set $\mathcal J$ is a Jordan curve, the map $\psi$ extends to a homeomorphism between the circle $\partial\D$ and $\mathcal{J}$ by Carathéodory's
theorem.

We apply $\psi$ to the rails that we constructed in $\mathbb D^*$ to obtain the corresponding rails in $\mathrm{Exterior}(\mathcal{J})$:
\begin{definition} \leavevmode
\begin{enumerate}
	\item The 	\emph{stations} of $f_c$ are the points  $s_{n,k,c}=\psi(s_{n,k})$.


\item The \emph{$c$-tracks} are the curves of the form $\psi \left(\gamma_{s,s'}\right)$, where $\gamma_{s,s'}$ is a track. They are classified as express or peripheral according to the corresponding classification of $\gamma_{s,s'}$. 
Express tracks lie on \emph{external} rays of the filled Julia set $\mathcal K$, while peripheral tracks lie on the equipotentials of $\mathcal K$.

\item Let $z_1,z_2 \in \mathcal J$ and let $\zeta_i=\psi^{-1}(z_i)$ be the corresponding points on $\partial \mathbb D^*$. 
The \emph{$c$-itineraries} are $\eta_{z_1,z_2}=\psi(\eta_{\zeta_1,\zeta_2})$.
\end{enumerate}

We omit $c$ from the notation for ease of reading. It will be clear from the context whether we work in $\mathbb D^*$ or in $\mathrm{Exterior}(\mathcal J)$.

%These itineraries can equivalently be obtained directly as in the case of $\mathbb D^*$, in terms of following the central itineraries from the common ancestor in the tree structure.
%Trivially since $\psi$ is a bijective correspondence between the two decompositions.
\end{definition}

Note that $\psi\left((1,\infty)\right)\subseteq\R$ since $\mathcal{J}$ is symmetric with respect to the real line. In particular $\psi(s_{0,0})\in\R$, i.e.\ the central station is real.

\begin{comment}
\begin{proof}
Formally, $\overline{\psi}(\overline{z})$
	is another conformal conjugacy between $f$ and $f_0$ which fixes infinity, so by the uniqueness of the Böttcher coordinate we obtain $\psi(z)=\overline{\psi}(\overline{z})$,
	hence $\psi(z)\in\R$ for $z\in\R$.
\end{proof}
\end{comment}

\section{Hyperbolic Maps}

A rational map is \emph{hyperbolic} if, under iteration, every critical point converges to an attracting cycle.
Hyperbolic maps enjoy the principle of the conformal elevator, which roughly says that any ball centered on the Julia set can be blown up to a definite size. More precisely, we have the following:

\begin{proposition}[The Principle of the Conformal Elevator] \label{elevator}
	Let $f$ be a hyperbolic rational map, $z\in \mathcal J$ be a point on the Julia set of $f$ and $r>0$. There exists some forward iterate $f^{\circ n}$ of $f$ which is injective on the ball $B(z,2r)$ such that
	$\diam {f^{\circ n}(B(z,r))}$ is bounded below uniformly in $z$ and $r$. 
\end{proposition}

%See \cite{bonk_quasisymmetries_2016} for a stronger version, details and a proof.

\begin{corollary} \label{elevator for points on julia}
	Let $f$ be a hyperbolic rational map. There exists  $\epsilon > 0$ such that every pair of points $z,w\in\mathcal{J}(f)$ has a forward iterate $f^{\circ n}$ for which $\left|f^{\circ n}(z)-f^{\circ n}(w)\right|>\epsilon$.	
\end{corollary}

\begin{definition}
	A point $z \in \mathcal J$ is \emph{rectifiably accessible} from $\mathrm{Exterior}(\mathcal J)$ if there is a rectifiable curve $\gamma: [0,1) \to \mathrm{Exterior}(\mathcal J)$ such that $\gamma (t) \to z$ as $t \to 1$.
\end{definition}

\begin{comment}
\begin{proof}
	Apply \cref{elevator} to a ball centered on the Julia set which contains $z,w$ on its boundary at roughly antipodal points. After blowing up we get points $f^{\circ n}(z),f^{\circ n}(w)$ which are a definite distance apart by Koebe's distortion theorem. %\todo{make this correct}
\end{proof}
\end{comment}

We are now ready to show the analog of \cref{quasiconvex disk}:
%\todo{We use more than hyperbolicity, also that the Julia set is a Jordan domain}
\begin{theorem}  \leavevmode
\begin{enumerate}[label=\normalfont(\roman*)]

\item Given $z\in\mathcal{J}$ decompose its central itinerary into tracks, 
\begin{equation*}
\eta_z = \gamma _1 +\gamma_2 +\dots.
\end{equation*}
We have the estimate
\begin{equation*}
\Length(\gamma_{k})\lesssim\theta^{-k},
\end{equation*}
uniformly in $z$, for some constant $\theta=\theta(c)>1$. In particular, any point on \(\mathcal{J}\) can be reached from $s_{0,0}$ by a curve of bounded length.

\item The domain $\mathrm{Exterior}(\mathcal{J})$ is quasiconvex with the itineraries $\eta_{z_1,z_2}$ as certificates.
	\end{enumerate}
\end{theorem}

\begin{proof} \leavevmode
\begin{enumerate}[label=\normalfont(\roman*)]
\item 
By the Schwarz lemma, any inverse branch $f^{-1}: \Exterior(\mathcal J) \to \Exterior(\mathcal J)$ is a strict contraction in the hyperbolic metric of the domain $\Exterior(\mathcal J)$. Hence there is a bound $\Vert (f^{-1})' \Vert _{\mathrm{hyp}}< \theta < 1$, and we have 
\begin{equation}
	\begin{split}
		\HypLength(\gamma_k)  & \leq \theta \cdot \HypLength(f(\gamma_k))\\ & \leq  \dots
		\\ & \leq \theta^k \cdot \HypLength(f^{\circ k}(\gamma_k)),
	\end{split}
\end{equation}

and the length of $f^{\circ k}(\gamma_k)$ is uniformly bounded because it is a track.

%By \cref{lemma-enough-visits-of-premain_station}, we thus have 
%$$\HypLength(\gamma_k) \lesssim \theta^k.$$
The hyperbolic metric is equivalent to the Euclidean metric on  $\Exterior(\mathcal J)$, hence
we conclude that $\Length(\gamma_k) \lesssim \theta^k$ for all tracks $\gamma_k$.

The corresponding bound $\Length(\delta_k) \lesssim \theta^k$ for peripheral tracks follows in an analogous manner. Alternatively, it follows from the corresponding bound on $\gamma_k$ by applying Koebe's distortion theorem on a neighborhood of the given itinerary: 
$$\Length(\delta_k) \, \asymp \, \frac{\Length(f^{\circ n}(\delta_k))}{\Length(f^{\circ n}(\gamma_k))} \cdot \Length(\gamma_k)$$ for any $n$. 



%The map $f$ has some iterate $f^{\circ N}$ such that $|(f^{\circ N})'(z)|>1$ for all $z \in \mathcal{J}$.
%By the compactness of $\mathcal J$, there is a $\theta >1$ such that $\left|(f^{\circ N})'\right (z)|>\theta$ on some neighborhood $\mathcal{U}$ of $\mathcal{J}$.
%Since every itinerary is eventually contained in $\mathcal{U}$, for 
%	This is false! And should be unimportant but I want to spell out the reason

% itineraries $\gamma$ we have 
%\begin{equation*}
%\Length ( f^{\circ N}(\gamma) ) \geq \theta \cdot \Length( \gamma). 
%\end{equation*}

%The peripheral tracks on circles $C_n$ of index $n \equiv k$ modulo $N$ have a total length bounded by a geometric series of rate $\theta$, hence finite. 
%The lengths of the express tracks can be bounded in the same way. % We conclude that the total length of the itinerary is bounded.

\item Since we already know that the lengths of tracks in the itinerary decay exponentially with rate $\theta>1$, the same proof of the case $c=0$ also shows quasiconvexity in this case.
We give a second proof, relying on \cref{elevator for points on julia}. This proof will better prepare us for the parabolic $c=1/4$ case, where we do not enjoy a uniform expansion of $f$ on the Julia set.

By \cref{elevator for points on julia}, there exists an $\epsilon>0$ such that any two points are $\epsilon$-apart under some iterate $f$. 
Let $z_{1},z_{2}\in\mathcal{J}(f)$. If $\left|z_{1}-z_{2}\right|\geq\epsilon,$ we are done since the length of $\eta_{z_1,z_2}$ is bounded above by part (i).
% we may just concatenate $\eta_{z_{1}}$ and $\eta_{z_{2}}$ and absorb this bounded length into the quasiconvexity constant

\begin{comment}
Explicitly, if $\Length\left(\eta_{z}\right)\leq L$
for all $z\in\mathcal{J}$ then we take $A\geq\frac{2L}{\epsilon}$
and then automatically $\Length\left(\eta_{z_{1}}+\eta_{z_{2}}\right)\leq A\left|z_{1}-z_{2}\right|$.
\end{comment}

On the other hand, if $\left|z_{1}-z_{2}\right|<\epsilon$, then
we may use  \cref{elevator for points on julia} to find an iterate $f^{\circ n}$ such that 
\begin{equation}
	|w_1-w_2|:=\left|f^{\circ n}(z_{1})-f^{\circ n}(z_{2})\right|\geq\epsilon.
\end{equation}
Koebe's distortion theorem implies that 
\begin{equation}
	\frac{\Length\left(\eta_{z_{1},z_{2}}\right)}{|z_1-z_2|}\asymp\frac{\Length\left(\eta_{w_1,w_2}\right)}{|w_1-w_2|}.
\end{equation}
Since the itineraries $\eta_{w_1,w_2}$ are certificates, the original itineraries $\eta_{z_1z,_2}$ are also certificates. 
\begin{comment}
By a distortion estimate
\begin{equation*}
\Length\left(\eta_{z_{0},z_{1}}\right)\asymp\frac{\Length\left(\eta_{w_0,w_1}\right)}{\left|\left(f^{\circ n}\right)'\left(\zeta\right)\right|}
\end{equation*}
 for some point $\zeta \in \mathcal{J}$. The denominator grows
with $n$ exponentially at rate $\theta$, while the numerator has
a bound of the form 
\[
\Length\left(\eta_{w_1,w_2}\right)\lesssim\left|w_1-w_2\right|\lesssim\theta^{n}\left|z_{1}-z_{2}\right|.
\]
Altogether 
\[
\Length\left(\eta_{z_{1},z_{2}}\right)\lesssim\frac{\theta^{n}\left|z_{1}-z_{2}\right|}{\theta^{n}}=\left|z_{1}-z_{2}\right|
\]
 so $\eta_{z_{1},z_{2}}$ is a quasiconvexity certificate.
\end{comment}
\end{enumerate}
\end{proof}

\section{The Cauliflower}
In this section, $c=\frac 14$ and $f=f_{1/4}: z\mapsto z^2+ \frac 14$.
%The cauliflower has cusps at the landing points of the dyadic external rays, i.e. the points whose central itinerary has only finitely-many peripheral tracks.
Our goal is to prove the quasiconvexity of $\mathrm{Exterior}(\mathcal{J})$, \cref{quasiconvex-cauliflower}. This is more complicated than the previous hyperbolic case because the postcritical set $\PostCrit$ of $f$ accumulates at the parabolic fixed point $p=\frac 12$. Since one no longer has a uniform bound on the distortion of inverse iterates, we cannot immediately deduce the quasiconvexity of the itinerary $\eta_{z_1,z_2}$ from the quasiconvexity of $\eta_{w_1,w_2}$ using Koebe's distortion theorem. As a substitute, we present an analog of the principle of the conformal elevator in this parabolic setting.

\subsection{Itineraries have finite length}
We first show that each itinerary $\eta_{z_1,z_2}$ has a finite length. We will in fact show an exponential decay of the lengths of the constituent tracks. For this to hold it is necessary to glue together consecutive express tracks: for example, the tracks in the central itinerary $\eta_{\tfrac 12}$ have only a quadratic rate of length decay.

\begin{definition}
	The \emph{reduced decomposition} of an itinerary $\eta$ is the unique decomposition $\eta=\gamma_1 + \delta_1 + \dots$ where each $\gamma_i$ is a concatenation of express tracks and is followed by a single peripheral track $\delta_i$.
\end{definition}

\begin{comment}
The following definition is used in the proof of \cref{prop:finite-length}.
\begin{definition}
	The \emph{shadow} of a station $s$ is the set of all terminals $z\in \mathcal J$ that are the endpoints of itineraries starting at $s$.
\end{definition}

Using this definition, we claim:
\end{comment}


\begin{proposition} \label{prop:finite-length}
	Let $z \in \mathcal J$, and let $\eta_z= \gamma_1 + \delta_1 + \dots  $ be the reduced decomposition of its itinerary. Then $\Length(\gamma_k)\lesssim \theta^k$ and $\Length(\delta_k)\lesssim \theta^k$ for some $\theta < 1$. In particular, $\Length(\eta_z)<\infty$ and all points $z\mathcal \in \mathcal J$ are rectifiably accessible.
\end{proposition}

For the proof, call $s_{-1}:=s_{1,1}$ the \emph{pre-central station} and let $\mathcal U_{-1}$ be the Jordan domain enclosed by the unit circle, the rightmost itinerary starting from the pre-central station and the leftmost one. This domain is constructed so that it contains all itineraries that start at the pre-central station.  
\begin{comment}
We call an itinerary \emph{non-positive} if it is not contained in the positive real axis.
\end{comment}

\begin{lemma} \label{lemma-enough-visits-of-premain_station}
	Let $\gamma = \gamma_1 + \delta_1 + \dots $ be the reduced decomposition of an itinerary $\gamma$. Then for every $k \geq 1$, there exist $k-1$ iterates $n_1 < \dots < n_{k-1}$ such that $f^{\circ {n_i}}(\gamma_k) \subset \mathcal U_{-1}$. % This is actually an off-by-1 lie, since n_1=0 has no predecessor.
\end{lemma}
%More importantly, the lemma assumes that the curve is not contained in its entirety in the positive real axis but it may have a subreduced path IS contained there, and then the proof breaks.  

\begin{proof}
	Every station $s \not \in (0, \infty)$ has a first iterate $f^{\circ n_s}(s)$ lying on the negative real axis $(-\infty, 0)$.
	For any $i \in \{2, \dots, k-1\}$, let $s_i$ be the first station of $\gamma _i$ and take $n_i := n_{s_i}$.
	%be the first iterate such that the first station of the track $f^{\circ n_i}(\gamma_i)$ is in $(-\infty, 0)$.
	%The reduced tracks $\gamma_i$ are disjoint from the positive axis $(0,\infty)$, and thus $\gamma_i$ has a first iterate $n_i$ such that the first station of the track $f^{\circ n_i}(\gamma_i)$ is in $(-\infty, 0)$.
	By the definition of $\,\mathcal U_{-1}$, the itinerary $f^{\circ n_i}(\gamma)$ is contained in  $\,\mathcal U_{-1}$ from the station $f^{\circ n_i}(s_i)$ onwards, and in particular $f^{\circ n_i}(\gamma_k) \subset \mathcal U_{-1}$.
%	 In particular, we have $f^{\circ n_i}(\gamma_i) \subset \mathcal U_{-1}$. 
%	 all reduced tracks in the itinerary $f^{\circ n_i}(\gamma)$ that appear after $f^{\circ n_i}(\gamma_i)$ are contained in $\mathcal U_{-1}$ too, and in particular $f^{\circ n_i}(\gamma_k) \subset \mathcal U_{-1}$.
\end{proof}
\begin{proof}[Proof (\cref{prop:finite-length})] By the Schwarz lemma, any inverse branch $f^{-1}: \hat{\mathbb C} \setminus \Postcrit \to \hat{\mathbb C} \setminus \Postcrit$ is a contraction in the hyperbolic metric of the domain $\hat{\mathbb C} \setminus \Postcrit$. The contraction is strict as it is a composition of the contraction $\tilde f^{-1}:\hat{\mathbb C} \setminus \Postcrit \to \hat{\mathbb C} \setminus f^{-1} (\Postcrit)$ and the inclusion $\iota: \hat{\mathbb C} \setminus f^{-1} (\Postcrit)  \to \hat{\mathbb C} \setminus \Postcrit$. The inclusion is a strict contraction as $f^{-1}(\PostCrit) \supsetneq \mathcal \PostCrit$.

As $\mathcal U_{-1}$ is compactly contained in $\hat{\mathbb C} \setminus \Postcrit$, there is a bound $\Vert (f^{-1})' \Vert _{\mathrm{hyp}}< \theta < 1$ on $\mathcal U_{-1}$ that holds for both branches $f^{-1}: \mathcal U_{-1} \to \mathcal U_{\pm i}$. Then, in the notation of \cref{lemma-enough-visits-of-premain_station}, we have 
\begin{equation}
	\begin{split}
	\HypLength(\gamma_k)  &  \leq \HypLength(f^{\circ (n_1-1)}(\gamma_k)) \\ & \leq \theta \cdot \HypLength(f^{\circ n_1}(\gamma_k))\\ & \leq  \dots
	\\ & \leq \theta^k \cdot \HypLength(f^{\circ n_k}(\gamma_k)),
	\end{split}
\end{equation}
and the length of $f^{\circ n_k}(\gamma_k)$ is uniformly bounded because it is a track.

%By \cref{lemma-enough-visits-of-premain_station}, we thus have 
%$$\HypLength(\gamma_k) \lesssim \theta^k.$$
As the hyperbolic metric is locally equivalent to the Euclidean metric, $\HypLength \asymp \Length$ on $\hat {\mathbb C} \setminus \Delta$ where $\Delta$ is a small neighborhood of the point $\tfrac 12 \in \PostCrit$. 
We conclude that $\Length(\gamma_k) \lesssim \theta^k$ for reduced tracks $\gamma_k$ that are disjoint from $\Delta$. In particular, this holds for all paths that start at the pre-central station $s_{1,1}$.

%Alternatively, this bound follows from the corresponding bound on $\gamma_k$
%by applying Koebe's distortion theorem on a neighborhood of the given itinerary: 
%$$\Length(\delta_k) \, \asymp \, \frac{\Length(f^{\circ n}(\delta_k))}{\Length(f^{\circ n}(\gamma_k))} \cdot \Length(\gamma_k)$$ for any $n$. We choose as before the minimal $n$ for which $f^{\circ n}(\gamma_k)$ lies on the real axis, and then both the numerator $\Length(f^{\circ n}(\delta_k))$ and the denominator $\Length(f^{\circ n}(\gamma_k))$ of the first term are bounded from above and below, hence $\Length(\delta_k) \asymp \Length(\gamma_k)$.

This concludes the proof for itineraries that start at the pre-central station. The result for a general itinerary follows by Koebe's distortion theorem: Given an itinerary $\gamma_s$ whose first station is $s \neq s_{\mathrm{central}}$, let $\mathcal U_s$ be the preimage of $\mathcal U_{-1}$ under $f$ corresponding to $s$, then Koebe's distortion on the corresponding iterate $f^{\circ(-m)}: \mathcal U_{-1} \to \mathcal U_{-s}$ immediately gives that $\Length(\gamma_{s,k}) \lesssim \theta ^k$. This proves the bound on express tracks. The analogous bound $\Length(\delta_k)\lesssim \theta^k$ for peripheral tracks is similar.
%since this bound holds for the reduced tracks in the itinerary $\gamma := f^{\circ m} (\gamma_s)$.
\end{proof}

\begin{comment}
\begin{proof}
The idea is that peripheral tracks never turn twice in the same direction, so cannot approach the cusp $p$ in a parabolic manner. Thus at large scales the length decays by a definite factor, and by the self-similarity of itineraries the same happens on the small-scale.
To make this rigorous, consider first the set $\Gamma_{\downarrow \downarrow \downarrow}$ of reduced tracks that start at the central station $s_{0,0}$ with at least three consecutive express tracks. Take a neighborhood $N$ of the union of all those tracks on which $f$ is univalent. % so in particular does not contain the critical point $0$
For every station $s$, the corresponding family $\Gamma^s_{\downarrow \downarrow \downarrow}$ of tracks starting at $s$ is the preimage of $\Gamma_{\downarrow \downarrow \downarrow}$  under some branch of an iterate of $f^{-1}$ which maps the central station $s_{0,0}$ to $s$.
Koebe's distortion theorem on this branch gives that
\begin{equation}
\diam(\shadow(s)) \asymp \dist (s,\mathcal J) \asymp \diam (N_s) \asymp \min_{\gamma \in \Gamma^s_{\downarrow \downarrow \downarrow}} \Length(\gamma)
\end{equation}
where $N_s$ is the preimage of $N$. Thus the length of $\eta_j$ decays at the same rate as $\dist(s,\mathcal J)$. %We claim that this rate is exponential.

We used the assumption that the itinerary starts with at least three express tracks to have the neighborhood $N$. For itineraries that do not start with three consecutive express tracks, we consider separately the family $\Gamma_{\downarrow \downarrow \leftarrow}$ of those itineraries which start with two express tracks followed by a left itinerary, and the similarly-defined $\Gamma_{\downarrow \downarrow \rightarrow}$. For each of these families, the previous argument holds since there is such a neighborhood $N$ of univalence.
\end{proof}
The shadows were not strictly necessary for the preceding proof, but they draw an analogy with similar estimates for the Farey tesselation of the disk.
\end{comment}

\subsection{Some Estimates and Notation}

To estimate the length of express tracks, we introduce the notations $s_n := s_{n, 0}$ and 
\begin{equation}
	\ell_k := \Length([s_{n},s_{n+1}]) = s_{n}-s_{n+1}.
\end{equation}

\begin{lemma} \label{lem-ell_n}
	The lengths ${\ell_n}$ satisfy:
	
	(i) 
	\begin{equation}
		\frac {|p-s_n|}{\ell_n} \to \infty,
	\end{equation}
	and 
	
	(ii)
	\begin{equation}
		\frac{\ell_n}{\ell_{n+1}} \to 1.
	\end{equation}
	In particular, for any $C > 0$, there is a sufficiently large integer $d$ such that
	\begin{align*}
		\ell_{m}+\ldots+\ell_{n} \geq C (\ell_m+\ell_n)
	\end{align*}
	whenever $|m-n| \geq d$.
\end{lemma}

\begin{proof} \leavevmode
	\emph{(i)} This follows from the affine conjugacy of the map $f$ to the map $g: z\mapsto z^2+z$, sending the fixed point $\frac 12$ of $f$ to $0$.
	% Both parts can be deduced explicitly as follows. The map $f$ is affinely conjugate to the map $g: z\mapsto z^2+z$, sending the fixed point $\frac 12$ of $f$ to $0$. A point near $0$ is attracted by $g^{-(\circ k)}$ at rate $\asymp 1/k$, hence the distance between consecutive points in the orbit decays at rate $\ell_k \asymp 1/{k^2}$. We now give an alternative proof of \emph{(ii)} based on a distortion argument.  % Why is that wrong?
	
	\emph{(ii)} For every $n \geq 1$, the ball $B_n = B(s_n, |p - s_n|)$
	is disjoint from the post-critical set $\PostCrit$, hence for every $m \geq n$ we have a univalent branch of $g_{m,n}=f^{\circ -(m-n)}$ on $B_m$ sending $s_m$ to $s_n$. 
	%Since $g_{m,n}\left([s_m,s_{m+1}]\right) = [s_n,s_{n+1}]$, we have $${\ell_m} \geq {\ell_n} \cdot \min_{z \in [s_n,s_{n+2}]} |g_{m,n}'(z)|.$$
	
	Denoting $R_n= |p-s_n|$ and $r_n = \ell_n+\ell_{n+1}$, we apply Harnack's inequality on $|g_{m,n}'|$ in the ball $B_n$ to obtain 
	\begin{align*}
		\frac{\ell_{m+1}}{\ell_m}  & \leq \frac{\ell_{n+1}}{\ell_n} \cdot \frac{\max_{z\in [s_n,s_{n+2}]} |g'(z)|}{\min_{z \in [s_n,s_{n+2}]} |g'(z)|}  \\ &
		 \leq \frac{\ell_{n+1}}{\ell_n} \cdot \frac{1+\frac{r_n}{R_n}}{1-\frac{r_n}{R_n}}
		 \\ &= \frac{\ell_{n+1}}{\ell_n} \cdot (1+o(1)),
	\end{align*}
	where the first inequality follows since $g_{m,n}\left([s_m,s_{m+1}]\right) = [s_n,s_{n+1}]$.
	
	%By part (i), we have $\frac{R_n+r_n}{R_n-r_n} \to 1$ as $n \to \infty$. 
	Together with the analogous lower bound, we obtain that the sequence  $a_n = \frac{\ell_{n+1}}{\ell_n}$ satisfies $c_n \leq \frac {a_m}{a_n} \leq d_n$ for every $m \geq n$, for some sequences $c_n,d_n$ both tending to $1$, which forces $a_n$ to converge. Let $L$ be the limit, then we have $L \leq 1$ since $\sum_{}\ell_n < \infty$. Moreover, we cannot have $L<1$ since this would imply that $\ell_n \asymp \sum_{k=n}^\infty \ell_k$, contradicting part (i). Thus $L=1$, as desired.
\end{proof}

\begin{definition}
The \emph{relative distance} of a curve $\gamma$ to the post-critical set $\PostCrit$ is 
$$\Delta (\gamma, \mathcal P) = \frac {\dist(\gamma, \mathcal P)}{\min (\diam (\gamma), \diam (\mathcal P))}.$$

%the smallest $\eta$ such that  $\diam{\gamma} \leq \eta \cdot \dist(\gamma, \mathcal P)$.  ???
We say that the curve $\gamma$ is \emph{$\eta$-relatively separated} from the post-critical set if $\Delta (\gamma, \mathcal P) \geq \eta$.  
%We say that the curve is \emph{relatively separated} from the post-critical set if it is $\frac 12$-relatively separated.
\end{definition}
 
If an itinerary $\gamma$ is relatively separated from $\PostCrit$, then preimages of $\gamma$ under $f$ have bounded distortion. In particular, if $\gamma$ is a quasiconvexity certificate, then Koebe's distortion theorem implies that $f^{-1}(\gamma)$ also is a certificate with a comparable constant.

\begin{comment}
\begin{lemma}
The Julia set $\mathcal J$ is \emph{quasi-locally connected}, meaning that there exists a constant $L>0$ such that for every two points $z,w \in \mathcal J$, the itinerary $\gamma_{z,w}$ is contained in the ball $B(z_1, L|z_1-z_2|)$.
\end{lemma}
\end{comment}

\begin{lemma}
There exists a constant $k>0$ such that for any pair of points $z_1, z_2 \in \mathcal J$, we have $|f(z_1)-f(z_2)| \leq k|z_1-z_2|$.
\end{lemma}

\begin{proof}
By Lagrange's theorem, we may take $k=\max_{z \in B} |f'(z)|$, where $B$ is any ball containing $\mathcal J$. 
\end{proof}

%The following lemma says that nearby points are connected by a short itinerary.
%\begin{lemma}
%	For every $\epsilon>0$, there exists a $\delta > 0$ such that for any pair of points $(w_1,w_2)$ with distance $|w_1-w_2|<\delta$, we have $\Length(\eta_{w_1,w_2}) < \epsilon$.
%\end{lemma}

\begin{comment}

\begin{lemma}
For any pair $(z_1,z_2)$ of distinct points on the Julia set of $f(z)=z^2+\frac 14$,
if $f(z_1)=f(z_2)$ then $|z_1-z_2|\geq c_0$ for a universal constant $c_0 > 0$.
\end{lemma}

\begin{proof}
If $f(z_1)=f(z_2)$ then $z_1^2=z_2^2$, and the claim is trivial.
%The claim even holds for any $z_1, z_2 \not \in B(0,c)$, where $B(0,c)$ is a small ball disjoint from the Julia set.
\end{proof}
\end{comment}

\begin{comment}
	A more general argument: Pulling back to the exterior unit disk, the result holds for $z \mapsto z^2$. \textbf{Not convincing? Use that $\mathcal J$ is a Jordan curve.}
\end{comment}

\subsection{Dynamics near the cusp}

% We proceed to show quasiconvexity of the itineraries, as sketched above. 
The purpose of the following definition is to organize points on the Julia set $\mathcal J$ according to their distance from the main cusp in an $f$-invariant way. We decompose the points of $\mathcal J$ according to the first \emph{departure}\,: the first time that the central itinerary made a turn.
\input{"marker path 3.tex"}

\begin{definition}
Let $n \in \mathbb N$. We define the $n$-th \emph{departure set} $I_{n, \mathbb D} \subset \partial \mathbb D ^*$ to be the set of points $\zeta \in \partial \mathbb D^*$ whose central itinerary $\eta_{\zeta}$ starts with $n$ express tracks, followed by a peripheral track.
See \cref{fig:departure-decomposition}.
\end{definition}

%$|C_1-C_2| \geq d+2$

This decomposition is invariant under $f_0$ in the sense that $f_0(I_{n+1, \mathbb D})=I_{n, \mathbb D}$, because of the invariance of $\eta _\zeta$.
Applying the Böttcher map $\psi$, we obtain a corresponding departure decomposition $I_n = \psi(I_{n, \mathbb D})$ of $\mathcal J$ that is invariant under $f$.

We now use this decomposition to analyze the case where the points $w_1,w_2$ lie in ``well-separated cusps''. Namely, suppose that
\begin{align} \label{parabolic separation}
	w_1 \in I_n, \quad w_2 \in I_m, \quad m-n > d,
\end{align}
where $d$ is a large enough integer, to be chosen later. 
This gives some control from below on $|w_1-w_2|$. We now bound the length of the itinerary $\eta=\eta_{w_1,w_2}$ from above.
We represent $\eta$ as a concatenation of three paths: the radial segment $\gamma_{m,n}=[s_{m,0},s_{n,0}]$ and the two other components, $\gamma _m$ and $\gamma_n$. See \cref{fig:Three-parts-of-eta} for the picture in the exterior unit disk.
Thus we have
\begin{equation}
	\Length(\eta)=\Length(\gamma_m)+\Length(\gamma_{m,n})+\Length(\gamma_n).
\end{equation}

\input{"departure-decomposition.tex"}


%\begin{definition}
%The \emph{inner diameter} of a connected set $\Omega \subset \mathbb C$ is its diameter relative to the inner metric,
%\begin{equation}
%	\operatorname{InnerDiameter}(\Omega)=\sup _{p,q\in \Omega} \inf_\gamma \Length(\gamma)
%\end{equation}
%where the infimum is over all paths $\gamma$ in $\Omega$ that connect the point $p$ to $q$.
%\end{definition}
%
%We have the following:
%\begin{lemma}
%	\begin{equation*}
%		\operatorname{InnerDiameter}(\psi(I_n)) \lesssim \ell_n
%	\end{equation*}
%	where the hidden constant is independent of $n$.
%\end{lemma}

The condition $m-n \geq d$ prevents the line segment $\gamma _{m,n}$ from being small in comparison to $\gamma_m$ and $\gamma_n$:
\begin{proposition} \label {case-3-proof} %\leavevmode
There exists an integer $d$ large enough so that whenever $m-n \geq d$, we have: 
	\begin{equation}
		\Length(\gamma_{m,n}) \lesssim |w_1-w_2|.
	\end{equation}
\end{proposition}

We henceforth fix a value of $d$ as in the proposition.
\begin{proof}
The triangle inequality gives
\begin{equation} \label{eq:triang}
	\begin{split}
		\Length(\gamma_{m,n}) & \leq |w_1-w_2|+\Length(\gamma_m)+\Length(\gamma_n),
	\end{split}
\end{equation}
and Koebe's distortion theorem on iterates of $f^{-1}$ shows that
\begin{equation} \label{eq:gamma_m_not_large}
		\Length(\gamma_m) \leq C \ell_m
	\end{equation}
for some constant $C \geq 0$. Notice that this holds for $m=1$ by the proof of \cref{prop:finite-length}, which gives a uniform bound on the length of an itinerary.


%\begin{equation}
%	\Length(\gamma_m) \lesssim \Length(\gamma_{m,n}).
%\end{equation}
% TODO: Move some of these estimates to the estimates section
% and (\ref{eq:gamma_m_not_large}),
%\begin{equation} \label{eq:triang}
%	\begin{split}
%\Length(\gamma_{m,n}) & \leq |w_1-w_2|+\Length(\gamma_m)+\Length(\gamma_n) \\ 
%& \leq |w_1-w_2|+C(\ell_m + \ell_n).
%	\end{split}
%\end{equation}
By \cref{lem-ell_n}, there exists an integer $d$ such that
\begin{align}
C(\ell_m + \ell _n) \leq \Length(\gamma_{m,n})
\end{align}
whenever $m-n \geq d$.

which together with \eqref{eq:triang} concludes the proof.
\end{proof}

%Fix a small ball $B$ around the main cusp $1/2$ and trace its preimages until we encounter the points $z_1,z_2$.

\subsection{Quasiconvexity: three special cases}

We now show that the itineraries $\eta_{w_1,w_2}$ are certificates in three special cases. To state them, we introduce some notation.
\subsubsection{Notation}
% TODO: Move this definition to somewhere else
For each $n$, we denote by  $\alpha_n$ the union of the two outermost tracks emanating from the station $s_{m,0}$. Explicitly, $\alpha_n=\alpha_{n,\rightarrow}\cup\alpha_{n,\leftarrow}$ where $\alpha_{n,\rightarrow}$ has after each express track a peripheral track whose pullback to the exterior unit disk is pointing clockwise, and $\alpha_{n,\leftarrow}$ is defined analogously with anti-clockwise turns. Notice that the curves $\alpha_i$ are pairwise disjoint since this holds for their pullbacks to the exterior unit disk.

We define the constants $C_1, C_2, \epsilon$ as follows. We first choose $C_1 \ge 2$, then we let $C_2 = C_1 + d+2$ and choose $\epsilon >0$ small enough so that we have 
\begin{equation}
	\dist(\alpha_{C_2}, \alpha_{C_1}) \geq k \epsilon.
\end{equation}

The constant $C_2$ was chosen so that for any pair $(m,n)$ of integers, we have at least one of the following three cases: either $m,n$ are both greater than $C_2$, or both are smaller than $C_1$, or $|m-n| > d$.

\subsubsection{Three Special Cases}
In this section we treat the following special cases:
\begin{enumerate}
	\item $|w_1-w_2| \geq  \epsilon$, $\quad |m-n|<d$, $\quad m,n < C_2$, $\quad m,n \geq 2$; %relatively separated
	\item $|w_1-w_2| \geq \epsilon$, $\quad |m-n|<d$, $\quad  m,n> C_1$; %relatively separated
	\item $|w_1-w_2| \leq k \epsilon$, $\quad |m-n| \geq d$.
\end{enumerate}

Notice that Case 2 overlaps with Case 1.
%\emph{Case 1.} The points $w_1,w_2$ are ``far apart'': $|w_1-w_2| \geq \epsilon$ for some fixed $\epsilon >0$.
%In this case the length of the itinerary $\eta_{w_1,w_2}$ is bounded above by compactness, so we have a uniform bound on the quasiconvexity constant.

% Cases \emph{1.} and \emph{2.} follow since they force the itinerary to be relatively separated from the post-critical set:
 We denote the domain enclosed by $\alpha_m, \alpha_n$ and $\mathcal J$ by $\mathcal K_{m,n}$, and denote the domain enclosed by $\mathcal J$ and $\alpha_n$ by $\mathcal K_n$.

\begin{lemma}
Let $w_1 \in I_m$ and $w_2 \in I_n$, for $n \geq m \geq 2$. Then the itinerary $\eta_{w_1,w_2}$ is contained in the domain $\mathcal K_{m,n+1}$.
\end{lemma}

%\begin{corollary}
%Let $w_1,w_2$ be as in case 2. Then the itinerary $\eta_{w_1,w_2}$ is relatively separated from the postcritical set.
%\end{corollary}

\begin{lemma} \label{case 1 rel. sep}
Let $w_1,w_2 \in \mathcal J$. In each of the three special cases, the itinerary $\gamma _{w_1,w_2}$ is a quasiconvexity certificate. In Cases 1 and 2, $\gamma _{w_1,w_2}$ is relatively separated.
\end{lemma}

\begin{proof}
\emph{Case 1.} The itinerary, in this case, is contained in the domain $K_{2, C_2+1}$.
%\begin{equation*}
%. %$R=2L+|s_{0,0}|.
%\end{equation*}
This domain is $\eta$-relatively separated, for some $\eta>0$, because $\dist(K_{2, C_2+1}, \mathcal P)>0$. 
%Indeed, the distance is realized at the point $p$ and $\dist(\mathcal J \setminus K_{C_2+1}, p)>0$.

\emph{Case 2.}
Assuming without loss of generality that $n \geq m$, notice that the itinerary is contained in the domain $\mathcal K_{m,n+1}$, which has a positive relative distance to the cusp $p$.

\emph{Case 3} is the content of \cref{case-3-proof}.
\end{proof}


%\begin{corollary}
%	Let $d \gg 1$ be a positive integer. There exists a constant $A>0$ such that for every pair of points $w_1,w_2$ that satisfy Case 3, the itinerary $\eta_{w_1,w_2}$ is $A$-quasiconvex.
%\end{corollary}

\subsection{Quasiconvexity: general case}
We apply a stopping time argument to promote the quasiconvexity of $\eta_{w_1,w_2}$ to quasiconvexity of $\eta_{z_1,z_2}$, thereby proving the following theorem:

\begin{theorem} \label{quasiconvex-cauliflower}
	The domain $\mathrm{Exterior}(\mathcal{J})$ is quasiconvex, with the itineraries $\eta_{z_1,z_2}$ as certificates.
\end{theorem}

\begin{proof}[Proof. (Parabolic Conformal Elevator on $\mathcal J$)] \label{parabolic-elevator}
Let  $(z_1,z_2)$ be a pair of points in $\mathcal J$. Repeatedly apply $f$ to $(z_1, z_2)$ until either of the three special cases occurs. Denote by $w_i=f^{\circ N}(z_i)$ the resulting points. We have already proved that the itinerary $\eta_{w_1,w_2}$ satisfies
\begin{equation*}
	\Length(\eta_{w_1,w_2}) \leq A |w_1-w_2|,
\end{equation*}
for some $A>0$. We deduce that the original pair of points $(z_1,z_2)$ enjoys a similar estimate,
\begin{equation*}
	\Length(\eta_{z_1,z_2}) \leq C |z_1-z_2|,
\end{equation*}
where $C$ depends only on $A$.

In Cases 1 and 2, we are done by \cref{case 1 rel. sep}. In Case 3, note that
the itinerary $\eta_{w_1,w_2}$ is contained in $\mathcal K_2$ and let $\mathcal K_{-2}$ be the preimage of $\mathcal K_2$ under $f$ that contains the negative preimage $f^{-1}(p)=-\tfrac 12$ of the cusp $p$. The set $\mathcal K_{-2}$ is relatively separated from $\PostCrit$ and contains the curve $f^{\circ (N-1)}(\eta_{z_1,z_2})$.
%This follows from Koebe's distortion theorem applied $N$ times on the univalent function $f^{-1}$, but some care is due in constructing the inverse branches in a neighborhood of the itineraries.
We can thus conclude by applying Koebe's distortion theorem on a suitable branch of $f^{\circ (-N)}$:

We take the branch of $f^{\circ -(N-1)}$ on $\mathcal K_{-2}$ sending the pair of points $(f^{-1}(w_1),f^{-1}(w_2))$ to $(z_1,z_2)$, and compose it with the branch of $f^{-1}$ that sends $p$ to $-p$. We obtain
\begin{equation}
	\frac{\Length(\eta_{z_1,z_2})}{|z_1-z_2|} \asymp \frac{\Length(\eta_{w_1,w_2})}{|w_1-w_2|},
\end{equation}
hence $\eta_{z_1,z_2}$ is a certificate.
\end{proof}


\begin{comment}
%We again consider the branch of $f^{}-1}$
%and we obtain quasiconvexity of $\eta_{z_1,z_2}$ as in case (\emph{i}).


%We define the level of each of $z_1, z_2$ to be the level of the corresponding ball containing them.
%By the stopping criterion, 
%, which implies that
%	\begin{equation*}
%		|w_1-w_2| \lesssim |z_1-z_2| \cdot \|(f^{\circ N})'|_{\mathcal J}\|_{\infty}
%	\end{equation*}
%	and that
%	\begin{align*}
%		\Length (\eta_{w_1,w_2}) = \Length(f^{\circ N}(\eta_{z_1,z_2})) \\
%		\asymp \Length(\eta_{z_1,z_2}) \cdot |(f^{\circ N})'|.
%	\end{align*}
%	The conclusion follows. From now suppose that we are in case (ii).
%By the choice of $w_1,w_2$, for every $k=0,\dots,N$ there is an annulus that separates $f^{\circ k}(z_1),f^{\circ k}(z_2)$ from the post-critical set of $f$, with modulus bounded from below uniformly in $N$.
%By Koebe's distortion we have 
%\begin{align*}
%|f^{\circ k}(z_1)-f^{\circ k}(z_2)|\geq c |z_1-z_2|
%\end{align*}
%for some constant $c>1$.

%Indeed, in this case there is an annulus of definite modulus that has both $z_1,z_2$ in its bounded complementary component and has all points of $\CriticalOrbit(f)$ in the unbounded complementary component.
%
%Instead of keeping track of the distance to $\mathrm{CriticalOrbit}(f)$, it is enough to consider the distance to the fixed point $1/2$, which is its only accumulation point on $\mathcal J$.
%
%Thus by repeatedly applying $f$ we either manage to separate $z_1,z_2$ a definite distance apart, in which case the argument for the hyperbolic case works, or we manage to make
%\begin{align*}
%\big|f^{\circ m} (z_1)-\tfrac 12\big| \leq c |f^{\circ m} (z_1)-f^{\circ m} (z_2)|
%\end{align*}
%for a constant $c$.

\end{comment}}

%This concludes the proof.

\begin{comment}
\subsubsection*{Diameter Comparisons}

Denote by $\diam\left(A\right)$ the Euclidean diameter of a set $A$,
and by $i.\diam\left(A\right)$ the inner diameter on the escaping
set, i.e.\ the diameter induced by the Riemann mapping from the escaping
set to the disk.

\textbf{Claim. }$\diam T_{p,n}\asymp i.\diam T_{p,n}$.

\textbf{Claim. }$\diam U_{p,n}\asymp i.\diam U_{p,n}$.

\textbf{Observation. }Let $R>0$ be the distance from the post-critical
set to the Julia set. Consider the ball $B=B(z,\frac{R}{2})$.

Since $B$ is disjoint from the post-critical set, none of its preimages
$f^{-n}\left(B\right)$ contains the critical point $0$.

Thus $f$ restricted on each such preimage is univalent and 2-to-1.

By Koebe's distortion Theorem, applied to iterates $f^{\circ n}$,
the diameter of the preimages of $B$ changes by at most a multiplicative
constant.

This principle shows that all points $z$ on the Julia set $J(p_{c})$
are accessible, since after repeated applications of $f^{n}(B)$ we
get a topological ball of definite size which we may use to construct
an explicit path exiting $K$ in a rectifiable way.

\section{Accessibility in the Parabolic Case}

The basic strategy will be to investigate the geometry near the main
cusp $p=\frac{1}{4}$, since all other cusps are preimages of $p$
under $f$,

whence the principle of the conformal elevator reduces the general
case to the case of a point sufficiently close to the main cusp $p$.

To implement this strategy, we take some arbitrary other point on
the Julia set, connect it by a curve $\gamma$ to $a$, then consider
all preimages of $\gamma$.

In general, consecutive images near a parabolic point have consecutive
distances comparable to $\frac{1}{n^{2}}$ , where $n$ is the index
of the preimage.
\end{comment}

\bibliographystyle{acm}
\bibliography{quasiconvex_cauliflower}
\end{document}
