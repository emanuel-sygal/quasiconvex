% \usepackage{pgfplots}
% \usetikzlibrary{calc}
% \usepgfplotslibrary{colormaps}


\RequirePackage[l2tabu, orthodox]{nag}
\documentclass[12pt]{article}
\usepackage{geometry}                % See geometry.pdf to learn the layout options. There are lots.
\geometry{letterpaper}
\usepackage[autostyle=false, style=english]{csquotes}
\usepackage{mathtools}
%\MakeOuterQuote{"}
\setlength{\marginparwidth}{2cm}

\usepackage{nomencl}
\makenomenclature
% Run the following command in terminal to update:
% makeindex "main.nlo" -s nomencl.ist -o "main.nls"
\nomenclature{$f, f_c$}{The map $z \mapsto z^2+c.$}
\nomenclature{$\mathcal J$}{The Julia set of $f$.}
\nomenclature{$K$}{The filled Julia set of $f$.}
\nomenclature{$\gamma_{z_1, z_2}$}{The track connecting $z_1$ and $z_2$. It can be either angular (\enquote{peripheral}) or radial (\enquote{express}).}
\nomenclature{$\eta_{z_1, z_2}$}{The itinerary connecting two points. When $z_1$ and $z_2$ are stations, this is the same as $\gamma_{z_1,z_2}$.}
\nomenclature{$A_\infty(f_c)$}{The exterior of the Julia set of $f_c$. The complement of $K_c$.}
\nomenclature{$\mathrm{Exterior}(\mathcal{J})$}{ $\;$ An Alternative notation for $A_\infty(f_c)$.}
\nomenclature{$\psi$}{The Böttcher coordinate $\mathbb D ^* \to \Exterior(\mathcal J)$ conjugating $f_0$ and $f$.}
\nomenclature{$s_{n,k}$}{A station in $\mathbb D^*$ or its image under $\psi$.}
\nomenclature{$\Delta (\gamma, \mathcal P)$}{The relative distance to the post-critical set.}
\nomenclature{$I_{n}$}{The $n$-th departure set.}
\nomenclature{$\ell_n$ }{$\Length([s_{n},s_{n+1}]) = s_{n,0}-s_{n+1,0}.$}
\nomenclature{$\alpha_n$}{the union of the two outermost tracks emanating from the station $s_{n,0}$.}


% Prevents line breaks at math inline
\relpenalty=9999
\binoppenalty=9999

\usepackage{graphicx,color,mathtools}
\usepackage{amssymb,amsmath,amsthm,mathrsfs}
%\usepackage[all,cmtip]{xy}
\usepackage{comment}
%\usepackage{todonotes}
\usepackage{enumitem}   
\usepackage{tikz} 
\usepackage{pgfplots}
\pgfplotsset{compat=1.18}
\usepackage{graphicx}
\usepackage{xcolor}

\usepackage{hyperref}
\usepackage[capitalize,nameinlink,noabbrev]{cleveref}

\usepackage[pdftex,bookmarks,pdfnewwindow,plainpages=false,unicode,pdfencoding=auto]{}


\DeclareGraphicsRule{.tif}{png}{.png}{`convert #1 `dirname #1`/`basename #1 .tif`.png}
\linespread{1.2}

\numberwithin{equation}{section}

\newtheorem{theorem}{Theorem}[section]
\newtheorem{conjecture}[theorem]{Conjecture}
\newtheorem{lemma}[theorem]{Lemma}
\newtheorem{claim}[theorem]{Claim}
\newtheorem{proposition}[theorem]{Proposition}
\newtheorem{corollary}[theorem]{Corollary}

\theoremstyle{remark}
\newtheorem*{remark}{Remark}

\theoremstyle{definition}
\newtheorem{definition}[theorem]{Definition}
\newtheorem{example}[theorem]{Example}

\DeclareMathOperator{\crit}{crit}

\DeclareMathOperator{\Hdim}{H.dim }
\DeclareMathOperator{\supp}{supp }
\DeclareMathOperator{\diam}{diam}
\DeclareMathOperator{\BV}{BV}
\DeclareMathOperator{\shadow}{Shadow}
\DeclareMathOperator{\idiam}{i.diam}
\DeclareMathOperator{\aut}{Aut}
\DeclareMathOperator{\hol}{Hol}
\DeclareMathOperator{\hyp}{hyp}
\DeclareMathOperator{\dist}{dist}
\DeclareMathOperator{\area}{Area}
\DeclareMathOperator{\loc}{loc}
\DeclareMathOperator{\Mod}{Mod}
\DeclareMathOperator{\Arg}{Arg}
\DeclareMathOperator{\Length}{Length}
\DeclareMathOperator{\HypLength}{HypLength}
\DeclareMathOperator{\CriticalOrbit}{CriticalOrbit}
\DeclareMathOperator{\Postcrit}{\mathcal P}
\DeclareMathOperator{\PostCrit}{\mathcal P}
\DeclareMathOperator{\Closure}{Closure}
\DeclareMathOperator{\RadialSuccessor}{RadialSuccessor}
\DeclareMathOperator{\Exterior}{Exterior}

\global\long\def\R{\mathbb{R}}%
\global\long\def\C{\mathbb{C}}%
\global\long\def\N{\mathbb{N}}%
\global\long\def\Z{\mathbb{Z}}%
\global\long\def\D{\mathbb{D}}%
\global\long\def\H{\mathbb{H}}%
\global\long\def\do#1#2{\frac{\partial#1}{\partial#2}}%
\global\long\def\nor#1{\left|#1\right|}%
\global\long\def\o#1{\overline{#1}}%
\global\long\def\norm#1{\left\Vert #1\right\Vert }%
\global\long\def\\sphere{\hat{\mathbb{C}}}%
\global\long\def\ep{\varepsilon}%
\global\long\def\mp#1{\left\langle #1\right\rangle }%
\global\long\def\lap{\triangle}%
\global\long\def\dz{\mathrm{\mathrm{dz}}}%
\global\long\def\dt{\mathrm{\mathrm{dt}}}%
\global\long\def\dtheta{\mathrm{\mathrm{d\theta}}}%
\global\long\def\dx{\mathrm{\mathrm{dx}}}%
\global\long\def\dw{\mathrm{\mathrm{dw}}}%
\global\long\def\T{\mathrm{\mathbb{T}}}%
\begin{document}

\section{Introduction}
A domain $\Omega\subseteq\C$ is called \textbf{quasiconvex }if its
intrinsic metric is comparable to the ambient Euclidean metric. Explicitly,
this means that there exists a constant $A\geq1$ such that every
two points $z_{1},z_{2}\in\Omega$ have a rectifiable path $\gamma:\left[0,1\right]\to\Omega$
connecting them which satisfies %
\begin{comment}
Let $X$ be a path-connected subset of the plane. A rectifiable path
$\gamma:\left[0,1\right]\to X$ is called $C$\textbf{-quasiconvex,
}for some constant $C$, if its length satisfies
\end{comment}
{} 
\[
\mathrm{Length}(\gamma)\leq A\cdot\left|z_{1}-z_{2}\right|.
\]

We call such a path $\gamma$ a \emph{quasiconvexity witness }for\emph{
$z_{1},z_{2}$.}%
\begin{comment}
The space $X$ is called \textbf{quasiconvex }if there is a constant
$C$ such that every pair of points $z_{1},z_{2}\in X$ can be connected
by a $C$-quasiconvex path. In other words, the intrinsic metric on
$X$ is comparable to the ambient Euclidean metric. 
\end{comment}

If $\Omega$ is the interior of a Jordan curve, then by \cite[Corollary F]{key-1}
it is enough to find certificates for points $z_{1},z_{2}$ that are
on the boundary curve $\partial\Omega$.%
\begin{comment}
It is also shown in \cite{key-1} that any quasidisk is quasiconvex.
\end{comment}

Our interest in quasiconvexity stems from its connection with the
John property: If $\Omega$ is a quasiconvex Jordan domain, then the
interior of its complement is John. See \cite[Corollary 3.4]{key-1}
for details. 

We want to show that the exterior of the developed deltoid is quasiconvex. 

We show that the exterior of the cauliflower, $\mathcal{J}^{\text{exterior}}(z^{2}+1/4)$,
is quasiconvex. We then adapt our argument to the exterior of the developed deltoid.

\begin{comment}
The Filled Julia set of $z^{2}+1/4$, called the cauliflower, has
an inward-pointing cusp and hence is not quasiconvex.
\end{comment}

\begin{comment}
Thus, for any $c$ in 

For values $c$ in quadratic polynomials $f_{c}(z)=z^{2}+c$
\end{comment}
\begin{comment}
If $f_{c}$ has an attracting fixed point then its Julia set $\mathcal{J}(f_{c})$
is a quasicircle, hence its interior and exterior are both quasiconvex.
This is the case for values of $c$ in the main cardioid of the Mandelbrot
set.
\end{comment}
\begin{comment}
, i.e. for 
\[
c\in\left\{ -\frac{\lambda}{2}-\frac{\lambda^{2}}{4}:\,\left|\lambda\right|<1\right\} .
\]
\end{comment}

\subsection{Sketch of the argument}

We show quasiconvexity by an explicit construction of paths connecting
given points on the boundary.\textbf{}

%We use the Böttcher coordinate on the exterior unit disk 
We use the conjugation of $f_{c}:z\mapsto z^{2}+c$ on the exterior
of the Julia set to the map $z^{2}$ on the exterior of the unit disk
$\D^{*}$. We decompose $\D^{*}$ into Carleson boxes invariant under
$z^{2}$ and connect points on the unit circle by traveling on the
boundary of these boxes. Both the boxes and the path on their boundary
respect the dynamics of $z^{2}$ there, which allow us to transport
these paths from $\D^{*}$ to the exterior of $\mathcal{J}(f_{c})$ 

We construct two collections of curves, which we call "express"
and "peripheral" tracks. We use the metaphor
of a train traveling between the endpoints and switching between tracks.
We stitch the quasiconvexity certificates out of these tracks.

The construction of the train tracks is done first in the case of
the exterior unit disk ($c=0$), and then transported to the 
$c=1/4$ case using the Böttcher coordinate. 
%Thus we first explain the method in the case of $c=0$. 

For clarity of exposition, we first show how the method works in the hyperbolic case, in which the conformal elevator makes the argument
simpler, and only then prove the parabolic case $c=1/4$. 
%We do this even though $J_{0.1}^{\text{exterior}}$ is known in advance to be quasiconvex, in virtue of being a quasidisk.


\section{Power map}

The exterior $\D^{*}=\left\{ \left|z\right|>1\right\} $ of the unit
disk is trivially quasiconvex by connecting points along the perimeter
of the circle. However, these paths follow the boundary too closely
and their length would blow up if we transport them to the exterior
of $\mathcal{J}(f_{c})$, $c\neq0$, via the Riemann map. Instead,
we connect points by traveling along the boundaries of Carleson boxes
which we now define.
\begin{definition}
Let $n\in\N_{0}$ and $k\in\left\{ 0,\ldots,2^{n}-1\right\} $. We
call the set
%\[B_{n,k}=\exp\left(\biggl(2^{-n-1}\log2,\,2^{-n}\log2\biggl]\times\biggl(\frac{k}{2^{n}}2\pi,\,\frac{(k+1)}{2^{n}}2\pi\biggl]\right)\]

\[
B_{n,k}=\left\{ z:\quad\left|z\right|\in\biggl(2^{2^{-n}},2^{2^{-n-1}}\biggl],\qquad\mathrm{arg}(z)\in\biggl(\frac{k}{2^{n}}2\pi,\frac{(k+1)}{2^{n}}2\pi\biggl]\right\} 
\]

an \emph{$\boldsymbol{f_{0}}$}\textbf{\emph{-Carleson box}}\emph{.} 

Observe that for a fixed $n$, the union $\bigsqcup_{k=0}^{2^{n}-1}B_{k,n}$
is a partition of the annulus 
\[
\left\{ 2^{2^{-n-1}}<\left|z\right|\leq2^{2^{-n}}\right\} 
\]
 into $2^{n}$ equally-spaced sectors.

The\emph{ }\textbf{\emph{Carleson }}\emph{$\boldsymbol{f_{0}}$}\textbf{\emph{-box
decomposition}} is the partition of $\D^{*}$ obtained by $f_{0}$-Carleson
boxes:
\[
\D^{*}=\left\{ z:\,\left|z\right|>2\right\} \sqcup\bigsqcup_{n=0}^{\infty}\bigsqcup_{k=0}^{2^{n}-1}B_{n,k}.
\]

The crucial property of this partition is its invariance under $f_{0}$,
stemming from the relation

\[
f_{0}\left(B_{n+1,k}\right)=B_{n,\lfloor\frac{k}{2}\rfloor}.
\]
\end{definition}

\begin{comment}
We describe the motion along quasigeodesics as a train drive. Accordingly,
we proceed to define "stations"{} and "tracks".
\end{comment}

\begin{definition}
The \emph{central station }is the point\emph{ $s_{0,0}=2$. Stations
}are the iterated preimages of the central station under the map $f_{0}:z\mapsto z^{2}$.
We index them as 
\[
s_{n,k}=2^{2^{-n}}\exp\left(\frac{k}{2^{n}}2\pi i\right),\qquad n\in\N_{0},\quad k\in\left\{ 0,\ldots,2^{n}-1\right\} .
\]

Stations are naturally structured in a binary tree, where the root
is the central station $2$ and the children of a node are its preimages.
The $2^{n}$ stations of generation $n$ in the tree are equally spaced
on the circle $C_{n}=\left\{ \left|z\right|=2^{1/2^{n}}\right\} $. 


\end{definition}

We next lay two types of "train tracks" on the boundaries of Carleson boxes, which we use to travel between stations.

\begin{definition}
Let $s=s_{n,k}$ be a station.

1. The \emph{peripheral neighbors} of $s$ are $s_{n,\left(k\pm1\right)\mod2^{n}}$,
the two stations adjacent to $s_{n,k}$ on $C_{n}$.

2. Given a peripheral neighbor $s'$ of $s$, the \emph{peripheral
	track }$\gamma_{s,s'}^{\text{peripheral}}$ between these stations
is the short arc of the circle $C_{n}$ connecting $s$ to $s'$.

3. The \emph{radial successor} of $s$ is $\mathrm{RadicalSuccessor}(s)=s_{n+1,2k}$, the unique station of generation $n+1$ on the radial segment $[0,s]$.

4. The \emph{Express track} $\gamma_{s}^{\text{express}}$ from $s$ is the radial segment $[s,\mathrm{RadicalSuccessor}(s)]$.

5. A \emph{train journey} is a concatenation of tracks. A journey is identified with its sequence of stations. %  $\left(\sigma_{i}\right)$
\end{definition}

Notice that the tracks preserve the dynamics: applying $z\mapsto z^{2}$
on a peripheral track between $s,s'$ gives a peripheral track between the
parents of $s,s'$ in the tree, and likewise for an express track.
 
\begin{lemma}
There is a family $\{\eta_{z}:z\in\partial\D\}$ of journeys with
the following properties:

1. Every $\eta_{z}$ is a journey $\left(\sigma_{0},\sigma_{1},\ldots\right)$
from the central station $\sigma_{0}=s_{0,0}=2$ to $\lim\sigma_{k}=z$.

2. There are no two consecutive peripheral tracks in $\eta_{z}$.

3. $\mathrm{Length}(\sigma_{k})\lesssim2^{-k}$ uniformly in $z$. 

4. The journeys are invariant under $f_{0}$, in the sense that $$f_{0}(\eta_{z})=\eta{}_{f_{0}(z)}\cup[2,4]$$
for every $z\in\partial\D$.
\end{lemma}

\begin{proof}
Let $z=z_{1}=\exp(2\pi i\theta)\in\partial\D$. We choose the stations
$\sigma_{i}$ inductively in pairs, in a greedy manner. In each step
we drive peripherally to the station closest to $z_{1}$ and then
drive to its radial successor. See Figure \ref{fig:Carleson1}. 
\end{proof}
\input{"marker path.tex"}

\begin{proof}
For the first station $\sigma_{1}$ we have no choice and we drive
to the station $\sigma_{1}=s_{1,0}=\sqrt{2}$.

Suppose that we already chose the stations $\left(\sigma_{0},\ldots,\sigma_{2k-1}\right)$.
Then from $\sigma_{2k-1}$ we drive to the station $\sigma_{2k}$
on the same circle, $\left|\sigma_{2k-1}\right|=\left|\sigma_{2k}\right|$,
that minimizes the angular distance $\left|\Arg\left(z_{1}\right)-\Arg\left(\sigma_{2k}\right)\right|$. 

The minimizer $\sigma_{2k}$ is adjacent peripherally to $\sigma_{2k-1}$,
since the angular distance between stations on $C_{n}$ is $\frac{2\pi}{2^{n}}$
and we maintain the invariant $\left|\Arg\left(z_{1}\right)-\Arg\left(\sigma_{2k}\right)\right|\leq\frac{2\pi}{2^{k}}$
throughout the journey. Thus the length of the peripheral track $\gamma_{\sigma_{2k-1},\sigma_{2k}}^{p}$
is either $r_{n}\cdot\frac{2\pi}{2^{k}}=2^{1/2^{k}}\frac{2\pi}{2^{k}}$
or $0$ (in case $\sigma_{2k-1}=\sigma_{2k}$), and in any case the
length is at most $\lesssim\frac{1}{2^{k}}$ for a global hidden constant.
The length of the $k$-th express track decays exponentially due to
the invariance under $f_{0}$. Explicitly it is $2^{1/2^{k}}-2^{1/2^{k+1}}\leq2^{2^{-k}}-1\lesssim2^{-k}$
since $\lim_{x\to0}\frac{2^{x}-1}{x}=\log2$.

Thus the total length of the journey is  bounded uniformly in $z$.
\begin{comment}
property $4$ is automatic from property $3$, and the rest are evident
from the construction.
\end{comment}
\end{proof}
%
\begin{comment}
\begin{lemma}
In particular, the length of a suffix $\sigma_{k}+\sigma_{k+1}+\ldots$
decays exponentially in $k$, uniformly in $z$.
\end{lemma}
\end{comment}

We call $\eta_{z}$ the \emph{central journey} of $z$.
\begin{theorem}
The domain $\D^{*}$ is quasiconvex with quasiconvexity certificates
that are journeys.
\end{theorem}

\begin{proof}
Fix two points (“terminal stations") $z_{1},z_{2}\in\partial\D$.
Let $\eta_{z_{1}}=\left(\sigma_{n}^{1}\right)_{n=0}^{\infty},\eta_{z_{2}}=\left(\sigma_{n}^{2}\right)_{n=0}^{\infty}$
be their central journeys, connecting each terminal to the central
station.

Let $\left(\sigma_{0},\ldots,\sigma_{N}\right)$ be the maximal common
prefix of $\eta_{z_{1}}$ and $\eta_{z_{2}}$. Let $\eta_{z_{i}}^{\text{truncated}}=\left(\sigma_{N},\sigma_{N+1}^{i},\ldots\right)$
be the truncated paths. By the maximality of $N$, we have that $\eta_{z_{1}}^{\text{truncated}}$
and $\eta_{z_{2}}^{\text{truncated}}$ are two journeys with a common
starting point, so we can concatenate them to obtain a bi-infinite
journey 
\[
\eta_{z_{1},z_{2}}=\left(\ldots\sigma_{N+2}^{2},\sigma_{N+1}^{2},\sigma_{N},\sigma_{N+1}^{1},\sigma_{N+2}^{1},\ldots\right)
\]
 connecting $z_{1}$ and $z_{2}$.
\input{"marker path 2.tex"}

We conclude the proof by showing that $\mathrm{Length}\left(\eta_{z_{1},z_{2}}\right)\lesssim\left|z_{1}-z_{2}\right|$.

As $\left|z_{1}-z_{2}\right|\asymp\left|\theta_{1}-\theta_{2}\right|$
and $\mathrm{Arg}\left(z_{i}\right)\propto\theta_{i}$, it is equivalent
to show
\[
\mathrm{Length}\left(\eta_{z_{1},z_{2}}\right)\lesssim\left|\Arg\left(z_{1}\right)-\Arg\left(z_{2}\right)\right|.
\]

By the choice of $N$, 
\[
\left|\Arg\left(z_{1}\right)-\Arg\left(z_{2}\right)\right|\leq\frac{2\pi}{2^{N}}.
\]

Thus it is enough to prove that $\mathrm{Length}\left(\eta_{z_{1},z_{2}}\right)\lesssim2^{-N}$.
But 
\[
\mathrm{Length}\left(\eta_{z_{1},z_{2}}\right)=\mathrm{Length}\left(\eta_{z_{1}}^{\text{truncated}}\right)+\mathrm{Length}\left(\eta_{z_{2}}^{\text{truncated}}\right),
\]
so it is enough to observe that 
\[
\mathrm{Length}\left(\eta_{z_{i}}^{\text{truncated}}\right)\lesssim\sum_{k=N}^{\infty}\frac{1}{2^{k}}\lesssim2^{-N}
\]
by part (3) of the previous lemma.

\end{proof}

\section{Hyperbolic Map}

Throughout this section we fix an arbitrary $c\in\left(-\frac{3}{4},\frac{1}{4}\right)$ and denote $f_c: z\mapsto z^2 + c$ by $f$.
Since the critical point $0$ is in the filled Julia set of $f$,
there is a conformal map $\psi:\D^{*}\to\mathcal{J}^{\text{exterior}}(f)$
conjugating $f$ to $z^{2}$, i.e. $f\circ\psi(z)=\psi\circ f_{0}(z)$
for every $z\in\D^{*}$.

The map $\psi$ extends to a homeomorphism between the circle
$\partial\D$ and the Julia set $\mathcal{J}(f)$ by Carathéodory's
theorem, since $\mathcal{J}$ is a Jordan curve.

All $f_{0}$-invariant constructions carry over from $\D^{*}$ to
$\mathcal{J}^{\text{exterior}}(f)$, and now they are $f$-invariant:
%We have $f$-Carelson boxes $B_{n,k,c}=\psi(B_{n,k})$ and
%an $f$-Carelson decomposition 
%\[\mathcal{J}^{\text{exterior}}(f)=\psi\left(\left\{ \left|z\right|>2\right\} \right)\sqcup\bigsqcup_{n,k,}B_{n,k,c},\]
We have stations $s_{n,k,c}=\psi(s_{n,k})$ and likewise tracks.
The express tracks lie on the external rays of $\psi$, and the
peripheral tracks are on the level sets of $\psi$, or equivalently
on the equipotentials of Green's function.

As an example of how invariance carries other, applying $\psi$
to both sides of the equation 
\[
f_{0}\left(B_{n+1,k}\right)=B_{n,\lfloor\frac{k}{2}\rfloor}
\]
gives the corresponding relation 
\[
f\left(B_{n+1,k,c}\right)=B_{n,\lfloor\frac{k}{2}\rfloor,c}.
\]

We observe that the central station is still on the real line:

\begin{lemma}$\psi(\mathbb{D}^{*}\cap\R)\subseteq\R$. In particular,
$\psi(s_{0,0})\in\R$.
\end{lemma}
\begin{proof}
This is true by symmetry of $\D^{*}$ and $\mathcal{J}^{\text{exterior}}(f)$
with respect to $\R$. Formally, $\overline{\psi}(\overline{z})$
is another conformal map with the same conjugation relation, so by
uniqueness of the Böttcher coordinate (with a given derivative at
$\infty$) we obtain $\psi(z)=\overline{\psi}(\overline{z})$,
hence $\psi(z)\in\R$ for $z\in\R$.
\end{proof}
Note that this lemma remains true for $c=1/4$.

\begin{comment}
$c=0.1\in(\frac{-3}{4},\frac{1}{4})$
\end{comment}

\begin{comment}
We explain the proof in the easier setting in which $f$ is inside
the main cardioid, e.g. $c\in\left(-\frac{3}{4},\frac{1}{4}\right)$. 
\end{comment}

\begin{comment}
We suppress the index $c$ at times. 
\end{comment}

As before, we have:
\begin{lemma}
There is a family $\{\eta_{z}:z\in\mathcal{J}(f)\}$ of journeys
with the following properties:

1. Every $\eta_{z}$ is a journey $\left(\sigma_{0},\sigma_{1},\ldots\right)$
from the central station $\sigma_{0}=s_{0,0,c}$ to $\lim\sigma_{k}=z$.

2. There are no two consecutive peripheral tracks in $\eta_{z}$.

3. $\mathrm{Length}(\sigma_{k})\lesssim\theta^{-k}$ uniformly in
$z$, for a constant $\theta=\theta(c)>1$.

4. The journeys are invariant under $f$, in the sense that $f(\eta_{z})=\eta{}_{f(z)}\cup f\left([2,4]\right)$
for every $z\in\partial\D$.
\end{lemma}

\begin{proof}
Let $z\in\mathcal{J}\left(f\right)$. Let $\zeta=\psi^{-1}(z)$
be the corresponding point on $\partial\D$, %
\begin{comment}
because $\psi$ extends to a homeomorphism in the hyperbolic case.
\end{comment}

which has a central journey $\eta_{\zeta}$ in $\D^{*}$. We choose
the central journey of $z$ to be $\eta_{z}=f\left(\eta_{\zeta}\right).$

We take the images of the corresponding journeys from the case of
$\D^{*}$. Parts $1,2,4$ are then automatic. We now check part (3). 

The map $f$ is pointwise expanding on the Julia set $\mathcal{J}(f)$,
so by compactness $f$ is uniformly expanding there, i.e. there
are a constant $\theta>1$ and a neighborhood $\mathcal{U}$ of $\mathcal{J}(f)$
on which $\left|f'\right|>\theta$. Since every journey is eventually
contained in $\mathcal{U}$, we have for $k\gg1$ that $\mathrm{Length}(f(\gamma))\geq\theta\mathrm{\cdot Length}\left(\gamma\right)$.
Thus the length of peripheral tracks decays exponentially at rate
$\theta$, and likewise for express tracks.%
\begin{comment}
Let $z=z_{1}=\exp(2\pi i\theta)\in\partial\D$. We choose the stations
$\sigma_{i}$ inductively in pairs, in a greedy manner. In each step
we drive peripherally to the station closest to $z_{1}$ and then
drive to its radial successor. See Figure. 
\end{comment}
\end{proof}
%
To take advantage of the preceding lemma for showing quasiconvexity,
we use the following claim.

\begin{claim}Let\textbf{ $f$ }be a rational map that is expanding
on its Julia set $\mathcal{J}(f)$. Then there exists a constant $\epsilon$
such that for every two points $z,w\in\mathcal{J}(f)$, there exists
$n\in\mathbb{N}$ for which $\left|f^{\circ n}(z)-f^{\circ n}(w)\right|>\epsilon$.
\end{claim}
\begin{proof}
\emph{(sketch.)}\textbf{ }This claim follows from the condition that
$\left|f'\right|>1$ on $\mathcal{J}$ since as long as two points
$z,w\in\mathcal{J}$ are close enough we can approximate $f$ linearly
to see that the images $f(z),f(w)$ must be further apart. 
\end{proof}
\begin{comment}
We may alternatively see this claim as a degenerate version of the
principle of conformal elevator, which we briefly recall.
\end{comment}

\begin{remark}
This claim has some similarity to the \emph{principle of the conformal
elevator}, which we now recall.

A rational map is said to be \emph{hyperbolic }if every critical point
converges to an attracting cycle, and no critical point is on the
Julia set.

\emph{The principle of the conformal elevator: }Let $f$ be a hyperbolic
map, let $\zeta\in\mathcal{J}(f)$ and let $r>0$ be sufficiently
small. Then there exists an iterate $f^{\circ n}(B(\zeta,r))$ of
the ball $B(\zeta,r)$ which is a set of diameter bounded below uniformly
in $\zeta,r$ and which is "almost round".
Since we do not need this control on the distortion of the balls,
we do not state the precise form of this latter constraint and refer
the reader to {[}2{]} for details.
\end{remark}

\begin{comment}
Since we only need to make a pair of points on the Julia set a fixed
distance apart.
\end{comment}

\begin{comment}
In this case we have available the conformal elevator principle, which
allows enlarging every ball centered at $\mathcal{J}(f)$ to a
definite size by iteratively applying $f$. 
\end{comment}

\begin{theorem}
The domain $\mathcal{J}^{\text{exterior}}(f)$ is quasiconvex.
\end{theorem}

\begin{proof}
Let $z_{1},z_{2}$ be two points on $\mathcal{J}(f)$. We construct
a quasiconformality certificate curve connecting $z_{1}$ and $z_{2}$.
We use the obvious candidate: let $\zeta_{i}=\psi^{-1}\left(z_{i}\right)\in\partial\D$,
then we have a quasiconformality certificate for them $\eta_{\zeta_{1},\zeta_{2}}$
from the $c=0$ case. We choose $\eta_{z_{1},z_{2},c}=\psi(\eta_{\zeta_{1},\zeta_{2}})$
to be the certificates. By the invariance of the construction, this is
a journey on the $f$-Carleson decomposition which can similarly
be described directly in terms of a common ancestor in the tree structure,
since $\psi$ is a bijective correspondence between the two decompositions.
Since we already know that the lengths of tracks in the journey decay
exponentially, with rate $\theta>1$, the same proof of the case $c=0$
also shows quasiconvexity in this case.
\end{proof}
%
We give a second proof, relying on the previous claim on separation
of points under iteration. This proof will better prepare us to the
parabolic $c=1/4$ case, in which we don't have uniform expansion
of $f$ on its Julia set.
\begin{proof}
By the claim, there exists some $\epsilon$ such that any two points
are $\epsilon$-apart under some iteration of $f$. Let $z_{1},z_{2}\in\mathcal{J}(f)$.
If $\left|z_{0}-z_{1}\right|\geq\epsilon$ then there is nothing to
prove, since we may just concatenate $\eta_{z_{1}}$ and $\eta_{z_{2}}$
and absorb this bounded length into the quasiconformality constant
$A$. Explicitly, if $\mathrm{Length}\left(\eta_{z}\right)\leq L$
for all $z\in\mathcal{J}$ then we take $A\geq\frac{2L}{\epsilon}$
and then automatically $\mathrm{Length}\left(\eta_{z_{1}}+\eta_{z_{2}}\right)\leq A\left|z_{0}-z_{1}\right|$.

If, on the other hand, $\left|z_{0}-z_{1}\right|<\epsilon$, then
we may use the claim to find an iterate $f^{\circ n}$ such that $\left|f^{\circ n}(z_{0})-f^{\circ n}(z_{1})\right|\geq\epsilon$.
Then there is a certificate journey $\eta_{f^{\circ n}(z_{0}),f^{\circ n}(z_{1}),c}$
between them, and we take the certificate $\eta_{z_{0},z_{1}}$ between
the original points to be the component of $f^{\circ-n}\left(\eta_{f^{\circ n}(z_{0}),f^{\circ n}(z_{1}),c}\right)$
that connects the points $z_{0},z_{1}$.

A distortion estimate: 
\[
\mathrm{Length}\left(\eta_{z_{0},z_{1},c}\right)\asymp\frac{\mathrm{Length}\left(\eta_{f^{\circ n}(z_{0}),f^{\circ n}(z_{1}),c}\right)}{\left|\left(f^{\circ n}\right)'\left(\zeta\right)\right|}
\]
 for some point $\zeta$ on $\mathcal{J}$. The denominator grows
with $n$ exponentially at rate $\theta$, while the numerator has
a bound of the form 
\[
\mathrm{Length}\left(\eta_{f^{\circ n}(z_{0}),f^{\circ n}(z_{1}),c}\right)\lesssim\left|f^{\circ n}(z_{0})-f^{\circ n}(z_{1})\right|\lesssim\theta^{n}\left|z_{0}-z_{1}\right|
\]
 so altogether 
\[
\mathrm{Length}\left(\eta_{z_{0},z_{1},c}\right)\lesssim\frac{\theta^{n}\left|z_{0}-z_{1}\right|}{\theta^{n}}=\left|z_{0}-z_{1}\right|
\]
 so $\eta_{z_{0},z_{1},c}$ is a quasiconformality certificate.

\begin{comment}
By compactness of $\mathcal{J}$, the derivative $\left|\left(f^{\circ n}\right)'\left(\zeta\right)\right|$
is uniformly bounded for a fixed $n$. Moreover, this distortion estimate
shows that the length of the curve scales with $n$ at exactly the
same rate in which the distance between the iterated images $f^{\circ n}(z_{0}),f^{\circ n}(z_{1})$
scale with $n$, so the curve $\gamma$ has a uniformly bounded quasiconvexity
constant, as needed.
\end{comment}
\begin{comment}
take the quasiconformality constant $A$ to be large enough so that
the sum of the lengths $\mathrm{Length}\left(\eta_{z_{1}}\right)+\mathrm{Length}\left(\eta_{z_{2}}\right)\leq A\left|z_{0}-z_{1}\right|$
\end{comment}
\begin{comment}
If $ $ then the condition of uniform quasiconvexity on the paths
connecting the points $z_{0}$ and $z_{1}$ is just having bounded
length, so we may concatenate the paths $\gamma_{z_{0},2}$ and $\gamma_{z_{1},2}$
which exist by part (a).
\end{comment}
\begin{comment}
In general, the points $z_{0},z_{1}$ may be close to one another,
but we use the expanding dynamics of $f$ on the Julia set to
make them far apart: There exists some iterate $f^{\circ n}$
such that $\left|f^{\circ n}\left(z_{0}\right)-f^{\circ n}\left(z_{1}\right)\right|\geq0.1$.
Then we have a path of uniformly bounded length $\gamma{}_{z_{0},z_{1},n}$
connecting them in $\D^{*}$. Let $\gamma$ be the component of $f^{\circ\left(-n\right)}(\gamma{}_{z_{0},z_{1},n})$
containing $z_{0},z_{1}$. 
\end{comment}
\begin{comment}

\section{The Cauliflower Case}

\subsection{All points on the boundary are accessible}

Let $J$\textbf{ }be the julia set of $f(z)=z^{2}+\frac{1}{4}$. Denote
the filled Julia set by $K$.

The point $z=\frac{1}{2}$ will be denoted $p$.

\textbf{Claim. }The external ray which lands at the main cusp $p$
is a straight line.

\textbf{Proof. }There is a symmetry around the real line.$\square$

Let $\delta_{p}$ be the line segment lying on the real line which
joins the main cusp $p$ with $\gamma_{p,0}$.

By the previous claim, $\delta_{p}$ is a geodesic of the basin of
infinity.

Define $\delta_{q}$ for any cusp $q=f^{-n}(p)$ by taking the connected
component of the preimage $f^{-n}(\delta_{q})$ which contains $q$.

This is again a geodesic which lies on an external ray.

\textbf{Observation. }Every $\delta_{q}$ is a rectifiable curve.

\textbf{Proof. }The inverse image of a rectifiable curve under a holomorphic
mapping is rectifiable.

\textbf{Claim. }Every cusp $q$ of the Julia set is accessible.

\textbf{Proof. $\delta_{q}$ }is rectifiable, starts at $q$ and thus
it is enough to observe that the other endpoint of $\delta_{q}$ is
accessible.

\textbf{Claim. }Every point $\zeta\in J=\partial K$ is accessible.

\textbf{Idea. }We already showed this in the case when $\zeta$ is
a cusp, and cusps are the “hardest to reach"{} points
on the Julia set. Given a non-cusp point $\zeta\in J$, we consider
a nearby cusp $q$ and connect $\zeta$ with the curve $\delta_{q}$
of $q$. For making the connection we introduce "highways"{}
in the next section.


\subsection{Railways}

\subsubsection*{Preamble}

Let $a,b$ be two points on $J$ to be chosen later.

Let $\gamma_{p,0}$ be a geodesic between $a,b$ with respect to the
inner metric of the basin of infinity $\mathbb{C}\setminus K$.

\textbf{Claim. }$J$ is a Jordan curve.

Let $\gamma_{a,b}$ be the curve connecting $a,b$ along $J$ “in
the same direction as $\gamma_{p,0}$".

Let $U_{p,0}$ to be the domain whose boundary is the union of $\gamma_{p,0}$
and the $\gamma_{a,b}$.

Define inductively $\gamma_{p,j}$ to be the component of the preimage
of $\gamma_{p,j-1}$ that is inside $U_{p,0}$.

\textbf{Claim. }There is a unique such preimage.

Define $U_{p,j}$ to be the domain whose boundary is the union of
$\gamma_{p,j}$ and the $\gamma_{a,b}$.

\textbf{Claim. }Every cusp point $q$ is a preimage $f^{\circ(-n)}(p)$
of the main cusp $p=\frac{1}{2}$, for a unique positive integer $n$.

Given a cusp $q=f^{\circ\left(-n\right)}\left(p\right)$, define $U_{q,j}=f^{\circ(-n)}U_{p,j}$.

\textbf{Choice of $a,b$. }We choose the initial geodesic $\gamma_{p,0}$
so that the union over all cusps $\cup_{q}U_{q,0}$ contains a one-sided
neighborhood of the Julia set $J$.

That is, there is $\epsilon$ such that the union contains all points
in $K$ whose distance to the boundary $J$ is at most $\epsilon$.

Let $T_{p,j}$ be the region $U_{p,j}\setminus U_{p,j-1}$.

\subsubsection*{Usage}

\textbf{Strategy. }Given a point $\zeta\in J$, find $j$ such that
$\zeta\in\partial T_{p,j}$. 

Define the size of a cusp $q=f^{-n}(p)$ to be $n$. We use this to
compare the size of different cusps.

Let $p_{n}$ be a sequence of cusps converging to $\zeta$.

Construct a curve from $\zeta$ that verifies the accessibility of
$\zeta$ as follows.

The curve starts along $\delta_{p}$, until the point $a_{p,q_{1}}$
on $\delta_{p}$ closest to $\delta_{q_{1}}$.

Add a linear segment from $a_{p,q_{1}}$ to $a_{\ensuremath{q_{1},p}}$,
then continue along $\delta_{q_{1},q_{2}}$ until $a_{q_{1},q_{2}}$.
Iterating this process, we obtain a curve $\gamma_{\zeta}$ .

\textbf{Claim. }The curve $\gamma_{\zeta}$ has finite length.

To show this, we need the following estimates.

\subsubsection*{Diameter Comparisons}

Denote by $\diam\left(A\right)$ the Euclidean diameter of a set $A$,
and by $i.\diam\left(A\right)$ the inner diameter on the escaping
set, i.e. the diameter induced by the Riemann mapping from the escaping
set to the disk.

\textbf{Claim. }$\diam T_{p,n}\asymp i.\diam T_{p,n}$.

\textbf{Claim. }$\diam U_{p,n}\asymp i.\diam U_{p,n}$.

\textbf{Proof. }Koebe distortion theorem.

\section{Accessibility in the Hyperbolic Case}

Consider the polynomial $f(z)=z^{2}+c$ for some positive constant
$c<\frac{1}{4}$.

This polynomial is hyperbolic: its attracting fixed point $z_{a}$
is in the interior of the filled Julia set, and thus the conformal
elevator principle applies.

We first recall the principle.

\textbf{Proposition. }The post-critical set has positive distance
from the Julia set.

\textbf{Proof. }Since \textbf{$z_{a}$ }is an attracting fixed point,\textbf{
}it must have a critical point converging to it. Since $0$ is the
unique critical point of the polynomial $p_{c}$, we deduce that $0$
is attracted to $z_{a}$. Thus the forward iterations of $0$ stay
a bounded distance away from the Julia set, since $z_{a}$ is with
positive distance to the Julia set. $\square$

\textbf{Observation. }Let $R>0$ be the distance from the post-critical
set to the Julia set. Consider the ball $B=B(z,\frac{R}{2})$.

Since $B$ is disjoint from the post-critical set, none of its preimages
$f^{-n}\left(B\right)$ contains the critical point $0$.

Thus $f$ restricted on each such preimage is univalent and 2-to-1.

By Koebe's distortion theorem, applied to iterates $f^{\circ n}$,
the diameter of the preimages of $B$ changes by at most a multiplicative
constant.

This principle shows that all points $z$ on the Julia set $J(p_{c})$
are accessible, since after repeated applications of $f^{n}(B)$ we
get a topological ball of definite size which we may use to construct
an explicit path exiting $K$ in a rectifiable way.

\section{Accessibility in the Parabolic Case}

Let $f(z)=z^{2}+\frac{1}{4}$. Let $J=J(f)$ be the Julia set of $f$.
Let $K$ be the filled Julia set, known as the cauliflower.

\textbf{Claim. }The Cauliflower is bounded.

\textbf{Proof. }As in the hyperbolic case.$\square$

Using claim 1, fix $R>0$ for which $B(0,R)\supset K$. Denote $B=B(0,R)$.

Let $\zeta\in J$. Our goal is to show that $\zeta$ is accessible.

This is immediately equaivalent to showing that there is a rectifiable
curve $\Gamma$ starting at $\zeta$ that exits $B$.

The basic strategy will be to investigate the geometry near the main
cusp $p=\frac{1}{4}$, since all other cusps are preimages of $p$
under $f$,

whence the principle of the conformal elevator reduces the general
case to the case of a point sufficiently close to the main cusp $p$.

To implement this strategy, we take some arbitrary other point on
the Julia set, connect it by a curve $\gamma$ to $a$, then consider
all preimages of $\gamma$.

In general, consecutive images near a parabolic point have consecutive
distances comparable to $\frac{1}{n^{2}}$ , where $n$ is the index
of the preimage.
\end{comment}
\end{proof}
\begin{thebibliography}{1}
\bibitem[1]{key-1}Hakobyan, Hrant, and Herron, David A.. "Euclidean
quasiconvexity.."{} Annales Academiae Scientiarum Fennicae.
Mathematica 33.1 (2008): 205-230.

\bibitem[2]{key-2}Mario Bonk, Mikhail Lyubich, Sergei Merenkov, Quasisymmetries
of Sierpiński carpet Julia sets, Advances in Mathematics, Volume 301,
2016, Pages 383-422, ISSN 0001-8708, https://doi.org/10.1016/j.aim.2016.06.007.
\end{thebibliography}

\end{document}
