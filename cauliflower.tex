\begin{figure}
    \centering
    
    \begin{tikzpicture}[scale=3]
    
    % Define the iteration function f(z) = z^2 + 1/4
    \pgfmathdeclarefunction{f}{2}{%
        \pgfmathparse{((#1)^2+(#2)+(1/4))}%
    }
    
    % Set the iteration depth and the scale factor
    \def\iterationDepth{50}
    \def\scaleFactor{2}
    
    % Set the bounding box
    \path[use as bounding box] (-\scaleFactor,-\scaleFactor) rectangle (\scaleFactor,\scaleFactor);
    
    % Draw the grid
    \draw[step=0.2cm,gray,very thin] (-\scaleFactor,-\scaleFactor) grid (\scaleFactor,\scaleFactor);
    
    % Draw the axes
    \draw[thick,->] (-\scaleFactor,0) -- (\scaleFactor,0) node[below right] {$\Re(z)$};
    \draw[thick,->] (0,-\scaleFactor) -- (0,\scaleFactor) node[above left] {$\Im(z)$};
    
    % Define a function to determine whether a point is in the Julia set
    \pgfmathdeclarefunction{julia}{2}{%
        \pgfmathparse{0}%
        \let\zr=#1
        \let\zi=#2
        \foreach \i in {1,...,\iterationDepth}{
            \pgfmathsetmacro{\zrn}{\zr*\zr-\zi*\zi+1/4}
            \pgfmathsetmacro{\zin}{2*\zr*\zi}
            \pgfmathsetmacro{\zabs}{sqrt(\zrn*\zrn+\zin*\zin)}
            \ifdim \zabs pt > 2pt
                \pgfmathparse{\i-1+\pgfmathfloatifflags{\zabs}{inf}{0}{0}}
                \breakforeach
            \else
                \let\zr=\zrn
                \let\zi=\zin
            \fi
        }
    }
    
    % Set the color map for the Julia set
    \colorlet{color min}{white}
    \colorlet{color max}{black}
    \colorlet{color map}{black}
    
    % Draw the Julia set
    \foreach \x in {-2,-1.975,...,2}
        \foreach \y in {-2,-1.975,...,2} {
            \pgfmathsetmacro{\juliaValue}{julia(\x,\y)}
            \pgfmathsetmacro{\hue}{100-\juliaValue*100/\iterationDepth}
            \definecolor{mycolor}{rgb}{\hue,\hue,\hue}
            \fill[mycolor] (\x,\y) circle (0.02cm);
        }
        
    \end{tikzpicture}
    
    \caption{The Julia set of $z^2+\frac{1}{4}$.}
    \end{figure}


% \begin{figure}
%     \centering
    
%     \begin{tikzpicture}[scale=3]
%     \def\iterationDepth{10}
%     \def\scaleFactor{2}
%     \def\xmin{-\scaleFactor}
%     \def\xmax{\scaleFactor}
%     \def\ymin{-\scaleFactor}
%     \def\ymax{\scaleFactor}
%     \def\c{0.25}
%     \def\zmax{100}

% % Draw the complex plane
% \draw [->] (\xmin,0) -- (\xmax,0) node [right] {$\Re(z)$};
% \draw [->] (0,\ymin) -- (0,\ymax) node [above] {$\Im(z)$};

% % Define the function f(z) = z^2 + c
% \pgfmathdeclarefunction{f}{2}{%
%     \pgfmathparse{((#1)^2 + #2)}%
% }

% % Define the inverse function f^(-1)(z) = sqrt(z-c)
% \pgfmathdeclarefunction{finv}{2}{%
%     \pgfmathparse{sqrt(#1 - #2)}%
% }

% \pgfmathsetmacro{\xstep}{(\xmax-\xmin)/200}
% \pgfmathsetmacro{\ystep}{(\ymax-\ymin)/200}

% % Draw the Julia set
% \foreach \x in {\xmin,\xmin+\xstep,...,\xmax}{
% \foreach \y in {\ymin,\ymin+\ystep,...,\ymax}{
% \pgfmathsetmacro{\cx}{\x}
% \pgfmathsetmacro{\cy}{\y}
% \foreach \i in {1,...,\iterationDepth}{
% % \pgfmathsetmacro{\zx}{f(\cx,\cy)}
% % \pgfmathsetmacro{\zy}{2*\cx*\cy}
% % \pgfmathsetmacro{\cy}{\zy+\c}
% % \pgfmathsetmacro{\cx}{\zx}
% }
% }
% }

% \end{tikzpicture}
% \label{fig:cauliflower_tikz}
% \caption{The cauliflower, which is the Julia set of $z^2+\frac{1}{4}$, plotted in TikZ.}
% \end{figure}




% \begin{figure}[h]
% \centering
% \begin{tikzpicture}[scale=2]
% \def\maxIter{100}
% \def\maxMod{2}

% \foreach \re in {-1.5,...,1.5}
% {
% \foreach \im in {-1.5,...,1.5}
% {
% \pgfmathsetmacro{\cx}{\re}
% \pgfmathsetmacro{\cy}{\im}
% \pgfmathsetmacro{\xn}{\cx}
% \pgfmathsetmacro{\yn}{\cy}
% \pgfmathsetmacro{\i}{0}
% \pgfmathsetmacro{\done}{0}
% \loop
% \pgfmathsetmacro{\xnpone}{\xn*\xn-\yn*\yn+\cx+0.25}
% \pgfmathsetmacro{\ynpone}{2*\xn*\yn+\cy}
% \pgfmathsetmacro{\modulus}{sqrt(\xnpone*\xnpone+\ynpone*\ynpone)}
% \pgfmathsetmacro{\i}{\i+1}
% \ifnum\i>\maxIter
% \pgfmathsetmacro{\done}{1}
% \fi
% \ifdim\modulus pt>\maxMod pt
% \pgfmathsetmacro{\done}{1}
% \fi
% \pgfmathsetmacro{\xn}{\xnpone}
% \pgfmathsetmacro{\yn}{\ynpone}
% \ifnum\done=0
% \repeat
% \ifnum\i>\maxIter
% \else
% \fill[black] (\cx,\cy) circle (0.03);
% \fi
% }
% }
% \end{tikzpicture}
% \label{fig:cauliflower_tikz}
% \caption{The cauliflower, which is the Julia set of $z^2+\frac{1}{4}$, plotted in TikZ.}
% \end{figure}
