\documentclass[hebrew,english]{article}
\usepackage{amssymb}
\usepackage{fontspec}
\setmainfont[Ligatures=TeX]{David}
\setmonofont{Miriam Fixed}

\makeatletter

%%%%%%%%%%%%%%%%%%%%%%%%%%%%%% LyX specific LaTeX commands.
\providecommand\textquotedblplain{%
  \bgroup\addfontfeatures{RawFeature=-tlig}\char34\egroup}

\makeatother

\usepackage{polyglossia}
\setdefaultlanguage[variant=american]{english}
\setotherlanguage{hebrew}
\begin{document}
\begin{hebrew}%
\global\long\def\R{\mathbb{R}}%
\global\long\def\C{\mathbb{C}}%
\global\long\def\N{\mathbb{N}}%
\global\long\def\Z{\mathbb{Z}}%
\global\long\def\D{\mathbb{D}}%
\global\long\def\H{\mathbb{H}}%
\global\long\def\Im{\mathrm{Im}}%
\global\long\def\Re{\mathrm{Re}}%
\global\long\def\Arg{\mathrm{Arg}}%
\global\long\def\do#1#2{\frac{\partial#1}{\partial#2}}%
\global\long\def\nor#1{\left|#1\right|}%
\global\long\def\o#1{\overline{#1}}%
\global\long\def\norm#1{\left\Vert #1\right\Vert }%
\global\long\def\no{\emptyset}%
\global\long\def\ep{\varepsilon}%
\global\long\def\mp#1{\left\langle #1\right\rangle }%
\global\long\def\lap{\triangle}%
\global\long\def\dz{\mathrm{\mathrm{dz}}}%
\global\long\def\dt{\mathrm{\mathrm{dt}}}%
\global\long\def\dtheta{\mathrm{\mathrm{d\theta}}}%
\global\long\def\dx{\mathrm{\mathrm{dx}}}%
\global\long\def\dw{\mathrm{\mathrm{dw}}}%
\global\long\def\res{\mathrm{\mathrm{res}}}%
\global\long\def\ind{\mathrm{\mathrm{ind}}}%
\global\long\def\rad{\mathrm{\mathrm{rad}}}%
\global\long\def\S{\mathrm{\mathbb{S}}}%
\global\long\def\post{\mathrm{\mathrm{post}}}%
\global\long\def\lcm{\mathrm{\mathrm{lcm}}}%
\global\long\def\T{\mathrm{\mathbb{T}}}%
\global\long\def\diam{\mathrm{\mathrm{diam}}}%

\end{hebrew}%
\begin{abstract}
We show that the exterior of the cauliflower Julia set is quasiconvex.
\end{abstract}
Let $X$ be a metric space. A rectifiable path $\gamma:\left[0,1\right]\to X$
is $C$\textbf{-quasiconvex, }for some constant $C$, if its length
$\ell\left(\gamma\right)$ is at most $C$ times the distance between
its endpoints:
\[
\ell(\gamma)\leq C\cdot\mathrm{dist}\left(\gamma(0),\gamma(1)\right).
\]

The space $X$ is called \textbf{quasiconvex }if there is a constant
$C$ such that each pair of points can be joined by a $C$-quasiconvex
path. 

Any quasidisk is quasiconvex with respect to the Euclidean metric.
Thus, for example, the Koch snowflake has quasiconvex interior and
exterior.

Consider the quadratic polynomials $f_{c}(z)=z^{2}+c$. If $f_{c}$
has an attracting fixed point then its Julia set $J(f_{c})$ is a
quasicircle, hence its interior and exterior are both quasiconvex.
This is the case for values of $c$ in the main cardioid of the Mandelbrot
set, i.e. for 
\[
c\in\left\{ -\frac{\lambda}{2}-\frac{\lambda^{2}}{4}:\,\left|\lambda\right|<1\right\} .
\]

A natural question, then, is what happens for $c$ on the boundary
of the main cardioid. Are the interior and exterior of the Julia set
$J(f_{c})$ quasiconvex? We concentrate on the case $c=1/4$, known
as the cauliflower.

In {[}Hakobyan, Herron EUCLIDEAN QUASICONVEXITY{]} it is proved that
a John disk is a quasidisk if and only if it is quasiconvex. 

Since the interior of $J(f_{1/4})$ is a John domain, it follows that
it is not quasiconvex. However, in {[}Hakobyan, Herron EUCLIDEAN QUASICONVEXITY{]}
there is an example of a John disk whose complement is not quasiconvex.
Thus the question remains whether the exterior of $J(f_{1/4})$ is
quasiconvex or not. We will prove that the exterior is quasiconvex,
by a hands-on construction of quasiconvex paths. To do this we construct
two collections of curves, the so-called \textquotedbl express\textquotedbl{}
and \textquotedbl suburban\textquotedbl{} tracks, see figure {[}{]}.
We use the suggestive metaphor of a train travelling between the endpoints
and switching between tracks. We stitch the quasiconvex paths from
these tracks, using the suburban tracks near the endpoints. We make
repeated use of the principle of the conformal elevator in constructing
the tracks.

To check quasiconvexity, we will only need to connect points that
lie on the boundary:

\textbf{Lemma 1. }Let $U\subset\C$ be a domain. The closure $\overline{U}$
is quasiconvex if and only if there exists a constant $c$ such that
any two points $z_{1},z_{2}\in\partial U$ can be joined by a $c$-quasiconvex
path.

\textbf{Proof.} ...

The construction of the highways is done first in the case of the
exterior unit disk ($c=0$), and then transported to the nontrivial
$c=1/4$ case using the Böttcher linearization. 

Thus we first explain the method in the case of $c=0$. For clarity
of exposition, we then show how the method works in the case of $c=0.1$,
and only then prove the case $c=1/4$. As previously noted, the first
two cases are already known to be quasiconvex in virtue of being quasidisks.


\subsection{}

\section{The exterior of a disk}

The exterior $\D^{*}=\left\{ \left|z\right|>1\right\} $ of the unit
disk is trivially quasiconvex by connecting points along the perimeter
of the circle. However, these paths follow the boundary too closely
and their length would blow up upon transportation to the case of
$f_{c}$. Instead, we construct the curves using the classical notion
of Carleson boxes, which behave well under conjugation of a general
mapping $f_{c}$ on its basin of infinity to $z^{2}$ on the exterior
unit disk.

We show that the exterior $\D^{*}=\left\{ \left|z\right|>1\right\} $
of the unit disk is quasiconvex.  

Points outside the convex hull of $K$ can be joined by a straight
line, so we may disregard them in the construction. Thus in all three
cases of $c=0,\,0.1,\,1/4$ we may concentrate on the disk of radius
$4$ around the origin.

We consider two families of paths: 

1. \emph{Suburban tracks }are the circles $\delta_{n}=\left\{ \left|z\right|=2^{-n}\right\} $for
integers $n\geq-1$.

2. \emph{Express tracks} are the iterated preimages of the interval
$\left[1,4\right]$ under the map $f_{0}(z)=z^{2}$. These are radial
curves ending at the unit circle. Explicitly, for each dyadic direction
$\theta=\exp\left(2^{-k}\pi i\right)\in\partial\D$, where $k\geq0$
is an integer, we have the suburban track 
\[
\gamma_{\theta}=\left\{ r\theta:\,r\in\left[1,2^{2-k}\right]\right\} .
\]

From this construction we quickly conclude quasiconvexity:

\textbf{Theorem 1. }

a. Any point on $\partial\D$ can be joined to the point $4$ by a
curve of length $O(1)$ that is contained in $\D^{*}$.

b. The domain $\D^{*}$ is quasiconvex.

\textbf{Proof.}

a. Starting from the point $z_{0}=4$, we want to reach a given point
$z_{1}=\exp(2\pi i\theta)\in\partial\D$. To do that, we travel along
the suburban tracks to iteratively approximate the required angle
$\theta$, 

travel along the express and suburban tracks.

\section{A hyperbolic Example: $c=0.1$}

We explain the proof in the easier setting of $c=0.1$, for which
the conformal elevator enlarges balls to a definite size without a
parabolic \textquotedbl trap\textquotedbl .

\textbf{Theorem 2.}

Let $K_{0.1}$ be the filled Julia set of the polynomial $f_{0.1}=z^{2}+0.1$,
and denote by $K_{0.1}^{\mathrm{ext}}$ its closed exterior.

a. Any point on $\partial K_{0.1}$ can be joined to the point $4$
by a curve of length $O(1)$ that is contained in $K_{0.1}^{\mathrm{ext}}$.

b. The domain $K_{0.1}^{\mathrm{ext}}$ is quasiconvex.

\section{Proof Structure}

\subsection{All points on the boundary are accessible}

Let $J$\textbf{ }be the julia set of $f(z)=z^{2}+\frac{1}{4}$. Denote
the filled Julia set by $K$.

The point $z=\frac{1}{2}$ will be denoted $p$.

\textbf{Claim. }The external ray which lands at the main cusp $p$
is a straight line.

\textbf{Proof. }There is a symmetry around the real line.$\square$

Let $\delta_{p}$ be the line segment lying on the real line which
joins the main cusp $p$ with $\gamma_{p,0}$.

By the previous claim, $\delta_{p}$ is a geodesic of the basin of
infinity.

Define $\delta_{q}$ for any cusp $q=f^{-n}(p)$ by taking the connected
component of the preimage $f^{-n}(\delta_{q})$ which contains $q$. 

This is again a geodesic which lies on an external ray.

\textbf{Observation. }Every $\delta_{q}$ is a rectifiable curve.

\textbf{Proof. }The inverse image of a rectifiable curve under a holomorphic
mapping is rectifiable. (No distortion theorem needed.)

\textbf{Claim. }Every cusp $q$ of the Julia set is accessible.

\textbf{Proof. $\delta_{q}$ }is rectifiable, starts at $q$ and thus
it is enough to observe that the other endpoint of $\delta_{q}$ is
accessible.

\textbf{Claim. }Every point $\zeta\in J=\partial K$ is accessible.

\textbf{Idea. }We already showed this in the case when $\zeta$ is
a cusp, and cusps are the “hardest to reach\textquotedbl{} points
on the Julia set. Given a non-cusp point $\zeta\in J$, we consider
a nearby cusp $q$ and connect $\zeta$ with the curve $\delta_{q}$
of $q$. For making the connection we introduce \textquotedbl highways\textquotedbl{}
in the next section.


\subsection{Highways}

\subsubsection*{Preamble}

Let $a,b$ be two points on $J$ to be chosen later. 

Let $\gamma_{p,0}$ be a geodesic between $a,b$ with respect to the
inner metric of the basin of infinity $\mathbb{C}\setminus K$.

\textbf{Claim. }$J$ is a Jordan curve.

Let $\gamma_{a,b}$ be the curve connecting $a,b$ along $J$ “in
the same direction as $\gamma_{p,0}$\textquotedbl .

Let $U_{p,0}$ to be the domain whose boundary is the union of $\gamma_{p,0}$
and the $\gamma_{a,b}$.

Define inductively $\gamma_{p,j}$ to be the component of the preimage
of $\gamma_{p,j-1}$ that is inside $U_{p,0}$.

\textbf{Claim. }There is a unique such preimage.

Define $U_{p,j}$ to be the domain whose boundary is the union of
$\gamma_{p,j}$ and the $\gamma_{a,b}$.

\textbf{Claim. }Every cusp point $q$ is a preimage $f^{\circ(-n)}(p)$
of the main cusp $p=\frac{1}{2}$, for a unique positive integer $n$.

Given a cusp $q=f^{\circ\left(-n\right)}\left(p\right)$, define $U_{q,j}=f^{\circ(-n)}U_{p,j}$.

\textbf{Choice of $a,b$. }We choose the initial geodesic $\gamma_{p,0}$
so that the union over all cusps $\cup_{q}U_{q,0}$ contains a one-sided
neighborhood of the Julia set $J$. 

That is, there is $\epsilon$ such that the union contains all points
in $K$ whose distance to the boundary $J$ is at most $\epsilon$.

Let $T_{p,j}$ be the region $U_{p,j}\setminus U_{p,j-1}$.

\subsubsection*{Usage}

\textbf{Strategy. }Given a point $\zeta\in J$, find $j$ such that
$\zeta\in\partial T_{p,j}$. 

Define the size of a cusp $q=f^{-n}(p)$ to be $n$. We use this to
compare the size of different cusps.

Let $p_{n}$ be a sequence of cusps converging to $\zeta$.

Construct a curve from $\zeta$ that verifies the accessibility of
$\zeta$ as follows. 

The curve starts along $\delta_{p}$, until the point $a_{p,q_{1}}$
on $\delta_{p}$ closest to $\delta_{q_{1}}$. 

Add a linear segment from $a_{p,q_{1}}$ to $a_{\ensuremath{q_{1},p}}$,
then continue along $\delta_{q_{1},q_{2}}$ until $a_{q_{1},q_{2}}$.
Iterating this process, we obtain a curve $\gamma_{\zeta}$ .

\textbf{Claim. }The curve $\gamma_{\zeta}$ has finite length.

To show this, we need the following estimates.

\subsubsection*{Diameter Comparisons}

Denote by $\diam\left(A\right)$ the Euclidean diameter of a set $A$,
and by $i.\diam\left(A\right)$ the inner diameter on the escaping
set, i.e. the diameter induced by the Riemann mapping from the escaping
set to the disk.

\textbf{Claim. }$\diam T_{p,n}\asymp i.\diam T_{p,n}$.

\textbf{Claim. }$\diam U_{p,n}\asymp i.\diam U_{p,n}$.

\textbf{Proof. }Koebe distortion theorem.

\section{Accessibility in the Hyperbolic Case}

Consider the polynomial $f(z)=z^{2}+c$ for some positive constant
$c<\frac{1}{4}$.

This polynomial is hyperbolic: its attracting fixed point $z_{a}$
is in the interior of the filled Julia set, and thus the conformal
elevator principle applies.

We first recall the principle.

\textbf{Proposition. }The post-critical set has positive distance
from the Julia set.

\textbf{Proof. }Since \textbf{$z_{a}$ }is an attracting fixed point,\textbf{
}it must have a critical point converging to it. Since $0$ is the
unique critical point of the polynomial $p_{c}$, we deduce that $0$
is attracted to $z_{a}$. Thus the forward iterations of $0$ stay
a bounded distance away from the Julia set, since $z_{a}$ is with
positive distance to the Julia set. $\square$

\textbf{Observation. }Let $R>0$ be the distance from the post-critical
set to the Julia set. Consider the ball $B=B(z,\frac{R}{2})$. 

Since $B$ is disjoint from the post-critical set, none of its preimages
$f^{-n}\left(B\right)$ contains the critical point $0$. 

Thus $f$ restricted on each such preimage is univalent and 2-to-1.

By Koebe's distortion theorem, applied to iterates $f^{\circ n}$,
the diameter of the preimages of $B$ changes by at most a multiplicative
constant.

This principle shows that all points $z$ on the Julia set $J(p_{c})$
are accessible, since after repeated applications of $f^{n}(B)$ we
get a topological ball of definite size? which we may use to construct
an explicit path exiting $K$ in a rectifiable way.

\section{Accessibility in the Parabolic Case}

Let $f(z)=z^{2}+\frac{1}{4}$. Let $J=J(f)$ be the Julia set of $f$.
Let $K$ be the filled Julia set, also called the cauliflower.

\textbf{Claim. }The Cauliflower is bounded.

\textbf{Proof. }As in the hyperbolic case.$\square$

Using claim 1, fix $R>0$ for which $B(0,R)\supset K$. Denote $B=B(0,R)$.

Let $\zeta\in J$. Our goal is to show that $\zeta$ is accessible. 

This is immediately equaivalent to showing that there is a rectifiable
curve $\Gamma$ starting at $\zeta$ that exits $B$.

The basic strategy will be to investigate the geometry near the main
cusp $p=\frac{1}{4}$, since all other cusps are preimages of $p$
under $f$,

whence the principle of the conformal elevator reduces the general
case to the case of a point sufficiently close to the main cusp $p$.

To implement this strategy, we take some arbitrary other point on
the Julia set, connect it by a curve $\gamma$ to $a$, then consider
all preimages of $\gamma$.

In general, consecutive images near a parabolic point have consecutive
distances comparable to $\frac{1}{n^{2}}$ , where $n$ is the index
of the preimage.
\end{document}
