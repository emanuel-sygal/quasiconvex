\begin{abstract}
\emph{The basin of attraction to infinity} of a polynomial $f$
is the set of complex numbers $z \in \mathbb C$ for which $f^{\circ n}(z) \to \infty$.

This work proves that when $f(z)=z^2+1/4$, the basin of attraction is quasiconvex. 
%This means that any two points can be joined 

% In this work, it is proved that the basin of attraction $A_\infty (f)$

% When $f: z \mapsto z^2+c$ is a quadratic polynomial with $-\frac 34 < c < \frac 14$, the basin $\mathcal A_{\infty}(f)$ is known to be 
% a \emph{quasidisk}, the image of a round disk under a quasiconformal map.

% A Jordan domain $D$ is a quasidisk if and only if the following condition holds:
% for every two points $z_1,z_2 \in \partial D \setminus \{\infty\}$, and for each component $\gamma$ of $\partial D \setminus \{z_1,z_2\}$, we have 
% \begin{equation*}
%     \operatorname{diam}(\gamma) \lesssim |z_1-z_2|.
% \end{equation*}

% When $c=\frac 14$, the basin of attraction of $f$ is not a quasidisk. However, this work proves that it is \emph{quasiconvex}:

% A Jordan domain $D$ is called quasiconvex if any two points $z_1,z_2 \in \partial D$ can be joined by a curve $\gamma$ in $D$ for which 
% \begin{equation}
%     \operatorname{Length(\gamma)} \lesssim |z_1-z_2|.
% \end{equation}

\end{abstract}
